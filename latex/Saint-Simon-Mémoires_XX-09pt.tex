\PassOptionsToPackage{unicode=true}{hyperref} % options for packages loaded elsewhere
\PassOptionsToPackage{hyphens}{url}
%
\documentclass[oneside,9pt,french,]{extbook} % cjns1989 - 27112019 - added the oneside option: so that the text jumps left & right when reading on a tablet/ereader
\usepackage{lmodern}
\usepackage{amssymb,amsmath}
\usepackage{ifxetex,ifluatex}
\usepackage{fixltx2e} % provides \textsubscript
\ifnum 0\ifxetex 1\fi\ifluatex 1\fi=0 % if pdftex
  \usepackage[T1]{fontenc}
  \usepackage[utf8]{inputenc}
  \usepackage{textcomp} % provides euro and other symbols
\else % if luatex or xelatex
  \usepackage{unicode-math}
  \defaultfontfeatures{Ligatures=TeX,Scale=MatchLowercase}
%   \setmainfont[]{EBGaramond-Regular}
    \setmainfont[Numbers={OldStyle,Proportional}]{EBGaramond-Regular}      % cjns1989 - 20191129 - old style numbers 
\fi
% use upquote if available, for straight quotes in verbatim environments
\IfFileExists{upquote.sty}{\usepackage{upquote}}{}
% use microtype if available
\IfFileExists{microtype.sty}{%
\usepackage[]{microtype}
\UseMicrotypeSet[protrusion]{basicmath} % disable protrusion for tt fonts
}{}
\usepackage{hyperref}
\hypersetup{
            pdftitle={SAINT-SIMON},
            pdfauthor={Mémoires XX},
            pdfborder={0 0 0},
            breaklinks=true}
\urlstyle{same}  % don't use monospace font for urls
\usepackage[papersize={4.80 in, 6.40  in},left=.5 in,right=.5 in]{geometry}
\setlength{\emergencystretch}{3em}  % prevent overfull lines
\providecommand{\tightlist}{%
  \setlength{\itemsep}{0pt}\setlength{\parskip}{0pt}}
\setcounter{secnumdepth}{0}

% set default figure placement to htbp
\makeatletter
\def\fps@figure{htbp}
\makeatother

\usepackage{ragged2e}
\usepackage{epigraph}
\renewcommand{\textflush}{flushepinormal}

\usepackage{indentfirst}
\usepackage{relsize}

\usepackage{fancyhdr}
\pagestyle{fancy}
\fancyhf{}
\fancyhead[R]{\thepage}
\renewcommand{\headrulewidth}{0pt}
\usepackage{quoting}
\usepackage{ragged2e}

\newlength\mylen
\settowidth\mylen{...................}

\usepackage{stackengine}
\usepackage{graphicx}
\def\asterism{\par\vspace{1em}{\centering\scalebox{.9}{%
  \stackon[-0.6pt]{\bfseries*~*}{\bfseries*}}\par}\vspace{.8em}\par}

\usepackage{titlesec}
\titleformat{\chapter}[display]
  {\normalfont\bfseries\filcenter}{}{0pt}{\Large}
\titleformat{\section}[display]
  {\normalfont\bfseries\filcenter}{}{0pt}{\Large}
\titleformat{\subsection}[display]
  {\normalfont\bfseries\filcenter}{}{0pt}{\Large}

\setcounter{secnumdepth}{1}
\ifnum 0\ifxetex 1\fi\ifluatex 1\fi=0 % if pdftex
  \usepackage[shorthands=off,main=french]{babel}
\else
  % load polyglossia as late as possible as it *could* call bidi if RTL lang (e.g. Hebrew or Arabic)
%   \usepackage{polyglossia}
%   \setmainlanguage[]{french}
%   \usepackage[french]{babel} % cjns1989 - 1.43 version of polyglossia on this system does not allow disabling the autospacing feature
\fi

\title{SAINT-SIMON}
\author{Mémoires XX}
\date{}

\begin{document}
\maketitle

\hypertarget{chapitre-premier.}{%
\chapter{CHAPITRE PREMIER.}\label{chapitre-premier.}}

1723

~

{\textsc{Mort de la duchesse d'Aumont (Guiscard).}} {\textsc{- Mort et
caractère de l'abbé Fleury.}} {\textsc{- Mort du duc d'Estrées\,; du
comte de Saillant.}} {\textsc{- Marquis d'Alègre gouverneur des
Trois-Évêchés.}} {\textsc{- Mort de la comtesse de Châtillon\,; de
l'abbé de Camps\,; du P. Daubenton à Madrid.}} {\textsc{- Le P. Bermudez
confesseur du roi d'Espagne\,; son caractère.}} {\textsc{- Mort du
cardinal Dubois.}} {\textsc{- Ses richesses.}} {\textsc{- Ses
obsèques.}} {\textsc{- Son esquisse.}} {\textsc{- Sa conduite à
s'emparer de M. le duc d'Orléans.}} {\textsc{- Ses négociations à
Hanovre et en Angleterre, et son énorme grandeur.}} {\textsc{- Sa
négociation en Espagne\,; causes de sa facilité.}} {\textsc{- Son
gouvernement.}} {\textsc{- Ses folles incartades.}} {\textsc{- M. le duc
d'Orléans, fort soulagé par la mort du cardinal Dubois, est fait premier
ministre.}} {\textsc{- Le roi l'aimait, et point du tout le cardinal
Dubois.}}

~

Plusieurs personnes moururent en ce même temps\,:

La duchesse d'Aumont, fille unique et héritière de Guiscard, à
trente-cinq ans, d'une longue maladie de poitrine, le 9 juillet\,;

L'abbé Fleury, sous-précepteur des enfants de France, qui avait été
premier confesseur du roi, célèbre par son \emph{Catéchisme historique},
par d'autres ouvrages, surtout par son \emph{Histoire de l'Église},
qu'il n'a pu conduire au delà du concile de Constance, et par les
excellents discours qu'il a mis à la tête de chaque volume, en manière
de préfaces, respectable par sa modestie, par sa retraite au milieu de
la cour, par une piété sincère, éclairée, toujours soutenue, une douceur
et une conversation charmante, et un désintéressement peu commun. Il
n'avait que le prieuré d'Argenteuil, près de Paris, et n'avait jamais
voulu plus d'un bénéfice, quoiqu'il eût fort peu d'ailleurs. Il avait
quatre-vingt-trois ans, avec la tête entière, et vivait depuis longtemps
dans la plus parfaite retraite\,;

Le duc d'Estrées à quarante ans. Il était fils unique du dernier duc
d'Estrées et petit-fils du duc d'Estrées, mort ambassadeur à Rome.
C'était un homme qui avait passé sa vie dans la plus basse et la plus
honteuse crapule, et qui n'était pas sans esprit, mais sans aucun
sentiment, et qui s'était ruiné. Il ne laissa point d'enfants de la
fille du duc de Nevers qu'il avait épousée, et sa dignité de duc et pair
passa au maréchal d'Estrées, cousin germain de son père, fils des deux
frères\,;

Le comte de Saillant, lieutenant général et lieutenant-colonel du
régiment des gardes françaises, gouverneur et commandant des
Trois-Évêchés\footnote{Toul, Metz et Verdun formaient, dans l'ancienne
  monarchie, un gouvernement particulier.}. C'était un homme de qualité
fort brave et fort honnête homme, mais court à l'excès, que Harlay,
intendant de Metz, avait désolé tant qu'il y fut, et qui, pour s'en
divertir, l'avait fait tomber dans les panneaux les plus ridicules. Le
marquis d'Alègre eut le gouvernement des Trois-Évêchés sans y aller
commander\,;

La comtesse de Châtillon, dont le mari est depuis devenu duc et pair et
tant d'autres choses. Elle n'avait que trente-un ans. Elle était fille
du feu chancelier Voysin, et ne laissa qu'une fille qui {[}a{]} été
depuis duchesse de Rohan-Chabot\,;

L'abbé de Camps, à quatre-vingt-trois ans, si connu par sa fortune et
par sa littérature, dont il a été parlé ailleurs amplement ici\,;

Le P. Daubenton, confesseur du roi d'Espagne, au noviciat des jésuites
de Madrid, où il fut enterré en grande pompe, et fort peu regretté. Il
mourut le 7 août, à soixante-seize ans. L'incartade que lui fit le
cardinal Dubois, qui a été racontée ici il n'y a pas longtemps, et sa
cause coûta cher à la France. Daubenton, jésuite français, avait
toujours gardé de grandes mesures avec notre cour\,; mais outré contre
le cardinal Dubois, il voulut le faire repentir de l'insulte qu'il en
avait si mal à propos reçue, et ne sut faire pis, se voyant mourir, que
de persuader au roi d'Espagne de prendre pour confesseur le P. Bermudez,
jésuite espagnol, qui fut nommé le lendemain de sa mort. Bermudez,
Espagnol jusque dans les moelles, haïssait la France et les Français,
était secrètement attaché à la maison d'Autriche et lié avec toute la
cabale italienne\,; maître jésuite d'ailleurs, qui avait été provincial
de la province de Tolède où est Madrid, de sorte qu'il ne se pouvait
faire un plus pernicieux choix pour les intérêts de la France, ainsi
qu'il y parut depuis en toutes occasions. Il était un des plus
ordinaires prédicateurs de la chapelle, où j'ai ouï très souvent ses
sermons sans en rien entendre, parce qu'ils étaient en espagnol\,; mais
le ton, le geste, le débit me parurent d'un grand prédicateur. On
prétendait assez publiquement qu'il prêchait de mot à mot les sermons du
P. Bourdaloue traduits en espagnol. Il ne pouvait mieux choisir\,; mais
les siens étaient plus courts. Il y a eu tant d'occasions de parler ici
du P. Daubenton que je ne crois pas avoir rien à y ajouter.

Le cardinal Dubois n'eut pas le plaisir d'apprendre sa mort. Il le
suivit trois jours après à Versailles. Il avait caché son mal tant qu'il
avait pu, mais sa cavalcade à la revue du roi l'avait aigri au point
qu'il ne put plus le dissimuler à ceux de qui il pouvait espérer du
secours. Il n'oublia rien cependant pour le dissimuler au monde\,; il
allait tant qu'il pouvait au conseil, faisait avertir les ambassadeurs
qu'il irait à Paris, et n'y allait point, et chez lui se rendait
invisible, et faisait des sorties épouvantables à quiconque s'avisait de
lui vouloir dire quelque chose dans sa chaise à porteur entre le vieux
château et le château neuf où il logeait, ou en entrant ou sortant de sa
chaise. Le samedi 7 août, il se trouva si mal que les chirurgiens et les
médecins lui déclarèrent qu'il lui fallait faire une opération qui était
très urgente, sans laquelle il ne pouvait espérer de vivre que fort peu
de jours, parce que l'abcès, ayant crevé dans la vessie le jour qu'il
avait monté à cheval, y mettrait la gangrène, si elle n'y était déjà,
par l'épanchement du pus, et lui dirent qu'il fallait le transporter
sur-le-champ à Versailles pour lui faire cette opération. Le trouble de
cette terrible annonce l'abattit si fort qu'il ne put être transporté en
litière de tout le lendemain dimanche 8\,; mais le lundi 9, il le fut à
cinq heures du matin.

Après l'avoir laissé un peu reposer, les médecins et les chirurgiens lui
proposèrent de recevoir les sacrements et de lui faire l'opération
aussitôt après. Cela ne fut pas reçu paisiblement\,; il n'était presque
point sorti de furie depuis le jour de la revue\,; elle avait encore
augmenté le samedi sur l'annonce de l'opération. Néanmoins, quelque
temps après, il envoya chercher un récollet de Versailles avec qui il
fut seul environ un quart d'heure. Un aussi grand homme de bien, et si
préparé, n'en avait pas besoin de davantage. C'est d'ailleurs le
privilège des dernières confessions des premiers ministres. Comme on
rentra dans sa chambre, on lui proposa de recevoir le viatique\,; il
s'écria que cela était bientôt dit, mais qu'il y avait un cérémonial
pour les cardinaux qu'il ne savait pas et qu'il fallait envoyer le
demander au cardinal de Bissy à Paris. Chacun se regarda et comprit
qu'il voulait tirer de longue\,; mais comme l'opération pressait, ils la
lui proposèrent sans attendre davantage. Il les envoya promener avec
fureur et n'en voulut plus ouïr parler.

La faculté, qui voyait le danger imminent du moindre retardement, le
manda à M. le duc d'Orléans, à Meudon, qui sur-le-champ vint à
Versailles dans la première voiture qu'il trouva sous sa main. Il
exhorta le cardinal à l'opération, puis demanda à la faculté s'il y
avait de la sûreté en la faisant. Les chirurgiens et les médecins
répondirent qu'ils ne pouvaient rien assurer là-dessus, mais bien que le
cardinal n'avait pas deux heures à vivre si on {[}ne{]} la lui faisait
tout à l'heure. M. le duc d'Orléans retourna au lit du malade et le pria
tant et si bien qu'il y consentit. L'opération se fit donc sur les cinq
heures, en cinq minutes, par La Peyronie, premier chirurgien du roi en
survivance de Maréchal, qui était présent avec Chirac et quelques autres
médecins et chirurgiens des plus célèbres. Le cardinal cria et tempêta
étrangement\,; M. le duc d'Orléans rentra dans la chambre aussitôt
après, où la faculté ne lui dissimula pas qu'à la nature de la plaie et
de ce qui en était sorti le malade n'en avait pas pour longtemps. En
{[}effet{]}, il mourut précisément vingt-quatre heures après, le mardi
10 août, à cinq heures du soir, grinçant les dents contre ses
chirurgiens et contre Chirac, auxquels il n'avait cessé de chanter
pouille.

On lui apporta pourtant l'extrême-onction. De communion il ne s'en parla
plus, ni d'aucun prêtre auprès de lui, et {[}il{]} finit ainsi sa vie
dans le plus grand désespoir et dans la rage de la quitter. Aussi la
fortune s'était-elle bien jouée de lui, se fit acheter chèrement et
longuement par toutes sortes de peines, de soins, de projets, de menées,
d'inquiétudes, de travaux et de tourments d'esprit, et se déploya enfin
sur lui par des torrents précipités de grandeurs, de puissance, de
richesses démesurées, pour ne l'en laisser jouir que quatre ans, dont je
mets l'époque à sa charge de secrétaire d'État et deux seulement si on
la met à son cardinalat et à son premier ministère, pour lui tout
arracher au plus riant et au plus complet de sa jouissance, à
soixante-six ans. Il mourut donc maître absolu de son maître, et moins
premier ministre qu'exerçant toute la plénitude et toute l'indépendance
de toute la puissance et de toute l'autorité royale\,; surintendant des
postes, cardinal, archevêque de Cambrai, avec sept abbayes, dont il fut
insatiable jusqu'à la fin, et avait commencé des ouvertures pour
s'emparer de celles de Cîteaux, de Prémontré et des autres chefs
d'ordre, et il fut avéré après qu'il recevait une pension d'Angleterre
de quarante mille livres sterling. J'ai eu la curiosité de rechercher
son revenu, et j'ai cru curieux de mettre ici ce que j'en ai trouvé, en
diminuant même celui des bénéfices, peur éviter toute enflure.

Cambrai

\begin{enumerate}
\def\labelenumi{\roman{enumi}.}
\setcounter{enumi}{53}
\tightlist
\item
  120 000
\end{enumerate}

Nogent sous Coucy

10 000

Saint-Just

10 000

Airvaux

12 000

Bourgueil

12 000

Bergues-Saint-Vinox

60 000

Saint-Bertin

80 000

Cercamp

20 000

324 000

Premier ministre

150 000

Les postes

100 000

250 000

La pension d'Angleterre, à 24 liv. la livre sterling

960 000

Ainsi en

Bénéfices

324 000

Premier ministre

150 000

Postes

100 000

Pension d'Angleterre

960 000

1 534 000

J'ai mis pareillement au rabais ce qu'il tirait de ses appointements de
premier ministre et des postes\,: je crois aussi qu'il avait vingt mille
livres du clergé comme cardinal, mais je n'ai pu le savoir avec
certitude. Ce qu'il avait eu et réalisé de Law était immense. Il s'en
était fort servi à Rome pour son cardinalat\,; mais il lui en était
resté un prodigieux argent comptant. Il avait une extrême quantité de la
plus belle vaisselle d'argent et de vermeil, et la plus admirablement
travaillée\,; des plus riches meubles, des plus rares bijoux de toute
sorte, des plus beaux et des plus rares attelages de tous pays, et des
plus somptueux équipages. Sa table était exquise et superbe en tout, et
il en faisait fort bien les honneurs, quoique extrêmement sobre et par
nature et par régime.

Sa place de précepteur de M. le duc d'Orléans lui avait procuré l'abbaye
de Nogent-sous-Coucy\,; le mariage de ce prince celle de Saint-Just\,;
ses premiers voyages d'Hanovre et d'Angleterre celle d'Airvaux et de
Bourgueil\,; les trois autres, sa toute-puissance. Quel monstre de
fortune et d'où parti\,! et comment et si rapidement précipité\,! C'est
bien littéralement à lui qu'on peut appliquer ce passage du psaume\,:

«\,J'ai passé, il n'était déjà plus, il n'en est rien resté\,; jusqu'à
ses traces étaient effacées.\,»

Vidi impium superexaltatum et elevatum sicut cedros Libani\,;

Et transivi, et ecce non erat, et non est inventas locus ejus.

(Ps. XXXVI, v. 35 et 36.)

Le mercredi au soir, lendemain de sa mort, il fut porté de Versailles à
Paris dans l'église du chapitre de Saint-Honoré, où il fut enterré
quelques jours après. Les académies dont il était lui firent faire
chacune un service où elles assistèrent, l'assemblée du clergé un autre
comme à leur président\,; et en qualité de premier ministre, il y en eut
un à Notre-Dame, où le cardinal de Noailles officia, et où les cours
supérieures assistèrent. Il n'y eut point d'oraison funèbre à aucun, on
n'osa le hasarder. Son frère, plus vieux que lui et honnête homme, qu'il
avait fait venir lorsqu'il fut secrétaire d'État, demeura avec la charge
de secrétaire du cabinet qu'il avait, et qu'il lui avait donnée, et les
ponts et chaussées qu'il lui procura à la mort de Beringhen, premier
écuyer, qui les avait, et qui s'en était très dignement acquitté. Ce
Dubois, qui était fort modeste, trouva un immense héritage. Il n'avait
qu'un fils, chanoine de Saint-Honoré qui n'avait jamais voulu ni places
ni bénéfices et qui vivait très saintement. Il ne voulut presque rien
toucher de cette riche succession. Il en employa une partie à faire à
son oncle une espèce de mausolée beau, mais modeste, plaqué contre la
muraille, au bas de l'église, où le cardinal est enterré avec une
inscription fort chrétienne, et distribua l'autre partie aux pauvres,
dans la crainte qu'elle ne lui portât malédiction.

On a bien des exemples de prodigieuse fortune, plusieurs même de gens de
peu, mais il n'y en a aucun de personne si destituée de tout talent qui
y porte et qui la soutienne que l'était le cardinal Dubois, si on en
excepte la basse et obscure intrigue. Son esprit était fort ordinaire,
son savoir des plus communs, sa capacité nulle, son extérieur d'un
furet, mais de cuistre, son débit désagréable, par articles, toujours
incertain, sa fausseté écrite sur son front, ses moeurs trop sans aucune
mesure pour pouvoir être cachées des fougues qui pouvaient passer pour
des accès de folie, sa tête incapable de contenir plus d'une affaire à
la fois, et lui d'y en mettre ni d'en suivre aucune que pour son intérêt
personnel\,: rien de sacré, nulle sorte de liaison respectée\,; mépris
déclaré de foi, de parole, d'honneur, de probité, de vérité\,: grande
estime et pratique continuelle de se faire un jeu de toutes ces
choses\,; voluptueux autant qu'ambitieux\,; voulant tout en tout genre,
se comptant lui seul pour tout, et tout ce qui n'était point lui pour
rien, et regardant comme la dernière démence de penser et d'agir
autrement. Avec cela, doux, bas, souple, louangeur, admirateur, prenant
toutes sortes de formes, avec la plus grande facilité, et revêtant
toutes sortes de personnages, et souvent contradictoires, pour arriver
aux différents buts qu'il se proposait et néanmoins très peu capable de
séduire. Son raisonnement par élans, par bouffées, entortillé même
involontairement, peu de sens et de justesse\,; le désagrément le
suivait partout. Néanmoins des pointes de vivacité plaisantes quand il
voulait qu'elles ne fussent que cela, et des narrations amusantes, mais
déparées par l'élocution qui aurait été bonne sans ce bégaiement dont sa
fausseté lui avait fait une habitude, par l'incertitude qu'il avait
toujours à répondre et à parler. Avec de tels défauts, il est peu
concevable que le seul homme qu'il ait su séduire ait été M. le duc
d'Orléans qui avait tant d'esprit, tant de justesse dans l'esprit, et
qui saisissait si promptement tout ce qui se pouvait connaître des
hommes. Il le gagna enfant, dans ses fonctions de précepteur\,; il s'en
empara jeune homme en favorisant son penchant pour la liberté, le faux
bel air, l'entraînement à la débauche, le mépris de toute règle\,; en
lui gâtant par les beaux principes des libertins savants le coeur,
l'esprit et la conduite, dont ce pauvre prince ne put jamais se
délivrer, non plus que des sentiments contraires de la raison, de la
vérité, de la conscience, qu'il prit toujours soin d'étouffer.

Dubois, insinué de la sorte, n'eut d'étude plus chère que de se
conserver bien par tous moyens avec son mettre à la faveur duquel tous
ses avantages étaient attachés, qui n'allaient pas loin alors, mais tels
qu'ils fussent, étaient bien considérables pour le valet du curé de
Saint-Eustache, puis de Saint-Laurent. Il ne perdit donc jamais de vue
son prince dont il connaissait tous les grands talents et tous les
grands défauts qu'il avait su mettre à profit, et qu'il y mettait tous
les jours, dont l'extrême faiblesse était le principal, et l'espérance
la mieux fondée de Dubois. Ce fut aussi celle qui soutint dans les
divers délaissements qu'il éprouva, et dans le plus fâcheux de tous, à
l'entrée de la régence, dont on a vu avec quel art il avait su se
rapprocher. C'était le seul talent où il fût maître, que celui de
l'intrigue obscure avec toutes ses dépendances. Il séduisit son maître
comme on l'a vu ici, par ces prestiges d'Angleterre qui firent tant de
mal à l'État, et dont les suites lui en causent encore de si fâcheux. Il
le força et tout de suite le lia à cet intérêt personnel, au cas de mort
du roi, de deux usurpateurs intéressés à se soutenir l'un l'autre, et M.
le duc d'Orléans s'y laissa entraîner par le babil de Canillac, les
profonds \emph{sproposito} du duc de Noailles, les insolences, les
grands airs de Stairs, qui lui imposaient, et cela sans aucun désir de
la couronne\,: c'est une vérité étrange que je ne puis trop répéter,
parce que je l'ai parfaitement et continuellement reconnue\,; et je dis
étrange, parce qu'il n'est pas moins vrai que si la couronne lui fût
échue et sans aucun embarras, même pour la recueillir et la conserver,
il s'en serait trouvé chargé, empêtré, embarrassé, sans comparaison
aucune, plus qu'il n'en aurait été satisfait.

De là, ce lien devenu nécessaire et intime entre lui et Dubois, quand
celui-ci fut parvenu à aller la première fois en Hollande, ce qui ne fut
pas sans peine, et qui le conduisit après à Hanovre, puis à Londres, et
à devenir seul maître de toute la négociation, partie l'arrachant à la
faiblesse de son maître, partie en l'infatuant qu'il ne s'y pouvait
servir de nul autre, parce que nul autre ne pouvait être comme lui
dépositaire du vrai noeud qui faisait le fondement secret de la
négociation, qui était, en cas de mort du roi, ce soutien réciproque des
deux usurpateurs, trop dangereux pour M. le duc d'Orléans à confier à
qui que ce soit qu'à lui, qui toutefois devait uniquement gouverner
toute la négociation, sans égard à tout autre intérêt de l'État le plus
marqué et le plus visible. Par là Dubois se mit en toute liberté de
traiter à Londres pour lui-même en accordant tout ce qu'il plut aux
Anglais, pour quoi il ne fallait pas grande habileté en négociations.
Aussi a-t-on vu plus d'une fois dans ce qui a été donné ici d'après
Torcy sur les affaires étrangères, que M. le duc d'Orléans ne
s'accommodait pas toujours de ce que Dubois voulait passer aux Anglais,
que ceux-ci lui reprochaient que son maître était plus difficile que
lui, et tacitement son peu de crédit, et lui faisaient sentir la
conséquence pour ce qu'il désirait personnellement d'eux, de pouvoir
davantage sur M. le duc d'Orléans et de l'amener à ce qui leur
convenait. De là ces lettres véhémentes dont M. le duc d'Orléans me
parlait quelquefois, et auxquelles il ne pouvait résister\,; de là son
brusque retour d'Angleterre, sans ordre ni préparatif, pour emporter par
sa présence ce que, pour cette fois, ses lettres n'avaient pu faire, et
son prompt passage à Londres, dès qu'il eut réussi à ce qu'il s'était
proposé, pour en aller triompher chez les ministres anglais, et leur
montrer par l'essai d'un court voyage ce qu'ils pouvaient attendre de
son ascendant sur le régent lorsqu'il serait à demeure à ses côtés, par
conséquent combien il leur serait nécessaire, et leur intérêt sensible
de le satisfaire personnellement, de façon qu'ils pussent compter sur
lui.

Voilà ce qui sans capacité aucune a conclu les traités que Dubois a
faits avec les Anglais, si opposés à l'intérêt de la France et au bien
de toute l'Europe, en particulier si préjudiciables à l'Espagne, et qui
d'un même tour de main a fondé et précipité la monstrueuse grandeur de
Dubois, qui, en revenant tout à fait d'Angleterre, culbuta les conseils
pour culbuter le maréchal d'Huxelles et le conseil des affaires
étrangères, et les mettre uniquement dans sa main, sous le titre de
secrétaire d'État. Outre la prétention d'une telle récompense de sa
négociation dont il sut faire valoir à son maure toute la délicatesse,
l'habileté et le fruit qu'il en tirait, tout nul qu'il fût, il lui
persuada encore la nécessité de ne confier qu'à lui seul les affaires
étrangères, pour entretenir et consolider l'intime confiance si
nécessaire à conserver avec les Anglais, et leur ôter les entraves du
maréchal d'Huxelles, de Canillac, de ce même conseil que Dubois voulait
déjà écarter, et que toutes les affaires ne passassent plus que par un
seul canal agréable au ministère anglais, dont il ne pût prendre aucune
défiance. De secrétaire d'État à tout le reste, le chemin fut rapide et
aisé\,; la guerre qu'il fit entreprendre contre l'Espagne sans la cause
la plus légère, pour ruiner leur marine au désir des Anglais, et contre
le plus sensible intérêt de la France, et le plus personnel de M. le duc
d'Orléans, fut le prix du chapeau, qui bientôt après le mena au premier
ministère.

Que si après avoir développé comment, sans capacité aucune, Dubois s'est
fait si grand par l'Angleterre, en lui sacrifiant la France, mais
beaucoup plus l'Espagne, on s'étonne comment si promptement après il est
venu à bout du double mariage, surtout avec les impressions personnelles
prises en Espagne contre M. le duc d'Orléans, dès avant sa régence et
depuis, ce point sera facile à démêler. Le roi d'Espagne, quelque
prévenu qu'il fût contre M. le duc d'Orléans par ce {[}que{]} la
princesse des Ursins lui imputa avant la mort du roi, quelques blessures
qu'il en eût reçues depuis la régence par le ministère de Dubois pour
plaire aux Anglais, jamais homme ne fut attaché à sa maison et à sa
nation originelle si intrinsèquement ni si indissolublement que Philippe
V. Cette passion, si vive en lui et toujours active, le rendait
infatigable à tout souffrir de la France sans cesser de désirer avec la
plus violente ardeur de se pouvoir lier et réunir indissolublement avec
elle. C'est ce qui lui fit recevoir l'espérance qui lui fut montrée,
puis aussitôt proposée du mariage du roi, comme le comble de ses voeux,
à quelque condition que ce pût être, en sorte que celle du mariage
actuel du prince des Asturies ne fut pas capable seulement de le
refroidir. D'un autre côté, la reine qui avait la même passion pour un
établissement sûr et solide de son fils aîné en Italie, et par
affection, et par vanité, et pour se retirer auprès de lui et éviter le
sort des reines veuves d'Espagne, qui avait toujours été le point de son
horreur, sentirent tous deux qu'ils n'y pouvaient parvenir malgré
l'empereur\,; qu'il n'y avait que le roi d'Angleterre, si parfaitement
bien alors avec la cour de Vienne, qui pût parvenir à lui faire donner
les mains à cet établissement, et que l'Espagne ne pouvait espérer
là-dessus aucun secours de l'Angleterre que par M. le duc d'Orléans,
même par l'abbé Dubois, au point où ils étaient avec Georges et avec ses
ministres. Ce ne fut donc pas merveilles si le double mariage fut conclu
si facilement et si promptement, en quoi toute l'habileté de l'abbé
Dubois ne fut que de l'imaginer et d'avoir la hardiesse de le proposer.
C'est ce que je vis très clairement en Espagne, et que l'esprit du roi
d'Espagne n'avait jamais été guéri sur M. le duc d'Orléans, ni sur son
ministre, ni celui de la reine non plus, à travers toutes les mesures et
les plus exactes réserves que, quelque soin qu'ils prissent, ils ne me
purent épaissir ce voile plus que la consistance d'une gaze, et je
sentis le même dans le marquis de Grimaldo. Telles furent les merveilles
de la prétendue capacité de Dubois.

Il n'en montra pas davantage dans sa manière de gouverner quand il fut
devenu le véritable maître. Toute son application tournée à ce que son
maître, dont il connaissait tout le glissant, ne lui échappât pas,
s'épuisa à épier tous les moments de ce prince, ce qu'il faisait, qui il
voyait, les temps qu'il donnait à chacun, son humeur, son visage, ses
propos à l'issue de chaque audience ou de chaque partie de plaisir\,;
qui en était, quels propos et par qui tenus, et à combiner toutes ces
choses\,; surtout à effrayer et à effaroucher pour empêcher qui que ce
fût d'être assez hardi pour aller droit au prince, et à rompre toutes
mesures à qui en avait la témérité sans en avoir obtenu son congé et son
aveu. Ce sont les espionnages qui occupaient toutes ses journées, sur
lesquels il réglait toutes ses démarches, et à tenir le monde, sans
exception, de si court, que tout ne fût que dans sa main, affaires,
grâces, jusqu'aux plus petites bagatelles, et à faire échouer tout ce
qui osait essayer de lui passer entre les doigts, et de ne le pas
pardonner aux essayeurs, qu'il poursuivait partout d'une façon
implacable. Cette application et quelque écorce indispensable d'ordres à
donner, ravissaient tout son temps, en sorte qu'il était devenu
inabordable, hors quelques audiences publiques ou quelques autres aux
ministres étrangers. Encore la plupart d'eux ne le pouvaient joindre, et
se trouvaient réduits à l'attendre aux passages sur des escaliers, et en
d'autres endroits par lesquels il dérobait son passage, où il ne
s'attendait pas à les rencontrer. Il jeta une fois dans le feu une
quantité prodigieuse de paquets de lettres toutes fermées, et de toutes
parts, puis s'écria d'aise qu'il se trouvait alors à son courant. À sa
mort il s'en trouva par milliers, tout cachetées.

Ainsi tout demeurait en arrière, en tout genre, sans que personne, même
des ministres étrangers, osât s'en plaindre à M. le duc d'Orléans, et
sans que ce prince, tout livré à ses plaisirs, et toujours sur le chemin
de Versailles à Paris, prit la peine d'y penser, bien satisfait de se
trouver dans cette liberté, et ayant toujours suffisamment de bagatelles
dans son portefeuille pour remplir son travail avec le roi, qui n'était
que de bons à lui faire mettre aux dépenses arrêtées, ou aux demandes
des emplois ou des bénéfices vacants. Ainsi aucune affaire n'était
presque décidée, et tout demeurait et tombait en chaos. Pour gouverner
de la sorte il n'est pas besoin de capacité. Deux mots à chaque ministre
chargé d'un département, et quelque légère attention à garnir les
conseils devant le roi des dépêches les moins importantes, brochant les
autres seul avec M. le duc d'Orléans, puis les laissant presque toutes
en arrière, faisaient tout le travail du premier ministère, et
l'espionnage, les avis de l'intérieur de M. le duc d'Orléans, les
combinaisons de ces choses, les parades, les adresses, les batteries,
faisaient et emportaient tout celui du premier ministre\,; ses
emportements pleins d'injures et d'ordures, dont ni hommes ni femmes, de
quelque rang et de quelque considération qu'ils fussent, {[}n'étaient{]}
à couvert, le délivraient d'une infinité d'audiences, parce qu'on aimait
mieux aller par des bricoles subalternes, ou laisser périr ses affaires,
que s'exposer à essuyer ces fureurs et ces affronts. On en a vu un
échantillon vague par ce qui a été raconté ici de ce qui arriva en
pleine et nombreuse audience d'ambassadeurs, prélats, dames et de toutes
sortes de gens considérables, à l'officier que j'avais dépêché de Madrid
avec le contrat de mariage du roi.

Les folies publiques du cardinal Dubois, depuis surtout que devenu le
maître il ne les contint plus, feraient un livre. Je n'en rapporterai
que quelques-unes pour échantillon. La fougue lui faisait faire
quelquefois le tour entier et redoublé d'une chambre courant sur les
tables et les chaises sans toucher du pied la terre, et M. le duc
d'Orléans m'a dit plusieurs fois en avoir été souvent témoin en bien des
occasions.

Le cardinal de Gesvres se vint plaindre à M. le duc d'Orléans de ce que
le cardinal Dubois venait de l'envoyer promener dans les termes les plus
sales. On a vu ailleurs qu'il en avait usé de même avec la princesse de
Montauban, et la réponse que M. le duc d'Orléans avait faite à ses
plaintes. La vérité est qu'elle ne méritait pas mieux. L'étonnant fut
qu'il dit de même à un homme des moeurs, de la gravité et de la dignité
du cardinal de Gesvres, qu'il avait toujours trouvé le cardinal Dubois
de bon conseil, et qu'il croyait qu'il ferait bien de suivre celui qu'il
lui venait de donner. C'était apparemment pour se défaire de pareilles
plaintes après un tel exemple\,; et en effet on ne lui en porta plus
depuis.

M\textsuperscript{me} de Cheverny, devenue veuve, s'était retirée
quelque temps après aux Incurables. Sa place de gouvernante des filles
de M. le duc d'Orléans avait été donnée à M\textsuperscript{me} de
Conflans. Un peu après le sacre, M\textsuperscript{me} la duchesse
d'Orléans lui demanda si elle avait été chez le cardinal Dubois,
là-dessus M\textsuperscript{me} de Conflans répondit que non, et qu'elle
ne voyait pas pourquoi elle irait, la place que Leurs Altesses Royales
lui avaient donnée étant si éloignée d'avoir trait à aucune affaire.
M\textsuperscript{me} la duchesse d'Orléans insista sur ce que le
cardinal était à l'égard de M. le duc d'Orléans. M\textsuperscript{me}
de Conflans se défendit, et finalement dit que c'était un fou qui
insultait tout le monde, et qu'elle ne voulait pas s'y exposer. Elle
avait de l'esprit et du bec, et souverainement glorieuse, quoique fort
polie. M\textsuperscript{me} la duchesse d'Orléans se mit à rire de sa
frayeur, et lui dit que n'ayant rien à lui demander ni à lui
représenter, mais seulement à lui rendre compte de l'emploi que M. le
duc d'Orléans lui avait donné, c'était une politesse qui ne pouvait que
plaire au cardinal, et lui en attirer de sa part, bien loin d'avoir rien
de désagréable à en craindre, et finit par lui dire que cela convenait
et qu'elle voulait qu'elle y allât.

La voilà donc partie, car c'était à Versailles, au sortir de dîner, et
arrivée dans un grand cabinet, où il y avait huit ou dix personnes qui
attendaient à parler au cardinal, qui était auprès de sa cheminée avec
une femme qu'il galvaudait\footnote{Galvauder se disait familièrement
  pour \emph{maltraiter de paroles, gourmander}.}. La peur en prit à
M\textsuperscript{me} de Conflans, qui était petite et qui en rapetissa
encore. Toutefois, elle s'approcha comme cette femme se retirait. Le
cardinal la voyant s'avancer lui demanda vivement ce qu'elle lui
voulait. «\, Monseigneur, dit-elle. --- Ho, monseigneur\,!
monseigneur\,! interrompit le cardinal\,; cela ne se peut pas. --- Mais,
monseigneur, reprit-elle. --- De par tous les diables, je vous le dis
encore, interrompit de nouveau le cardinal, quand je vous dis que cela
ne se peut pas. --- Monseigneur,\,» voulut encore dire
M\textsuperscript{me} de Conflans pour expliquer qu'elle ne demandait
rien\,; mais à ce mot le cardinal lui saisit les deux pointes des
épaules, la revire, la pousse du poing par le dos, et\,: «\,Allez à tous
les diables, dit-il, et me laissez en repos.\,» Elle pensa tomber toute
plate\,; et s'enfuit en furie, pleurant à chaudes larmes, et arrive en
cet état chez M\textsuperscript{me} la duchesse d'Orléans, à qui, à
travers ses sanglots, elle conte son aventure.

On était si accoutumé aux incartades du cardinal, et celle-là fut
trouvée si singulière et si plaisante que le récit en causa des éclats
de rire qui achevèrent d'outrer la pauvre Conflans, qui jura bien que de
sa vie elle ne remettrait le pied chez cet extravagant.

Le jour de Pâques d'après qu'il fut cardinal, il s'éveille sur les huit
heures et sonne à rompre ses sonnettes, et le voilà à blasphémer
horriblement après ses gens, à vomir mille ordures et mille injures, et
à crier à pleine tête de ce qu'ils ne l'avaient pas éveillé, qu'il
voulait dire la messe, qu'il ne savait plus où en prendre le temps avec
toutes les affaires qu'il avait. Ce qu'il fit de mieux après une si
belle préparation, ce fut de ne la dire pas, et je ne sais s'il l'a
jamais dite depuis son sacre.

Il avait pris pour secrétaire particulier un nommé Venier qu'il avait
défroqué de l'abbaye de Saint-Germain des Prés, où il était frère
convers, et en faisait les affaires depuis vingt ans avec beaucoup
d'esprit et d'intelligence. Il s'était fait promptement aux façons du
cardinal, et s'était mis sur le pied de lui dire tout ce qu'il lui
plaisait. Un matin qu'il était avec le cardinal, il demanda quelque
chose qui ne se trouva pas sous la main. Le voilà à jurer, à blasphémer,
à crier à pleine tête contre ses commis, et que s'il n'en avait pas
assez, il en prendrait vingt, trente, cinquante, cent, et à faire un
vacarme épouvantable. Venier l'écoutait tranquillement, le cardinal
l'interpella, si cela n'était pas une chose horrible, d'être si mal
servi, à la dépense qu'il y faisait, et à s'emporter tout de nouveau, et
à le presser de répondre. «\,Monseigneur, lui dit grenier, prenez un
seul commis de plus, et lui donnez pour emploi unique de jurer et de
tempêter pour vous, et tout ira bien, vous aurez beaucoup de temps de
reste, et vous vous trouverez bien servi.\,» Le cardinal se mit à rire
et s'apaisa.

Il mangeait tous les soirs un poulet pour tout souper et seul. Je ne
sais par quelle méprise ce poulet fut oublié un soir par ses gens. Comme
il fut près de se coucher, il s'avisa de son poulet, sonna, cria,
tempêta après ses gens, qui accoururent et qui l'écoutèrent froidement.
Le voilà à crier de plus belle après son poulet et après ses gens de le
servir si tard. Il fut bien étonné qu'ils lui répondirent tranquillement
qu'il avait mangé son poulet, mais que, s'il lui plaisait, ils en
allaient faire mettre un autre à la broche. «\, Comment, dit-il, j'ai
mangé mon poulet\,!» L'assertion hardie et froide de ses gens le
persuada, et ils se moquèrent de lui. Je n'en dirai pas davantage, parce
que, encore une fois, on en ferait un vrai volume. C'en est assez pour
montrer quel était ce monstrueux personnage dont la mort soulagea grands
et petits, et en vérité, toute l'Europe, enfin jusqu'à son frère même
qu'il traitait comme un nègre. Il voulut une fois chasser son écuyer
pour lui avoir prêté un de ses carrosses pour aller quelque part dans
Paris.

Le plus soulagé de tous fut M. le duc d'Orléans. Il gémissait en secret
depuis assez longtemps sous le poids d'une domination si dure, et sous
les chaînes qu'il s'était forgées. Non seulement il ne pouvait plus
disposer ni décider de rien, mais il exposait inutilement au cardinal ce
qu'il désirait qui fût sur grandes et petites choses. Il lui en fallait
passer sur toutes par la volonté du cardinal qui entrait en furie, en
reproches, et le pouillait comme un particulier, quand il lui arrivait
de le trop contredire. Le pauvre prince sentait aussi l'abandon où il
s'était livré, et par cet abandon, la puissance du cardinal et l'éclipse
de la sienne. Il le craignait, il lui était devenu insupportable, il
mourait d'envie de s'en débarrasser\,; cela se montrait en mille choses,
mais il n'osait, il ne savait comment s'y prendre, et isolé et sans
cesse épié comme il l'était, il n'avait personne avec qui s'en ouvrir
tout à fait, et le cardinal bien averti, en redoublait ses frasques pour
retenir par la frayeur ce que ses artifices avaient usurpé, et qu'il
n'espérait plus de se conserver par une autre voie.

Dès qu'il fut mort, M. le duc d'Orléans retourna à Meudon apprendre au
roi cette nouvelle, qui le pria aussitôt de se charger de toute la
conduite des affaires, le déclara premier ministre, et en reçut son
serment le lendemain, dont la patente tôt expédiée fut vérifiée au
parlement. Cette déclaration si prompte sur laquelle M. le duc d'Orléans
n'avait rien préparé, fut l'effet de la crainte qu'eut l'évêque de
Fréjus de voir un particulier premier ministre. Le roi aimait M. le duc
d'Orléans, comme on l'a déjà dit, par le respect qu'il en recevait, et
par sa manière de travailler avec lui, qui sans danger d'être pris au
mot, le laissait toujours le maître des grâces sur le choix des
personnes qu'il lui proposait, et d'ailleurs de ne l'ennuyer jamais, ni
de contraindre ses amusements par les heures de ce travail. Quelques
soins, quelques souplesses que le cardinal Dubois eût employées pour
gagner l'esprit du roi et l'apprivoiser avec lui, jamais il n'en avait
pu venir à bout, et on remarquait, même sans avoir de trop bons yeux,
une répugnance du roi pour lui plus que très sensible. Le cardinal en
était désolé, mais redoublait de jambes dans l'espérance de réussir à la
fin. Mais, outre l'air peu naturel et le désagrément inséparable de ses
manières les plus occupées à plaire, il avait deux ennemis auprès du
roi, bien attentifs à l'éloigner de prendre avec ce jeune prince, le
maréchal de Villeroy, tant qu'il y fut, mais bien plus dangereusement le
Fréjus, qui ne pou voit haïr le cardinal que d'ambition, {[}et qui{]}
bien résolu de le culbuter si M. le duc d'Orléans venait à manquer, pour
n'être ni primé, encore moins dominé par un particulier, n'avait garde
de ne pas le ruiner journellement dans l'esprit du roi, en s'y
établissant lui-même de plus en plus.

\hypertarget{chapitre-ii.}{%
\chapter{CHAPITRE II.}\label{chapitre-ii.}}

1723

~

{\textsc{Mort du premier président de Mesmes.}} {\textsc{- Je retrouve
et revois M. le duc d'Orléans comme auparavant.}} {\textsc{- Compagnie
d'Ostende.}} {\textsc{- Mort de La Houssaye\,; sa place de chancelier de
M. le duc d'Orléans donnée à Argenson, et les postes à Morville.}}
{\textsc{- Le mariage du prince et de la princesse des Asturies
consommé.}} {\textsc{- Mariage des deux fils du duc de Bouillon avec la
seconde fille du prince Jacques Sobieski, par la mort de l'aîné.}}
{\textsc{- Succès de ce mariage.}} {\textsc{- Inondation funeste à
Madrid, et incendie en même moment.}} {\textsc{- Nocé, Canillac et le
duc de Noailles rappelés.}} {\textsc{- Le premier bien dédommagé.}}
{\textsc{- Translation de l'évêque-duc de Laon à Cambrai\,; sa cause.}}
{\textsc{- Laon donné à La Fare, évêque de Viviers, au pieux refus de
Belsunce, évêque de Marseille.}} {\textsc{- Quel était ce nouvel évêque
de Laon.}} {\textsc{- Mort et caractère de Besons, archevêque de
Rouen.}} {\textsc{- Rouen donné à Tressan, évêque de Nantes\,; Besançon
à l'abbé de Monaco\,; Luçon à l'abbé de Bussy, etc.}} {\textsc{-
M\textsuperscript{me} de Chelles écrit fortement à M. le duc d'Orléans
sur ses choix aux prélatures.}} {\textsc{- Mort du prince de Croï.}}
{\textsc{- Absurdité de cette nouvelle chimère de princerie.}}
{\textsc{- Mort de la duchesse d'Aumont (Brouilly).}} {\textsc{- Mort du
jeune duc d'Aumont\,; sa dépouille.}} {\textsc{- Triste et volontaire
état de la santé de M. le duc d'Orléans.}} {\textsc{- J'avertis l'évêque
de Fréjus de l'état de M. le duc d'Orléans, et l'exhorte à prendre ses
mesures en conséquence.}} {\textsc{- Fausseté et politique de ce prélat,
qui veut se rendre le maître de tout à l'ombre d'un prince du sang,
premier ministre de nom et d'écorce.}} {\textsc{- Mort de La Chaise,
capitaine de la porte Torcy obtient cette charge pour son fils.}}
{\textsc{- Secondes charges de la cour, proie des enfants des
ministres.}} {\textsc{- Mort de Livry.}} {\textsc{- Mort du grand-duc de
Toscane\,; sa famille, son caractère.}} {\textsc{- Mort de l'électeur de
Cologne.}} {\textsc{- Mort et caractère de la maréchale de Chamilly.}}
{\textsc{- Mort de M\textsuperscript{me} de Montsoreau, femme du grand
prévôt.}}

~

Un plus corrompu, s'il se peut, que le cardinal Dubois le suivit douze
ou treize jours après\,: ce fut le premier président de Mesmes, qui,
déjà fort appesanti par quelques légères apoplexies, en eut une qui
l'emporta en moins de vingt-quatre heures, à soixante et un ans, sans
que pendant ce peu de temps on en eût pu tirer le moindre signe de vie.
Je dis plus corrompu que Dubois par ses profondes et insignes noirceurs,
et parce que, né dans un état honorable et riche, il n'avait pas eu
besoin de se bâtir une fortune comme Dubois, qui était de la lie du
peuple, non que ce pût être une excuse à celui-ci, mais une tentation de
moins à l'autre, qui n'avait qu'à jouir de ce qu'il était, avec honneur.
J'ai eu tant d'occasions de parler et de faire connaître ce magistrat
également détestable et méprisable, que je crois pouvoir me dispenser
d'en salir davantage ce papier. On a vu ailleurs pourquoi et comment on
m'avait enfin forcé à me raccommoder avec lui, après ce beau mariage du
duc de Lorge avec sa fille, dont il eut tout lieu de se bien repentir,
comme il l'avoua souvent lui-même. J'étais paisiblement à la Ferté en
bonne compagnie depuis près de deux mois, sans en avoir voulu partir sur
les courriers que Belle-Ile et d'autres encore m'avaient dépêchés sur la
mort du cardinal Dubois, pour me presser de revenir. La vanité et
l'avidité d'avoir une pension m'en fit dépêcher un autre à la mort du
premier président par ses filles, pour me conjurer de revenir et de la
demander à M. le duc d'Orléans.

Je cédai encore en cette occasion à la vertu et à la piété de
M\textsuperscript{me} de Saint-Simon, qui voulut si absolument que je ne
leur refusasse pas cet office, et je partis. Elle revint à Paris
quelques jours après moi. La cour était retournée de Meudon à Versailles
le 13 août, il y avait dix ou douze jours, et j'y trouvai M. le duc
d'Orléans.

Dès qu'il me vit entrer dans son cabinet, il courut à moi, et me demanda
avec empressement si je voulais l'abandonner. Je lui répondis que tant
que son cardinal avait vécu, je m'étais cru fort inutile auprès de
lui\,; et que j'en avais profité pour ma liberté et pour mon repos\,;
mais qu'à présent que cet obstacle à tout bien n'était plus, je serais
toujours à son très humble service. Il me fit promettre de vivre avec
lui comme auparavant, et, sans entrer en rien sur le cardinal, se mit
sur les affaires présentes, domestiques et étrangères, m'expliqua où il
en était, et me conta l'émoi que prenaient l'Angleterre et la Hollande
de la nouvelle compagnie d'Ostende, que l'empereur formait, qu'il
voulait maintenir et que ces deux puissances voulaient empêcher de
s'établir par leur grand intérêt du commerce, enfin celui que la France
y pouvait trouver pour et contre, et ses vues de conduite dans cette
affaire. Je le trouvai content, gai, et reprenant le travail avec
plaisir. Quand nous eûmes bien causé du dehors, du dedans et du roi,
dont il était fort content, je lui parlai de la pension que les filles
du premier président lui demandaient. Il se mit à rire et à se moquer
d'elles, après l'argent immense qu'il avait si souvent prodigué à leur
père, ou qu'il lui avait su escroquer, et à se moquer de moi d'être leur
avocat en chose si absurde après tout ce qu'il y avait eu entre moi et
leur père, duquel il fit fort bien et en peu de mots l'oraison funèbre.
J'avouerai franchement que je n'insistai pas beaucoup pour une chose que
je trouvais aussi déplacée, et dont je ne me souciais point du tout. Je
vécus donc de là en avant avec M. le duc d'Orléans comme j'avais
toujours fait avant que le cardinal Dubois fût premier ministre, et lui
avec toute son ancienne confiance. Il faut pourtant que je convienne que
je ne cherchai pas à en faire beaucoup d'usage. Il fit alors la très
légère perte de La Houssaye, son chancelier, qui avait montré son
ignorance dans la place de contrôleur général des finances qu'il avait
été obligé de quitter. Il avait soixante et un ans. M. le duc d'Orléans
prit à sa place le lieutenant de police, second fils du feu garde des
sceaux d'Argenson. J'oubliais de marquer que les postes avaient été
données à Morville, secrétaire d'État des affaires étrangères, avec une
grande et juste diminution d'appointements.

On apprit en ce même temps que Leurs Majestés Catholiques avaient mis le
prince et la princesse des Asturies ensemble, et que leur mariage avait
été consommé.

Le duc de Bouillon, fort occupé d'étayer de plus en plus sa princerie
par des alliances étrangères, dont les siens s'étaient si bien trouvés,
avisa d'en éblouir, ainsi que de ses grands établissements, le prince
Jacques Sobieski, fils aîné du célèbre roi de Pologne, qui vivait retiré
dans ses terres en Silésie\,; répandit beaucoup d'argent autour de lui,
et fit si bien que le mariage de sa seconde fille fut conclu avec le
prince de Turenne, fils aîné du duc de Bouillon et de la fille du feu
duc de La Trémoille, sa première femme.

Ce mariage flattait extrêmement le duc de Bouillon. Le grand-père de sa
future belle-fille avait occupé longtemps le trône de Pologne, et en
avait illustré la couronne par ses grandes actions\,; sa femme était
soeur de l'impératrice, épouse de l'empereur Léopold, et mère des
empereurs Joseph et Charles, et soeur aussi de la reine douairière
d'Espagne, de la feue reine de Portugal, des électeurs de Mayence et
Palatin, et de la duchesse de Parme, mère de la reine, seconde femme du
roi d'Espagne. Enfin, la fille aînée du prince Jacques Sobieski avait
épousé le roi d'Angleterre, retiré à Rome. Le mariage fut célébré par
procureur, à Neuss, en Silésie, et en personne à Strasbourg, un mois
après. Mais le prince de Turenne tomba malade presque aussitôt, et
mourut douze jours après son mariage. Personne de la famille n'était
allé à Strasbourg que son frère\,; la mariée y était arrivée en fort
léger équipage. On comptait l'amener tout de suite à Paris, quand la
maladie de son mari les arrêta. Dès que la nouvelle en vint, le duc de
Bouillon pensa aussitôt au mariage de son second fils, si elle devenait
veuve, et à tout événement dépêcha le comte d'Évreux à Strasbourg pour
lui persuader de continuer son voyage, dans l'espérance de gagner son
consentement. Ils y réussirent, et la gardèrent tantôt chez eux à
Pontoise, tantôt dans un couvent du lieu, et n'en laissèrent approcher
personne qui la pût imprudemment détromper des grandeurs qu'elle croyait
aller épouser. Ils négocièrent en Silésie pour avoir le consentement,
puis à Rome pour la dispense, où il n'est question que du plus ou du
moins d'argent qu'on n'avait pas dessein d'épargner. Enfin, le mariage
se fit en avril 1724, fort en particulier, à cause du récent veuvage.

Quand elle commença à voir le monde et à être présentée à la cour, elle
fut étrangement surprise de s'y trouver comme soutes les autres
duchesses et princesses assises, et de ne primer nulle part avec toute
la distinction dont on l'avait persuadée, en sorte qu'il lui échappa
plus d'une fois qu'elle avait compté épouser un souverain, et qu'il se
trouvait que son mari et son beau-père n'étaient que deux bourgeois du
quai Malaquais. Ce fut bien pis quand elle vit le roi marié. Je n'en
dirai pas davantage. Ces regrets, qu'elle ne cachait pas, joints à
d'autres mécontentements, en donnèrent beaucoup aux Bouillon. Le mariage
ne fut pas heureux. La princesse, qui ne put s'accoutumer à l'unisson
avec nos duchesses et princesses, encore moins à vivre avec les autres,
comme il fallait qu'elle s'y assujettit, se rendit solitaire et obscure.
Elle eut des enfants, et, après plusieurs années, ne pouvant plus tenir
dans une situation si forcée, elle obtint aisément d'aller faire un
voyage en Silésie pour ménager son père et ses intérêts auprès de lui.
Son mari ne demandait pas mieux que d'en être honnêtement défait. Il ne
la pressa point de revenir, et au bout de peu d'années elle mourut en
Silésie, au grand soulagement de M. de Bouillon, qui ne laissa pas d'en
recueillir assez gros pour ses enfants.

Ce fut en ce temps-ci qu'arriva cette subite inondation à Madrid, proche
du Buen-Retiro, où la duchesse de la Mirandole fut noyée dans son
oratoire, où le prince Pio et quelques autres périrent, et dont le duc
de La Mirandole, le duc de Liria, l'abbé Grimaldo et l'ambassadeur de
Venise se sauvèrent avec des peines infinies, tandis que la superbe
maison du duc et de la duchesse d'Ossone, magnifiquement meublée,
brûlait dans le haut de la ville, sans qu'on pût en arrêter l'incendie
faute d'eau. Je me suis étendu ailleurs ici par avance sur cet étrange
et funeste événement, ce qui m'empêchera d'en rien répéter ici.

Nocé, qui avait été rapproché dans son exil, fut rappelé. M. le duc
d'Orléans, qui l'avait toujours aimé et qui ne l'avait éloigné que
malgré lui, l'en dédommagea par un présent de cinquante mille livres en
argent, et deux mille écus de pension. Canillac revint bientôt après, et
enfin le duc de Noailles. On fit beaucoup de contes de ses amusements
pendant qu'il fut dans ses terres, et de l'édification qu'il avait voulu
donner à ses peuples, en chantant avec eux au lutrin et en y portant
chape, et aux processions. On voit ainsi que ce n'est pas sans raison
qu'on l'appelait\,: \emph{Omnis homo}.

M. le duc d'Orléans donna plusieurs grands bénéfices. L'évêque duc de
Laon, et qui en avait fait la fonction au sacre, n'avait pu se faire
recevoir pair de France au parlement. Sa mère était la comédienne
Florence, et M. le duc d'Orléans ne l'avait point reconnu. Ce fut
l'obstacle qu'on ne put vaincre, parce qu'il faut dire qui on est, et le
prouver. Dans cet embarras, il fut transféré, avec conservation du rang
et honneurs d'évêque, duc de Laon. II ne perdit pas au change, puisqu'il
eut l'archevêché de Cambrai. Son successeur à Laon surprit et scandalisa
étrangement\,: ce fut le frère de La Fare, qui ne lui ressemblait en
rien. C'était un misérable déshonoré par ses débauches et par son
escroquerie, que personne ne voulait voir ni regarder, et que M. le duc
d'Orléans, qui me l'a dit lui-même, chassa du Palais-Royal pour avoir
volé cinquante pistoles qu'il envoyait, par lui, à M\textsuperscript{me}
de Polignac. Je la nomme, parce que sa vie a été si publique que je ne
crois pas manquer à la charité, à la discrétion, à la considération de
son nom.

Ce bon ecclésiastique fut une fois chassé des Tuileries à coups de pied,
depuis le milieu de la grande allée jusque hors la porte du Pont-Royal,
par les mousquetaires et d'autres jeunes gens qui s'y attroupèrent, avec
des {[}clameurs{]} épouvantables, répétées parla foule des laquais
amassés à la porte. Enfin, et c'est un fait qui fut très public, les
deux capitaines des mousquetaires leur défendirent à l'ordre de le voir.
Pour sortir d'un état si pitoyable, ce rebut du monde fit le converti,
frappa à plusieurs portes pour être ordonné prêtre sans y pouvoir
réussir, à ce que me conta lors Rochebonne, évêque-comte de Noyon, qui
fut un de ceux qui le refusèrent, malgré une prétendue retraite qu'il
fit dans un bénéfice qu'il avait dans Noyon. Enfin il trouva un prélat
plus traitable par la conformité de conduite. J'aurais horreur de le
nommer et de dire avec quel scandale il l'ordonna contre toutes les
règles de l'Église. Incontinent après, il se jeta au cardinal de Bissy
et à Languet, évêque de Soissons, à qui tout était bon moyennant le
fanatisme de la constitution, qui le rendit digne d'être grand vicaire
de Soissons, où il se signala en ce genre à mériter toute leur
protection. Avec ce secours et celui des jésuites, il trafiqua l'évêché
de Viviers avec Ratabon, qui y avait passé du siège d'Ypres, et que
l'épiscopat ennuyait, malgré la non-résidence. Il lui donna deux abbayes
qu'il avait, avec un bon retour, et fut sacré évêque de Viviers, au
scandale universel.

L'évêque de Marseille, Belsunce, qui s'était fait un si grand nom
pendant la peste, était venu à Paris sur la maladie du duc de Lauzun,
frère de sa mère, qui avait toujours pris soin de lui et de ses frères.
Il fut nommé à l'évêché de Laon avec un grand applaudissement. Allant un
jour voir M. de Lauzun, qui s'était retiré dans le couvent des
Petits-Augustins, j'arrivai par un côté du cloître à la porte de sa
chambre, et ce prélat, par un autre côté, en même temps qu'on appelait
déjà M. de Laon. Je me rangeai pour le laisser passer devant moi. Il
sourit en me regardant, et me poussant de la main\,: «\,Allez, monsieur,
me dit-il, ce n'est pas la peine\,;» et malgré moi me fit passer devant
lui. À ce mot je compris qu'il n'accepterait point Laon et qu'il
demeurerait à Marseille\,; mais qu'il n'osait refuser du vivant de son
oncle qui l'aurait dévoré, et qui n'avait que peu de semaines à vivre.
En effet, dès qu'il fut mort, il refusa Laon avec un attachement pour
son siège qui n'était plus connu, mais qui lui fit un grand honneur. La
Fare, évêque de Viviers, qui n'était pas pour être si délicat, fut mis à
Laon, à son refus, où on a vu depuis ce qu'il savait faire. Il y est
mort abhorré et banqueroutier, après avoir de gré ou de force escroqué
tout son diocèse qu'il avait d'ailleurs dévasté.

Rouen vaquait par la mort de Besons, frère du maréchal qui y avait été
transféré de Bordeaux, duquel j'ai eu occasion de parler ici plus d'une
fois. C'était un homme fort sage, doux, mesuré, avec un air et une mine
brutale et grossière, délié, qui savait le monde et ses devoirs\,; fort
instruit, fort décent, et le premier homme du clergé\,; en capacité sur
ses affaires temporelles, de l'esprit fait exprès pour le gouvernement
des diocèses\,; aimé, respecté et amèrement regretté dans les trois
qu'il avait eus. Tressan, évêque de Nantes, premier aumônier de M. le
duc d'Orléans, eut Rouen, et fut chargé des économats\footnote{On
  appelait économat l'administration des revenus d'un bénéfice
  ecclésiastique pendant la vacance de ce bénéfice.} qu'avait Besons\,;
et l'abbé de Monaco, déjà vieux, eut Besançon, dont l'abbé de Mornay
n'avait pas eu le temps de jouir ni d'être sacré.

L'abbé de Bussy-Rabutin eut Luçon, et plusieurs autres évêchés furent
donnés et beaucoup d'abbayes. Celles de Bergues-Saint-Vinox et de
Saint-Bertin à Saint-Omer furent rendues à des moines\,; Dubois ne les
avait eues que comme cardinal. M. le duc d'Orléans reçut une lettre de
M\textsuperscript{me} de Chelles, sa fille, sur cette distribution, qui
l'effraya, et qu'il lut et relut pourtant deux fois. Elle était
admirable sur le choix des sujets et sur l'abus qu'il en faisait, et le
menaçait de la colère de Dieu qui l'en châtierait promptement. Il en fut
assez ému pour en parler, et même pour la laisser voir, mais je ne sais
s'il en eût profité. Il n'en eut pas le temps.

Le fils aîné du feu comte de Solre mourut dans ses terres, en Flandre,
où il s'était retiré depuis la mort de son père, et que sa femme l'avait
avisé de faire le prince. Il était lieutenant général et n'avait que
quarante-sept ans. J'ai parlé ailleurs de cette folie qui a passé à ses
enfants, que le comte de Solre, n'avait jamais imaginée, qui ne
prétendit jamais aucun rang, qui fut chevalier de l'ordre en 1688, parmi
les gentilshommes, et dont j'ai vu toute ma vie la femme et la fille
debout au souper et à la toilette, jusqu'à ce qu'elles s'en allèrent en
Espagne, comme je l'ai raconté. Croï est une terre en Boulonnais qui a
donné son nom à cette maison, que ses établissements en Flandre ont si
fort illustrée. J'en ai parlé ici ailleurs.

La duchesse d'Aumont mourut à Passy, près Paris, 23 octobre, près de
sept mois après son mari, quatre mois après sa belle-fille, huit jours
avant son fils. Elle était fille d'Antoine de Brouilly, marquis de
Piennes, chevalier des ordres du roi, et soeur de l'épouse du marquis de
Châtillon, premier gentilhomme de la chambre de Monsieur et chevalier
des ordres du roi. Elle fut aussi dame d'atours de Madame. C'étaient
deux beautés fort différentes\,: toutes deux grandes et parfaitement
bien faites\,; intimement liées ensemble\,; qui n'avaient point de
frères, et toutes deux épousées par amour. La duchesse d'Aumont s'était
retirée et barricadée à Passy contre la petite vérole dont Paris était
plein. Elle ne l'évita pas et en mourut.

Le duc d'Aumont, son fils, en mourut aussi huit jours après elle, à
trente-deux ans. Il était aimé et estimé dans le monde, très bien fait,
avec un beau visage, et fort bien avec les dames. Il ne laissa que deux
fils enfants, dont le cadet mourut bientôt après. Je m'intéressai fort
au partage de sa dépouille, pour le duc d'Humières qui eut le
gouvernement de Boulogne et Boulonnais, et son petit-neveu eut la charge
de son père de premier gentilhomme de la chambre du roi.

On m'avait rendu tout le château neuf de Meudon, tout meublé, depuis le
retour de la cour à Versailles, comme je l'avais avant qu'elle vint à
Meudon. Le duc et la duchesse d'Humières y étaient avec nous, et bonne
compagnie. Le duc d'Humières voulut que je le menasse à Versailles
remercier M. le duc d'Orléans le matin. Nous le trouvâmes qu'il allait
s'habiller, et qu'il était encore dans son caveau dont il avait fait sa
garde-robe. Il y était sur sa chaise percée parmi ses valets et deux ou
trois de ses premiers officiers. J'en fus effrayé. Je vis un homme la
tête basse, d'un rouge pourpre, avec un air hébété, qui ne me vit
seulement pas approcher. Ses gens le lui dirent. Il tourna la tête
lentement vers moi sans presque la lever, et me demanda d'une langue
épaisse ce qui m'amenait. Je le lui dis. J'étais entré là pour le
presser de venir dans le lieu où il s'habillait, pour ne pas faire
attendre le duc d'Humières\,; mais je demeurai si étonné que je restai
court. Je pris Simiane, premier gentilhomme de sa chambre, dans une
fenêtre, à qui je témoignai ma surprise et ma crainte de l'état où je
voyais M. le duc d'Orléans. Simiane me répondit qu'il était depuis fort
longtemps ainsi les matins, qu'il n'y avait ce jour-là rien
d'extraordinaire en lui, et que je n'en étais surpris que parce que je
ne le voyais jamais à ces heures-là\,; qu'il n'y paraîtrait plus tant,
quand il se serait secoué en s'habillant. Il ne laissa pas d'y paraître
encore beaucoup lorsqu'il vint s'habiller. Il reçut le remerciement du
duc d'Humières d'un air étonné et pesant\,; et lui qui était toujours
gracieux et poli à tout le monde, et qui savait si bien dire à propos et
à point, à peine lui répondit-il\,; un moment après, nous nous retirâmes
M. d'Humières et moi. Nous dînâmes chez le duc de Gesvres, qui le mena
faire son remerciement au roi.

Cet état de M. le duc d'Orléans me fit faire beaucoup de réflexions. Il
y avait fort longtemps que les secrétaires d'État m'avaient dit que,
dans les premières heures des matinées, ils lui auraient fait passer
tout ce qu'ils auraient voulu, et signé tout ce qui lui eût été de plus
préjudiciable. C'était le fruit de ses soupers. Lui-même m'avait dit
plus d'une fois depuis un an, à l'occasion de ce qu'il me quittait
quelquefois, quand j'étais seul avec lui, que Chirac le purgeottait sans
cesse sans qu'il y parût, parce qu'il était si plein qu'il se mettait à
table tous les soirs sans faim et sans aucune envie de manger, quoiqu'il
ne prit rien les matins, et seulement une tasse de chocolat entre une et
deux heures après midi, devant tout le monde, qui était le temps public
de le voir. Je n'étais pas demeuré muet avec lui là-dessus\,; mais toute
représentation était parfaitement inutile. Je savais de plus que Chirac
lui avait nettement déclaré que la continuation habituelle de ses
soupers le conduirait à une prompte apoplexie ou à une hydropisie de
poitrine, parce que sa respiration s'engageait dans des temps, sur quoi
il s'était récrié contre ce dernier mal qui était lent, suffoquant,
contraignant tout, montrant la mort\,; qu'il aimait bien mieux
l'apoplexie qui surprenait et qui tuait tout d'un coup sans avoir le
temps d'y penser.

Un autre homme, au lieu de se récrier sur le genre de mort dont il était
promptement menacé, et d'en préférer un si terrible à un autre qui donne
le temps de se reconnaître, eût songé à vivre et faire ce qu'il fallait
pour cela par une vie sobre, saine et décente, qui, du tempérament qu'il
était, lui aurait pu procurer une fort longue vie, et bien agréable dans
la situation, très vraisemblablement durable, dans laquelle il se
trouvait\,; mais tel fut le double aveuglement de ce malheureux prince.
Je vivais fort en liaison avec l'évêque de Fréjus, et puisque, avenant
faute de M. le duc d'Orléans, il fallait avoir un maître autre que le
roi, en attendant qu'il pût ou voulût l'être, j'aimais mieux que ce fût
ce prélat qu'aucun autre. J'allai donc le trouver, je lui dis ce que
j'avais vu le matin de l'état de M. le duc d'Orléans\,; je lui prédis
que sa perte ne pouvait être longtemps différée et qu'elle arriverait
subitement, sans aucun préalable qui l'annonçât\,; que je conseillai
donc au prélat de prendre ses arrangements et ses mesures avec le roi,
sans y perdre un moment, pour en remplir la place, et que cela lui était
d'autant plus aisé qu'il ne doutait pas de l'affection du roi pour
lui\,; qu'il n'en avait pour personne qui en approchât, et qu'il avait
journellement de longs tête-à-tête avec lui, qui lui offraient tous les
moyens et toutes les facilités de s'assurer de la succession subite à la
place de premier ministre dans l'instant même qu'elle deviendrait
vacante. Je trouvai un homme très reconnaissant en apparence de cet avis
et de ce désir, mais modeste, mesuré, qui trouvait la place au-dessus de
son état et de sa portée.

Ce n'était pas la première fois que nos conversations avaient roulé
là-dessus en général, mais c'était la première fois que je lui en
parlais comme d'une chose instante. Il me dit qu'il y avait bien pensé,
et qu'il ne voyait qu'un prince du sang qui pût être déclaré premier
ministre sans envie, sans jalousie et sans faire crier le public\,;
qu'il ne voyait que M. le Duc à l'être. Je me récriai sur le danger d'un
prince du sang, qui foulerait tout aux pieds, à qui personne ne pourrait
résister, et dont les entours mettraient tout au pillage\,; que le feu
roi, si maître, si absolu, n'en avait jamais voulu mettre aucun dans le
conseil pour ne les pas trop autoriser et accroître. Et quelle
comparaison d'être simplement dans le conseil d'un homme qui gouvernait,
et qui était si jaloux de gouverner et d'être le maître, ou d'être
premier ministre sous un roi enfant, sans expérience, qui n'avait encore
de sa majorité que le nom, sous lequel un premier ministre prince du
sang serait pleinement roi. J'ajoutai qu'il avait eu loisir depuis la
mort du roi de voir avec quelle avidité les princes du sang avaient
pillé les finances, avec quelle opiniâtreté ils avaient protégé Lave et
tout ce qui favorisait leur pillage\,; avec quelle audace ils s'étaient
en toutes manières accrus\,; que de là il pouvait juger de ce que serait
la gestion d'un prince du sang premier ministre, et de M. le Duc en
particulier, qui joignait à ce que je venais de lui représenter une
bêtise presque stupide, une opiniâtreté indomptable, une fermeté
inflexible, un intérêt insatiable, et des entours aussi intéressés que
lui, et nombreux et éclairés, avec lesquels toute la France et lui-même
auraient à compter, ou plutôt à subir toutes les volontés uniquement
personnelles. Fréjus écouta ces réflexions avec une paix profonde, et
les paya de l'aménité d'un sourire tranquille et doux. Il ne me répondit
à pas une des objections que je venais de lui faire, que par me dire
qu'il y avait du vrai dans ce que je venais de lui exposer, mais que M.
le Duc avait du bon, de la probité, de l'honneur, de l'amitié pour
lui\,; qu'il devait le préférer par reconnaissance de l'estime et de
l'amitié que feu M. le Duc lui avait toujours témoignée, et de l'entière
confiance qu'il avait eue en lui à Dijon où il tenait les états, et où
il l'avait retenu comme il y passait pour le voir en revenant de
Languedoc\,; qu'au fond, de M. le duc d'Orléans à un particulier, la
chute était trop grande\,; qu'elle écraserait les épaules de tout
particulier qui lui succéderait, qui ne résisterait jamais à l'envie
générale et à tout ce que lui susciterait la jalousie de chacun\,; qu'un
prince du sang, si fort hors de parité avec qui que ce fût, n'avait rien
de tout cela à démêler\,; que dans la conjoncture dont je lui parlais
comme prochaine, il n'était pas possible de jeter les yeux que sur un
prince du sang, et parmi eux sur M. le Duc, qui était le seul, d'âge et
d'état à pouvoir remplir cette importante place\,; qu'au fond il n'était
point connu du roi et n'avait nulle familiarité avec lui, quoique la
place de surintendant de son éducation, qu'il avait emblée à M. le duc
du Maine, eût dû et pu lui procurer l'un et l'autre\,; qu'il aurait donc
besoin de ceux qui étaient autour du roi, et dans son goût et sa
privance\,; qu'avec ce secours et les mesures que M. le Duc serait
obligé d'avoir avec eux, tout irait bien\,; qu'enfin, plus il y pensait
et y avait pensé, plus il se trouvait convaincu qu'il n'y avait rien que
cela de praticable.

Ces derniers mots m'arrêtèrent tout court. Je lui dis qu'il était plus à
portée de voir les choses de près et avec plus de lumière que
personne\,; que je me contentais de l'avoir averti et de lui avoir
représenté ce que je croyais mériter de l'être\,; que je ne pouvais sans
regret lui voir laisser échapper la place de premier ministre pour
lui-même\,; mais qu'après tout je me rendais, quoique malgré mon
sentiment et mon désir, à plus clairvoyant que moi. Il est aisé de juger
de combien de propos de reconnaissance, d'amitié, de confiance cette
conversation fut assaisonnée de sa part. Je m'en retournai à Meudon avec
le duc d'Humières, bien persuadé que Fréjus n'était arrêté que par sa
timidité\,; qu'il n'en était pas moins avide du souverain pouvoir\,; que
pour allier son ambition avec sa crainte de l'envie et de la jalousie,
capables de le culbuter, ses réflexions l'avaient porté à les faire
taire en mettant un prince du sang dans cette place, dans la
satisfaction de trouver inepte de tous points le seul des princes du
sang par son âge et par son aînesse de MM. ses frères et de M. le prince
de Conti, qui pût y être mis, qui ne serait que le représentant et le
plastron de premier ministre, tandis que lui-même, Fréjus, deviendrait
le véritable premier ministre par sa situation avec le roi, du coeur et
de l'esprit duquel il se trouvait le plein et l'unique possesseur, ce
qui le rendrait si considérable et si nécessaire à M. le Duc qu'il
n'oserait faire la moindre chose sans son attache, en sorte que sans
envie, sans jalousie, conservant tout l'extérieur de modestie, tout en
effet serait entre ses mains. Heurter un projet si pourpensé, et un
projet de cette nature, eût été se casser le nez contre un mur. Aussi
enrayai-je tout court dès ce que je le sentis, et je me gardai bien de
lui dire que M\textsuperscript{me} de Prie et les autres entours de M.
le Duc le feraient sûrement se mécompter, parce qu'ils voudraient bien
sûrement gouverner et profiter, et qu'ils ne pourvoient l'espérer qu'en
faisant que M. le Duc voulût gouverner avec indépendance, et par
conséquent secouât très promptement le joug que Fréjus s'attendait de
lui imposer. Je le dis dès le soir à M\textsuperscript{me} de
Saint-Simon, pour qui je n'eus jamais de secret, et du grand sens de qui
je me trouvai si bien toute ma vie\,: elle en jugea tout comme moi.

La Chaise, fils du frère du feu P. de La Chaise confesseur du feu roi,
et capitaine des gardes de la porte du roi, mourut chez lui en Lyonnais.
Il ne laissa point de fils et avait un brevet de retenue. Torcy obtint
la charge pour son fils. Il y avait déjà longtemps que toutes les
secondes charges de la cour étaient devenues le préciput des fils de
ministres. Celle-ci est une des moindres, mais on tient par elle\,; et
on suit le roi partout.

Le vieux Livry mourut aussi, mais il avait obtenu de M. le duc d'Orléans
la survivance de sa charge de premier maître d'hôtel du roi pour son
fils. Livry père était un très bon homme, familier avec le feu roi, chez
qui on jouait toute la journée à des jeux de commerce. Il faisait assez
mauvaise chère et très mal propre, et s'y enivrait souvent les soirs. Il
est pourtant vrai qu'il ne buvait jamais de vin pur, mais une carafe
d'eau lui aurait bien duré une année. Il buvait sa bouteille en se
levant avec une croûte de pain, et a vécu quatre-vingts ans dans la
santé la plus égale et la plus parfaite, et la tête comme il l'avait eue
toute sa vie. Il eût été bien étonné de voir son fils chevalier de
l'ordre.

Le grand-duc {[}de Toscane{]} mourut en trois ou quatre jours, le
dernier octobre, à près de quatre-vingt-deux ans, et cinquante-quatre
ans de règne, regretté dans ses États comme le père de son peuple, et
dans toute l'Italie et à Rome, comme le plus habile politique, le plus
honnête homme et le plus sensé souverain qui eût paru depuis longtemps
en Europe, où il était généralement estimé, surtout en Italie et à Rome
où il avait beaucoup de crédit et de considération, et passa toujours
pour un prince très sage et très politique. Il avait épousé en 1661 une
fille de Gaston, frère de Louis XIII, partie d'ici avec l'esprit de
retour, qui vécut fort mal avec lui, et fort mal à propos, et qui après
lui avoir donné deux fils et une fille, revint en France passer une vie
méprisée et fort contrainte dans un couvent hors de Paris, suivant la
stipulation du grand-duc, et de laquelle il a été parlé suffisamment
ici. Son fils aîné était mort à quarante ans, en 1713 sans enfants, de
la soeur de M\textsuperscript{me} la Dauphine, de Bavière, une fille
veuve sans enfants de l'électeur palatin en 1716 et retirée à Florence,
et J. Gaston qui lui succéda, qui avait épousé la dernière princesse de
l'ancien Saxe-Lauenbourg, brouillée avec lui, sans enfants, et retirée
en Allemagne\,: prince dernier grand-duc de Toscane de la maison de
Médicis, qui eut de l'esprit et des lettres, régna voyant à peine ses
ministres, dans son lit ou dans sa robe de chambre, seul entre deux
Turcs qui le servaient, toujours la nappe mise dans sa chambre, d'où il
ne sortait presque jamais, presque toujours ivre, et se souciant peu de
ce qui arriverait après lui.

L'électeur de Cologne, frère de l'électeur de Bavière, mourut à Bonn à
cinquante-deux ans, le 12 novembre, quinze jours après le grand-duc. Il
était archevêque de Cologne, évêque de Hildesheim et de Liège. Il en a
été souvent parlé ici. Il était frère de M\textsuperscript{me} la
Dauphine, de Bavière. Son neveu, fils de l'électeur de Bavière, évêque
de Munster et de Paderborn lui succéda à Liège et à Cologne, dont il
était coadjuteur.

La maréchale de Chamilly mourut à Paris à soixante-sept ans, le 18
novembre. C'était une femme d'esprit, de grand sens, de grande piété, de
vertu constante, extrêmement aimable, et faite pour le grand monde et
pour la représentation, qui avait eu la plus grande part à la fortune de
son mari dont elle n'eut point d'enfants. Elle était fort de nos amies,
et nous la regrettâmes fort. Elle en avait beaucoup, et avait toujours
conservé beaucoup d'estime et de considération. Elle s'appelait du
Bouchet, était riche héritière et de naissance fort commune. Le grand
prévôt perdit aussi sa femme qu'il n'avait pas rendue heureuse, et qui
méritait un meilleur sort.

\hypertarget{chapitre-iii.}{%
\chapter{CHAPITRE III.}\label{chapitre-iii.}}

1723

~

{\textsc{Mort du duc de Lauzun\,; sa maison\,; sa famille.}} {\textsc{-
Raisons de m'étendre sur lui.}} {\textsc{- Son caractère.}} {\textsc{-
Sa rapide fortune.}} {\textsc{- Il manque l'artillerie par sa faute.}}
{\textsc{- Son inconcevable hardiesse pour voir clair à son affaire.}}
{\textsc{- Il insulte M\textsuperscript{me} de Montespan, puis le roi
même.}} {\textsc{- Belle action du roi.}} {\textsc{- Lauzun, conduit à
la Bastille, en sort peu de jours après avec la charge de capitaine des
gardes de corps, qu'avait le duc de Gesvres, qui est premier gentilhomme
de la chambre en la place du comte du Lude, fait grand maître de
l'artillerie à la place du duc Mazarin.}} {\textsc{- Aventures de Lauzun
avec Mademoiselle, dont il manque follement le mariage public.}}
{\textsc{- Il fait un cruel tour à M\textsuperscript{me} de Monaco, et
un plus hardi au roi et à elle.}} {\textsc{- Patente de général d'armée
au comte de Lauzun, qui commande un fort gros corps de troupes en
Flandre à la suite du roi.}} {\textsc{- Le comte de Lauzun conduit à
Pignerol.}} {\textsc{- Sa charge donnée à M. de Luxembourg, et son
gouvernement à M. de La Rochefoucauld.}} {\textsc{- Sa précaution pour
se confesser, fort malade.}} {\textsc{- Il fait secrètement connaissance
avec d'autres prisonniers\,; ils trouvent moyen de se voir.}} {\textsc{-
Lauzun entretient de sa fortune et de ses malheurs le surintendant
Fouquet, prisonnier, qui lui croit la tête entièrement tournée.}}
{\textsc{- Fouquet a grand peine à l'en croire sur tous les témoignages
d'autrui, et à la fin ils se brouillent pour toujours.}} {\textsc{-
Soeurs du comte de Lauzun.}} {\textsc{- Caractère et deuil extrême de
M\textsuperscript{me} de Nogent, toute sa vie, de son mari\,; imitée de
deux autres veuves.}} {\textsc{- Mademoiselle achète bien cher la
liberté de Lauzun, à leurs communs dépens, en enrichissant forcément le
duc du Maine, qui, à son grand dépit, prend ses livrées et les transmet
aux siens et à son frère.}} {\textsc{- Lauzun en liberté en Anjou et en
Touraine.}} {\textsc{- Lauzun à Paris, sans approcher la cour de deux
lieues\,; se jette dans le gros jeu\,; y gagne gros\,; passe avec
permission à Londres, où il est bien reçu, et n'est pas moins heureux.}}
{\textsc{- Lauzun sauve la reine d'Angleterre et le prince de Galles.}}
{\textsc{- Rappelé à la cour avec ses anciennes distinctions, il obtient
la Jarretière, est général des armées en Irlande, enfin duc vérifié en
1692.}} {\textsc{- Splendeur de la vie du duc de Lauzun, toujours outré
de l'inutilité de tout ce qu'il emploie pour rentrer dans la confiance
du roi.}} {\textsc{- Ses bassesses sous un extérieur de dignité.}}
{\textsc{- Son fol anniversaire de sa disgrâce.}} {\textsc{- Son
étranger singularité.}} {\textsc{- Il est craint, ménagé, nullement
aimé, quoique fort noble et généreux.}} {\textsc{- Étrange désespoir du
duc de Lauzun, inconsolable, à son âge, de n'être plus capitaine des
gardes, et son terrible aveu.}} {\textsc{- Réflexion.}} {\textsc{-
Combien il était dangereux.}} {\textsc{- Il était reconnaissant et
généreux.}} {\textsc{- Quelques-uns de ses bons mots à M. le duc
d'Orléans.}} {\textsc{- Il ne peut s'empêcher de lâcher sur moi un
dangereux trait.}} {\textsc{- Il tombe fort malade et se moque
plaisamment de son curé, de son cousin de La Force et de sa nièce de
Biron.}} {\textsc{- Sa grande santé.}} {\textsc{- Ses brouilleries avec
Mademoiselle.}} {\textsc{- Leur étrange raccommodement à Eu.}}
{\textsc{- Ils se battent dans la suite et se brouillent pour
toujours.}} {\textsc{- Son humeur solitaire.}} {\textsc{- Son incapacité
d'écrire ce qu'il avait vu, même de le raconter.}} {\textsc{- Sa
dernière maladie.}} {\textsc{- Sa mort courageuse et chrétienne.}}
{\textsc{- Causes de prolixité sur le duc de Lauzun.}}

~

Le duc de Lauzun mourut le 19 novembre à quatre-vingt-dix ans et six
mois. L'union intime des deux sœurs que lui et moi avions épousées, et
l'habitation continuelle de la cour, où môme nous avions un pavillon
fixé pour nous quatre à Marly tous les voyages, m'a fait vivre
continuellement avec lui, et depuis la mort du roi nous nous voyions
presque tous les jours à Paris, et nous mangions continuellement
ensemble chez moi et chez lui. Il a été un personnage si extraordinaire
et si unique en tout genre, que c'est avec beaucoup de raison que La
Bruyère a dit de lui dans ses \emph{Caractères}\footnote{Chap. \emph{de
  la cour}. Lauzun y figure sous le nom de Straton.} qu'il n'était pas
permis de rêver comme il a vécu. À qui l'a vu de près même dans sa
vieillesse, ce mot semble avoir encore plus de justesse. C'est ce qui
m'engage à m'étendre ici sur lui. Il était de la maison de Caumont dont
la branche des ducs de La Force a toujours passé pour l'aînée, quoique
celle de Lauzun le lui ait voulu disputer.

La mère de M. de Lauzun était fille du duc de La Force, fils du second
maréchal duc de La Force, et frère de la maréchale de Turenne, mais d'un
autre lit\,; la maréchale était du premier lit d'une La Roche-Faton, le
duc de La Force était fils d'une Belsunce dont le duc de La Force était
devenu amoureux, qu'il avait épousée en secondes noces, et dont le frère
avait été son page.

Le comte de Lauzun, leur gendre, père du duc de Lauzun dont le père et
le grand-père furent chevaliers de l'ordre en 1585 et en 1619, et
avaient la compagnie des cent gentilshommes de la maison du roi au bec
de corbin, était cousin germain du premier maréchal duc de Grammont, et
du vieux comte de Grammont (duquel et de sa femme morts, peu d'années
avant le feu roi, il a été souvent parlé ici), parce que sa mère était
leur tante paternelle. Le comte de Lauzun, père du duc, fut aussi
capitaine des cent gentilshommes de la maison du roi au bec de corbin,
mourut en 1660, et avait eu cinq fils et quatre filles. L'aîné mourut
fort jeune, le second vécut obscur flans sa province jusqu'en 1677, sans
alliance\,; le troisième fut Puyguilhem, depuis duc de Lauzun, cause de
tout ce détail\,; le quatrième languit obscur capitaine des galères,
sans alliance, jusqu'en 1692\,; le dernier fut chevalier de Lauzun qui
servit fort peu dans la gendarmerie, passa en Hongrie avec MM. les
princes de Conti, s'y attacha quelque temps au service de l'empereur en
qualité d'officier général, s'en dégoûta bientôt, revint à Paris après
un exil assez long\,; manière de philosophe bizarre, solitaire, obscur,
difficile à vivre, avec de l'esprit et des connaissances, souvent mal
avec son frère, qui lui donnait de quoi vivre, souvent à la
sollicitation de la duchesse de Lauzun. Il mourut à Paris sans alliance,
en 1707, à soixante ans.

Le duc de Lauzun était un petit homme, blondasse, bien fait dans sa
taille, de physionomie haute, pleine d'esprit, qui imposait, mais sans
agrément dans le visage, à ce que j'ai ouï dire aux gens de son temps\,;
plein d'ambition, de caprices, de fantaisies, jaloux de tout, voulant
toujours passer le but, jamais content de rien, sans lettres, sans aucun
ornement ni agrément dans l'esprit, naturellement chagrin, solitaire,
sauvage\,; fort noble dans toutes ses façons, méchant et malin par
nature, encore plus par jalousie et par ambition, toutefois bon ami
quand il l'était, ce qui était rare, et bon parent, volontiers ennemi
même des indifférents, et cruel aux défauts et à trouver et donner des
ridicules, extrêmement brave et aussi dangereusement hardi. Courtisan
également insolent, moqueur et bas jusqu'au valetage, et plein de
recherches d'industrie, d'intrigues, de bassesse pour arriver à ses
fins, avec cela dangereux aux ministres, à la cour redouté de tous, et
plein de traits cruels et pleins de sel qui n'épargnaient personne. Il
vint à la cour sans aucun bien, cadet de Gascogne fort jeune, débarquer
de sa province sous le nom de marquis de Puyguilhem. Le maréchal de
Grammont, cousin germain de son père, le retira chez lui. Il était lors
dans la première considération à la cour, dans la confidence de la reine
mère et du cardinal Mazarin, et avait le régiment des gardes et la
survivance pour le comte de Guiche son fils aîné, qui, de son côté,
était la fleur des braves et des dames, et des plus avant dans les
bonnes grâces du roi et de la comtesse de Soissons, nièce du cardinal,
de chez laquelle le roi ne bougeait, et qui était la reine de la cour.
Le comte de Guiche y introduisit le marquis de Puyguilhem, qui en fort
peu de temps devint favori du roi, qui lui donna son régiment de dragons
en le créant, et bientôt après le fit maréchal de camp, et créa pour lui
la charge de colonel général des dragons.

Le duc Mazarin, déjà retiré de la cour, en 1669 voulut se défaire de sa
charge de grand maître de l'artillerie\,; Puyguilhem en eut le vent des
premiers, il la demanda au roi qui la lui promit, mais sous le secret
pour quelques jours. Le jour venu que le roi lui avait dit qu'il le
déclarerait, Puyguilhem qui avait les entrées des premiers gentilshommes
de la chambre, qu'on nomme aussi les grandes entrées, alla attendre la
sortie du roi du conseil des finances, dans une pièce où personne
n'entrait pendant le conseil, entre celle où toute la cour attendait et
celle où le conseil se tenait. Il y trouva Nyert, premier valet de
chambre en quartier, qui lui demanda par quel hasard il y venait\,;
Puyguilhem sûr de son affaire crut se dévouer ce premier valet de
chambre en lui faisant confidence de ce qui allait se déclarer en sa
faveur\,; Nyert lui en témoigna sa joie, puis tira sa montre, et vit
qu'il aurait encore le temps d'aller exécuter, disait-il, quelque chose
de court et de pressé que le roi lui avait ordonné\,: il monte quatre à
quatre un petit degré au haut duquel était le bureau où Louvois
travaillait toute la journée, car à Saint-Germain les logements étaient
fort petits et fort rares, et les ministres et presque toute la cour
logeaient chacun chez soi, à la ville. Nyert entre dans le bureau de
Louvois, et l'avertit qu'au sortir du conseil des finances, dont Louvois
n'était point, Puyguilhem allait être déclaré grand maître de
l'artillerie, et lui conte ce qu'il venait d'apprendre de lui-même, et
où il l'avait laissé.

Louvois haïssait Puyguilhem, ami de Colbert, son émule, et il en
craignait la faveur et les hauteurs dans une charge qui avait tant de
rapports nécessaires avec son département de la guerre, et de laquelle
il envahissait les fonctions et l'autorité tant qu'il pouvait, ce qu'il
sentait que Puyguilhem ne serait ni d'humeur ni de faveur à souffrir. Il
embrasse Nyert, le remercie, le renvoie au plus vite, prend quelque
papier pour lui servir d'introduction, descend, et trouve Puyguilhem et
Nyert dans cette pièce ci-devant dite. Nyert fait le surpris de voir
arriver Louvois, et lui dit que le conseil n'est pas levé. «\,
N'importe, répondit Louvois, je veux entrer\,; j'ai quelque chose de
pressé à dire au roi\,;» et tout de suite entre\,; le roi surpris de le
voir lui demande ce qui l'amène, se lève et va à lui. Louvois le tire
dans l'embrasure d'une fenêtre, et lui dit qu'il sait qu'il va déclarer
Puyguilhem grand maître de l'artillerie, qu'il l'attend à la sortie du
conseil dans la pièce voisine, que Sa Majesté est pleinement maîtresse
de ses grâces et de ses choix, mais qu'il a cru de son service de lui
représenter l'incompatibilité qui est entre Puyguilhem et lui, ses
caprices, ses hauteurs\,; qu'il voudra tout faire et tout changer dans
l'artillerie\,; que cette charge a une si nécessaire connexion avec le
département de la guerre, qu'il est impossible que le service s'y fasse
parmi des entreprises et des fantaisies continuelles, et la
mésintelligence déclarée entre le grand maître et le secrétaire d'État,
dont le moindre inconvénient sera d'importuner Sa Majesté tous les jours
de leurs querelles et de leurs réciproques prétentions, dont il faudra
qu'elle soit juge à tous moments.

Le roi se sentit extrêmement piqué de voir son secret su de celui à qui
principalement il le voulait cacher\,; répond à Louvois d'un air fort
sérieux que cela n'est pas fait encore, le congédie et va se rasseoir au
conseil. Un moment après qu'il fut levé, le roi sort pour aller à la
messe, voit Puyguilhem et passe sans lui rien dire. Puyguilhem fort
étonné attend le reste de la journée, et voyant que la déclaration
promise ne venait point, en parle au roi à son petit coucher. Le roi lui
répond que cela ne se peut encore, et qu'il verra\,: l'ambiguïté de la
réponse et son ton sec alarment Puyguilhem\,; il avait le vol des dames
et le jargon de la galanterie\,; il va trouver M\textsuperscript{me} de
Montespan, à qui il conte son inquiétude, et qu'il conjure de la faire
cesser. Elle lui promet merveilles et l'amuse ainsi plusieurs jours.

Las de tout ce manège et ne pouvant deviner d'où lui vient son mal, il
prend une résolution incroyable si elle n'était attestée de toute la
cour d'alors. Il couchait avec une femme de chambre favorite de
M\textsuperscript{me} de Montespan, car tout lui était bon pour être
averti et protégé, et vient à bout de la plus hasardeuse hardiesse dont
on ait jamais ouï parler. Parmi tous ses amours le roi ne découcha
jamais d'avec la reine, souvent tard, mais sans y manquer, tellement que
pour être plus à son aise, il se mettait les après-dînées entre deux
draps chez ses maîtresses. Puyguilhem se fit cacher par cette femme de
chambre sous le lit dans lequel le roi s'allait mettre avec
M\textsuperscript{me} de Montespan, et par leur conversation, y apprit
l'obstacle que Louvois avait mis à sa charge, la colère du roi de ce que
son secret avait été éventé, sa résolution de ne lui point donner
l'artillerie par ce dépit, et pour éviter les querelles et l'importunité
continuelle d'avoir à les décider entre Puyguilhem et Louvois. Il y
entendit tous les propos qui se tinrent de lui entre le roi et sa
maîtresse, et que celle-ci qui lui avait tant promis tous ses bons
offices, lui en rendit tous les mauvais qu'elle put. Une toux, le
moindre mouvement, le plus léger hasard pouvait déceler ce téméraire, et
alors que serait-il devenu\,? Ce sont de ces choses dont le récit
étouffe et épouvante tout à la fois.

Il fut plus heureux que sage, et ne fut point découvert. Le roi et sa
maîtresse sortirent enfin de ce lit\,; le roi se rhabilla et s'en alla
chez lui, M\textsuperscript{me} de Montespan se mit à sa toilette pour
aller à la répétition d'un ballet où le roi, la reine et toute la cour
devait aller. La femme de chambre tira Puyguilhem de dessous ce lit, qui
apparemment n'eut pas un moindre besoin d'aller se rajuster chez lui. De
là il s'en vint se coller à la porte de la chambre de
M\textsuperscript{me} de Montespan.

Lorsqu'elle en sortit pour aller à la répétition du ballet, il lui
présenta la main, et lui demanda avec un air plein de douceur et de
respect, s'il pouvait se flatter qu'elle eût daigné se souvenir de lui
auprès du roi. Elle l'assura qu'elle n'y avait pas manqué, et lui
composa comme il lui plut tous les services qu'elle venait de lui
rendre. Par-ci, par-là il l'interrompit crédulement de questions pour la
mieux enferrer, puis s'approchant de son oreille, il lui dit qu'elle
était une menteuse, une friponne, une coquine, une p\ldots. à chien, et
lui répéta mot pour mot toute la conversation du roi et d'elle.
M\textsuperscript{me} de Montespan en fut si troublée qu'elle n'eut pas
la force de lui répondre un seul mot, et à peine de gagner le lieu où
elle allait, avec grande difficulté à surmonter et à cacher le
tremblement de ses jambes et de tout son corps, en sorte qu'en arrivant
dans le lieu de la répétition du ballet, elle s'évanouit. Toute la cour
y était déjà. Le roi tout effrayé vint à elle, on eut de la peine à la
faire revenir. Le soir elle conta au roi ce qui lui était arrivé, et ne
doutait pas que ce ne fût le diable qui eût sitôt et si précisément
informé Puyguilhem de tout ce qu'ils avaient dit de lui dans ce lit. Le
roi fut extrêmement irrité de toutes les injures que
M\textsuperscript{me} de Montespan en avait essuyées, et fort en peine
comment Puyguilhem avait {[}pu{]} être si exactement et si subitement
instruit.

Puyguilhem, de son côté, était furieux de manquer l'artillerie, de sorte
que le roi et lui se trouvaient dans une étrange contrainte ensemble.
Cela ne put durer que quelques jours. Puyguilhem, avec ses grandes
entrées, épia un tête-à-tête avec le roi et le saisit. Il lui parla de
l'artillerie et le somma audacieusement de sa parole. Le roi lui
répondit qu'il n'en était plus tenu, puisqu'il ne la lui avait donnée
que sous le secret, et qu'il y avait manqué. Là-dessus Puyguilhem
s'éloigne de quelques pas, tourne le dos au roi, tire son épée, en casse
la lame avec son pied, et s'écrie en fureur qu'il ne servira de sa vie
un prince qui lui manque si vilainement de parole. Le roi, transporté de
colère, fit peut-être dans ce moment la plus belle action de sa vie. Il
se tourne à l'instant, ouvre la fenêtre, jette sa canne dehors, dit
qu'il serait fâché d'avoir frappé un homme de qualité, et sort.

Le lendemain matin, Puyguilhem, qui n'avait osé se montrer depuis, fut
arrêté dans sa chambre et conduit à la Bastille. Il était ami intime de
Guitry, favori du roi, pour lequel il avait créé la charge de grand
maître de la garde-robe. Il osa parler au roi en sa faveur, et tacher de
rappeler ce goût infini qu'il avait pris pour lui. Il réussit à toucher
le roi d'avoir fait tourner la tête à Puyguilhem par le refus d'une
aussi grande charge, sur laquelle il avait cru devoir compter sur sa
parole, tellement que le roi voulut réparer ce refus. Il donna
l'artillerie au comte du Lude, chevalier de l'ordre en 1661, qu'il
aimait fort par habitude et par la conformité du goût de la galanterie
et de la chasse. Il était capitaine et gouverneur de Saint-Germain, et
premier gentilhomme de la chambre. Il le fit duc non vérifié ou à brevet
en 1675. La duchesse du Lude, dame d'honneur de M\textsuperscript{me} la
Dauphine-Savoie, était sa seconde femme et sa veuve sans enfants. Il
vendit sa charge de premier gentilhomme de la chambre, pour payer
l'artillerie, au duc de Gesvres, qui était capitaine des gardes du
corps, et le roi fit offrir cette dernière charge en dédommagement à
Puyguilhem, dans la Bastille. Puyguilhem, voyant cet incroyable et
prompt retour du roi pour lui, reprit assez d'audace pour se flatter
d'en tirer un plus grand parti, et refusa. Le roi ne s'en rebuta point.
Guitry alla prêcher sou ami dans la Bastille, et obtint à grand'peine
qu'il aurait la bonté d'accepter l'offre du roi. Dès qu'il eut accepté,
il sortit de la Bastille, alla saluer le roi, et prêter serment de sa
nouvelle charge, et vendit les dragons.

Il avait eu, dès 1665, le gouvernement de Berry, à la mort du maréchal
de Clerembault. Je ne parle point ici de ses aventures avec
Mademoiselle, qu'elle raconte elle-même si naïvement dans ses mémoires,
et l'extrême folie qu'il fit de différer son mariage avec elle, auquel
le roi avait consenti, pour avoir de belles livrées et pour obtenir que
le mariage fût célébré à la messe du roi, ce qui donna le temps à
Monsieur, poussé par M. le Prince, d'aller tous deux faire des
représentations au roi, qui l'engagèrent à rétracter son consentement\,;
ce qui rompit le mariage. Mademoiselle jeta feu et flammes\,; mais
Puyguilhem, qui, depuis la mort de son père, avait pris le nom de comte
de Lauzun, en fit au roi le grand sacrifice de bonne grâce, et plus
sagement qu'il ne lui appartenait. Il avait eu la compagnie des cent
gentilshommes de la maison du roi au bec de corbin, qu'avait son père,
et venait d'être fait lieutenant général.

II était amoureux de M\textsuperscript{me} de Monaco, soeur du comte de
Guiche, intime amie de Madame et dans toutes ses intrigues, tellement
que, quoique ce fût chose sans exemple et qui n'en a pas eu depuis, elle
obtint du roi, avec qui elle était extrêmement bien, d'avoir, comme
fille d'Angleterre, une surintendante comme la reine, et que ce fût
M\textsuperscript{me} de Monaco. Lauzun était fort jaloux et n'était pas
content d'elle. Une après-dînée d'été qu'il était allé à Saint-Cloud, il
trouva Madame et sa cour assises à terre sur le parquet, pour se
rafraîchir, et M\textsuperscript{me} de Monaco à demi couchée, une main
renversée par terre. Lauzun se met en galanterie avec les dames, et
tourne si bien qu'il appuie son talon dans le creux de la main de
M\textsuperscript{me} de Monaco, y fait la pirouette et s'en va.
M\textsuperscript{me} de Monaco eut la force de ne point crier et de
s'en taire. Peu après il lit bien pis. Il écuma que le roi avait des
passades avec elle, et l'heure où Bontems la conduisait enveloppée d'une
cape, par un degré dérobé, sur le palier duquel était une porte de
derrière des cabinets du roi et vis-à-vis, sur le même palier, un privé.
Lauzun prévient l'heure et s'embusque dans le privé, le ferme en dedans
d'un crochet, voit par le trou de la serrure le roi qui ouvre sa porte
et met la clef en dehors et la referme. Lauzun attend un peu, écoute à
la porte, la ferme à double tour avec la clef, la tire et la jette dans
le privé, où il s'enferme de nouveau. Quelque temps après arrive Bontems
et la dame, qui sont bien étonnés de ne point trouver la clef à la porte
du cabinet. Bontems frappe doucement plusieurs fois inutilement, enfin
si fort que le roi arrive. Bontems lui dit qu'elle est là et d'ouvrir,
parce que la clef n'y est pas. Le roi répond qu'il l'y a mise\,; Bontems
la cherche à terre pendant que le roi veut ouvrir avec le pêne, et il
trouve la porte fermée à double tour. Les voilà tous trois bien étonnés
et bien empêchés\,; la conversation se fait à travers la porte comment
ce contre-temps peut être arrivé\,; le roi s'épuise à vouloir forcer le
pêne, et ouvrir malgré le double tour. À la fin il fallut se donner le
bonsoir à travers la porte, et Lauzun, qui les entendait, à n'en pas
perdre un mot, et qui les voyait de son privé par le trou de la serrure,
bien enfermé au crochet comme quelqu'un qui serait sur le privé, riait
bas de tout son coeur, et se moquait d'eux avec délices.

En 1670, le roi voulut faire un voyage triomphant avec les dames, sous
prétexte d'aller visiter ses places de Flandre, accompagné d'un corps
d'armée et de toutes les troupes de sa maison, tellement que l'alarme en
fut grande dans les Pays-Bas, que le roi prit soin de rassurer. Il donna
le commandement du total au comte de Lauzun, avec la patente de général
d'armée. Il en fit les fonctions avec beaucoup d'intelligence, une
galanterie et une magnificence extrême. Cet éclat et cette marque si
distinguée de la faveur de Lauzun donna fort à penser à Louvois que
Lauzun ne ménageait en aucune sorte. Ce ministre se joignit à
M\textsuperscript{me} de Montespan, qui ne lui avait pas pardonné la
découverte qu'il avait faite et les injures atroces qu'il lui avait
dites, et {[}ils{]} firent si bien tous les deux qu'ils réveillèrent
dans le roi le souvenir de l'épée brisée, l'insolence d'avoir si peu
après et encore dans la Bastille, refusé plusieurs jours la charge de
capitaine des gardes du corps, le firent regarder comme un homme qui ne
se connaissait plus, qui avait suborné Mademoiselle jusqu'à s'être vu si
près de l'épouser, et s'en être fait assurer des biens immenses\,; enfin
comme un homme très dangereux par son audace, et qui s'était mis en tête
de se dévouer les troupes par sa magnificence, ses services aux
officiers, et par la manière dont il avait vécu avec elles au voyage de
Flandre, et s en était fait adorer. Ils lui firent un crime d'être
demeuré ami et en grande liaison avec la comtesse de Soissons, chassée
de la cour et soupçonnée de crimes. Il faut bien qu'ils en aient donné
quelqu'un à Lauzun que je n'ai pu apprendre, par le traitement barbare
qu'ils vinrent à bout de lui faire.

Ces menées durèrent toute l'année 1671, sans que Lauzun pût s'apercevoir
de rien au visage du roi ni à celui de M\textsuperscript{me} de
Montespan, qui le traitaient avec la distinction et la familiarité
ordinaire. Il se connaissait fort en pierreries et à les faire bien
monter, et M\textsuperscript{me} de Montespan l'y employait souvent. Un
soir du milieu de novembre 1671, qu'il arrivait de Paris, où
M\textsuperscript{me} de Montespan l'avait envoyé le matin pour des
pierreries, comme le comte de Lauzun ne faisait que mettre pied à terre,
et entrer dans sa chambre, le maréchal de Rochefort, capitaine des
gardes en quartier, y entra presque au même moment et l'arrêta. Lauzun,
dans la dernière surprise, voulut savoir pourquoi, voir le roi ou
M\textsuperscript{me} de Montespan, au moins leur écrire\,: tout lui fut
refusé. Il fut conduit à la Bastille, et peu après à Pignerol, où il fut
enfermé sous une basse voûte. Sa charge de capitaine des gardes du corps
fut donnée à M. de Luxembourg, et le gouvernement de Berry au duc de La
Rochefoucauld, qui, à la mort de Guitry, au passage du Rhin, 12 juin
1672, fut grand maître de la garde-robe.

On peut juger de l'état d'un homme tel qu'était Lauzun, précipité en un
clin d'oeil de si haut dans un cachot du château de Pignerol, sans voir
personne et sans imaginer pourquoi. Il s'y soutint pourtant assez
longtemps, mais à la fin il y tomba si malade qu'il fallut songer à se
confesser. Je lui ai ouï conter qu'il craignit un prêtre supposé\,; qu'à
cause de cela, il voulut opiniâtrement un capucin, et que dès qu'il fut
venu, il lui sauta à la barbe, et la tira tant qu'il put de tous côtés
pour voir si elle n'était point postiche. Il fut quatre ou cinq ans dans
ce cachot. Les prisonniers trouvent des industries que la nécessité
apprend. Il y en avait au-dessus de lui et à côté, aussi plus haut\,:
ils trouvèrent moyen de lui parler. Ce commerce les conduisit à faire un
trou bien caché pour s'entendre plus aisément, puis de l'accroître et de
se visiter.

Le surintendant Fouquet était enfermé dans leur voisinage depuis
décembre 1664, qu'il y avait été conduit de la Bastille, où on l'avait
amené de Nantes où le roi était, et où il l'avait fait arrêter le 5
septembre 1661, et mener à la Bastille. Il sut par ses voisins, qui
avaient trouvé aussi moyen de le voir, que Lauzun était sous eux.
Fouquet, qui ne recevait aucune nouvelle, en espéra par lui, et eut
grande envie de le voir. Il l'avait laissé jeune homme, pointant à la
cour par le maréchal de Grammont, bien reçu chez la comtesse de Soissons
d'où le roi ne bougeait, et le voyait déjà de bon oeil. Les prisonniers
qui avaient lié commerce avec lui firent tant qu'ils le persuadèrent de
se laisser hisser par leur trou pour voir Fouquet chez eux, que Lauzun
aussi était bien aise de voir. Les voilà donc ensemble, et Lauzun à
conter sa fortune et ses malheurs à Fouquet. Le malheureux surintendant
ouvrait les oreilles et de grands yeux quand il entendit dire à ce cadet
de Gascogne, trop heureux d'être recueilli et hébergé chez le maréchal
de Grammont, qu'il avait été général des dragons, capitaine des gardes,
et eu la patente et la fonction de général d'armée. Fouquet ne savait
plus où il en était, le crut fou, et qu'il lui racontait ses visions,
quand il lui expliqua comment il avait manqué l'artillerie, et ce qui
s'était passé après là-dessus\,; mais il ne douta plus de la folie
arrivée à son comble, jusqu'à avoir peur de se trouver avec lui, quand
il lui raconta son mariage consenti par le roi avec Mademoiselle,
comment rompu, et tous les biens qu'elle lui avait assurés. Cela
refroidit fort leur commerce, du côté de Fouquet, qui, lui croyant la
cervelle totalement renversée, ne prenait que pour des contes en l'air
toutes les nouvelles que Lauzun lui disait de tout ce qui s'était passé
dans le monde depuis la prison de l'un jusqu'à la prison de l'autre.

Celle du malheureux surintendant fut un peu adoucie avant celle de
Lauzun. Sa femme, et quelques officiers du château de Pignerol, eurent
permission de le voir et de lui apprendre des nouvelles du monde. Une
des premières choses qu'il leur dit fut de plaindre ce pauvre
Puyguilhem, qu'il avait laissé jeune et sur un assez bon pied à la cour
pour son âge, à qui la cervelle avait tourné, et dont on cachait la
folie dans cette même prison\,; mais quel fut son étonnement quand tous
lui dirent et lui assurèrent la vérité des mêmes choses qu'il avait,
sues de lui\,! Il n'en revenait pas, et fut tenté de leur croire à tous
la cervelle dérangée il fallut du temps pour le persuader. À son tour
Lauzun fut tiré du cachot, et eut une chambre, et bientôt après la même
liberté qu'on avait donnée à Fouquet, afin de se voir tous deux tant
qu'ils voulurent. Je n'ai jamais su ce qui en déplut à Lauzun\,; mais il
sortit de Pignerol son ennemi, et a fait depuis tout du pis qu'il a pu à
Fouquet, et après sa mort, jusqu'à la sienne, à sa famille.

Le comte de Lauzun avait quatre soeurs, qui toutes n'avaient rien.
L'aînée fut fille d'honneur de la reine mère, qui la fit épouser, en
1663, à Nogent, qui était Bautru, et capitaine de la porte, et maître de
la garde-robe, tué au passage du Rhin, laissant un fils et des filles.
La seconde épousa Belsunce, et passa sa vie avec lui dans leur
province\,; la troisième fut abbesse de Notre-Dame de Saintes, et la
quatrième, du Ronceray\footnote{Il faut lire \emph{Ronceray} et non
  \emph{Romeray}, comme le portent les anciennes éditions. Cette abbaye,
  de l'ordre de Saint-Benoît, avait été fondée, au XIe siècle, par
  Foulques-Nera, comte d'Anjou.} à Angers.

M\textsuperscript{me} de Nogent n'avait ni moins d'esprit, ni guère
moins d'intrigue que son frère, mais bien plus suivie et bien moins
d'extraordinaire que lui, quoiqu'elle en eût aussi sa part. Mais elle
fut fort arrêtée par l'extrême douleur de la perte de son mari, dont
elle porta tout le reste de sa vie le premier grand deuil de veuve, et
en garda toutes les contraignantes bienséances. Ce fut la première qui
s'en avisa. M\textsuperscript{me} de Vaubrun, sa belle-soeur, suivit son
exemple. Elles avaient épousé les deux frères, et dans ces derniers
temps M\textsuperscript{me} de Cavoye, de qui j'ai assez parlé ici.
Malgré ce deuil, M\textsuperscript{me} de Nogent plaça l'argent des
brevets de retenue de la dépouille de son frère, et des dragons qu'il
avait eus pour rien, régiment et charge de colonel général qu'il avait
vendus\,; elle prit soin du reste de son bien, et en accumula si bien
les revenus, et le fit si bien valoir pendant sa longue prison, qu'il en
sortit extrêmement riche. Elle eut enfin la permission de le voir, et
fit plusieurs voyages à Pignerol.

Mademoiselle était inconsolable de cette longue et dure prison, et
faisait toutes les démarches possibles pour délivrer le comte de Lauzun.
Le roi résolut enfin d'en profiter pour le duc du Maine et de la lui
faire acheter bien cher. Il lui en fit faire la proposition, qui n'alla
pas à moins qu'à assurer, après elle, au duc du Maine et à sa postérité
le comté d'Eu, le duché d'Aumale et la principauté de Dombes. Le don
était énorme, tant par le prix que par la dignité et l'étendue de ces
trois morceaux. Elle avait de plus assuré les deux premiers à Lauzun,
avec le duché de Saint-Fargeau et la belle terre de Thiers en Auvergne,
lorsque leur mariage fut rompu, et il fallait le faire renoncer à Eu et
à Aumale, pour que Mademoiselle en pût disposer en faveur du duc du
Maine. Mademoiselle ne se pouvait résoudre à passer sous ce joug et à
dépouiller Lauzun de bienfaits si considérables. Elle fut priée jusqu'à
la dernière importunité, enfin menacée par les ministres, tantôt
Louvois, tantôt Colbert, duquel elle était plus contente, parce qu'il
était bien de tout temps avec Lauzun, et qu'il la maniait plus doucement
que Louvois, son ennemi, qui était toujours réservé à porter les plus
dures paroles, et qui s'en acquittait encore plus durement. Elle sentait
sans cesse que le roi ne l'aimait point, et qu'il ne lui avait jamais
pardonné le voyage d'Orléans\footnote{Voy. les \emph{Mémoires de
  Mademoiselle}, à l'année 1652. Elle entra dans Orléans par escalade le
  27 mars 1652, et ferma cette ville aux troupes royales.}, qu'elle
rassura dans sa révolte, moins encore le canon de la Bastille, qu'elle
fit tirer en sa présence sur les troupes du roi, et qui sauva M. le
Prince et les siennes au combat du faubourg Saint-Antoine. Elle comprit
donc enfin que le roi, éloigné d'elle sans retour, et qui ne consentait
à la liberté de Lauzun que par sa passion d'élever et d'enrichir ses
bâtards, ne cesserait de la persécuter jusqu'à ce qu'elle eût consenti,
sans aucune espérance de rien rabattre\,; {[}elle{]} y donna enfin les
mains avec les plaintes et les larmes les plus amères. Mais pour la
validité de la chose, on trouva qu'il fallait que Lauzun fût en liberté
pour renoncer au don de Mademoiselle, tellement qu'on prit le biais
qu'il avait besoin des eaux de Bourbon, et M\textsuperscript{me} de
Montespan aussi, pour qu'ils y pussent conférer ensemble sur cette
affaire.

Lauzun y fut amené et gardé à Bourbon par un détachement de
mousquetaires commandé par Maupertuis. Lauzun vit donc plusieurs fois
M\textsuperscript{me} de Montespan chez elle à Bourbon. Mais il fut si
indigné du grand dépouillement qu'elle lui donna pour condition de sa
liberté, qu'après de longues disputes, il n'en voulut plus ouïr parler,
et fut reconduit à Pignerol comme il en avait été ramené.

Cette fermeté n'était pas le compte du roi pour son bâtard bien-aimé. Il
envoya M\textsuperscript{me} de Nogent à Pignerol\,; après, Barin, ami
de Lauzun, et qui se mêlait de toutes ses affaires, avec des menaces et
des promesses, qui, avec grande peine, obtinrent le consentement de
Lauzun, qui firent résoudre à un second voyage de Bourbon de lui et de
M\textsuperscript{me} de Montespan, sous le même prétexte des eaux. Il y
fut conduit comme la première fois, et n'a jamais pardonné à Maupertuis
la sévère pédanterie de son exactitude. Ce dernier voyage se fit dans
l'automne de 1680. Lauzun y consentit à tout, M\textsuperscript{me} de
Montespan revint triomphante. Maupertuis et ses mousquetaires prirent
congé du comte de Lauzun à Bourbon, d'où il eut permission d'aller
demeurer à Angers, et incontinent après cet exil fut élargi, en sorte
qu'il eut la liberté de tout l'Anjou et la Touraine. La consommation de
l'affaire fut différée au commencement de février 1681, pour lui donner
un plus grand air de pleine liberté. Ainsi Lauzun n'eut de Mademoiselle
que Saint-Fargeau et Thiers, après n'avoir tenu qu'à lui de l'épouser en
se hâtant de le faire, et de succéder à la totalité de ses immenses
biens. Le duc du Maine fut instruit à faire sa cour à Mademoiselle, qui
le reçut toujours très fraîchement, et qui lui vit prendre ses livrées
avec grand dépit, comme une marque de sa reconnaissance, en effet pour
s'en relever et honorer, car c'était celles de Gaston, que dans la suite
le comte de Toulouse prit aussi, non par la même raison, mais sous
prétexte de conformité avec son frère, et {[}ils{]} l'ont fait passer à
leurs enfants.

Lauzun, à qui on avait fait espérer un traitement plus doux, demeura
quatre ans à se promener dans ces deux provinces, où il ne s'ennuyait
guère moins que Mademoiselle faisait de son absence. Elle cria, se fâcha
contre M\textsuperscript{me} de Montespan et contre son fils, se
plaignit hautement qu'après l'avoir impitoyablement rançonnée on la
trompait encore en tenant Lauzun éloigné, et fit tant de bruit qu'enfin
elle obtint son retour à Paris, et liberté entière, à condition de
n'approcher pas plus près de deux lieues de tout le lieu où le roi
serait. Il vint donc à Paris où il vit assidûment sa bienfaitrice.
L'ennui de cette sorte d'exil, pourtant si adouci, le jeta dans le gros
jeu et il y fut extrêmement heureux\,; toujours beau et sûr joueur, et
net en tout au possible, et il gagna fort gros. Monsieur, qui faisait
quelquefois de petits séjours à Paris, et qui y jouait gros jeu, lui
permit de venir jouer avec lui au Palais-Royal, puis à Saint-Cloud, où
il faisait l'été de plus longs séjours. Lauzun passa ainsi plusieurs
années, gagnant et prêtant beaucoup d'argent fort noblement\,; mais plus
il se trou voit près de la cour et parmi le grand monde, plus la défense
d'en approcher lui était insupportable. Enfin, n'y pouvant plus tenir,
il fit demander au roi la permission d'aller se promener en Angleterre,
où on jouait beaucoup et fort gros. Il l'obtint, et il y porta beaucoup
d'argent qui le fit recevoir à bras ouverts à Londres, où il ne fut pas
moins heureux qu'à Paris.

Jacques II y régnait, qui le reçut avec distinction. La révolution s'y
brassait déjà. Elle éclata au bout de huit ou dix mois que Lauzun fut en
Angleterre. {[}Elle{]} sembla faite exprès pour lui par le succès qui
lui en revint et qui n'est ignoré de personne. Jacques II, ne sachant
plus ce qu'il allait devenir, trahi par ses favoris et ses ministres,
abandonné de toute sa nation, le prince d'Orange maître des coeurs, des
troupes et des flottes, et près d'entrer dans Londres, le malheureux
monarque confia à Lauzun ce qu'il avait de plus cher, la reine et le
prince de Galles qu'il passa heureusement à Calais. Cette princesse
dépêcha aussitôt un courrier à Versailles qui suivit de près celui que
le duc de Charost, qui prit depuis le nom de duc de Béthune, gouverneur
de Calais, et qui y était alors, avait envoyé à l'instant de l'arrivée
de la reine. Cette princesse, après les compliments, insinua dans sa
lettre que, parmi la joie de se voir en sûreté sous la protection du
roi, avec son fils, elle avait la douleur de n'oser mener à ses pieds
celui à qui elle devait de l'avoir sauvée avec le prince de Galles. La
réponse du roi, après tout ce qu'il y mit de généreux et de galant, fut
qu'il partageait cette obligation avec elle, et qu'il avait hâte de lui
témoigner en revoyant le comte de Lauzun et lui rendant ses bonnes
grâces. En effet, lorsqu'elle le présenta au roi dans la plaine de
Saint-Germain, où le roi avec la famille royale et toute sa cour vint au
devant d'elle, il traita Lauzun parfaitement bien, lui rendit là même
les grandes entrées et lui promit un logement au château de Versailles
qu'il lui donna incontinent après\,; et de ce jour-là il en eut un à
Marly tous les voyages et à Fontainebleau, en sorte que jusqu'à la mort
du roi il ne quitta plus la cour. On peut juger quel fut le ravissement
d'un courtisan aussi ambitieux, qu'un retour si éclatant et si unique
ramenait des abîmes et remettait subitement à flot. Il eut aussi un
logement dans le château de Saint-Germain choisi pour le séjour de cette
cour fugitive, où le roi Jacques II arriva bientôt après.

Lauzun y fit tout l'usage qu'un habile courtisan sait faire de l'une et
l'autre cour, et de se procurer par celle d'Angleterre les occasions de
parler souvent au roi, et d'en recevoir des commissions. Enfin, il sut
si bien s'en aider que le roi lui permit de recevoir dans Notre-Dame, à
Paris, l'ordre de la Jarretière des mains du roi d'Angleterre, le lui
accorda à son second passage en Irlande pour général de son armée
auxiliaire, et permît qu'il le fût en même temps de celle du roi
d'Angleterre, qui la même campagne perdit l'Irlande avec la bataille de
la Boyne, et revint en France avec le comte de Lauzun, pour lequel enfin
il obtint des lettres de due, qui furent vérifiées au parlement, en mai
1692. Quel miraculeux retour de fortune\,! Mais quelle fortune en
comparaison du mariage public avec Mademoiselle, avec la donation de
tous ses biens prodigieux, et le titre et la dignité actuelle de duc et
pair de Montpensier\,! Quel monstrueux piédestal, et avec des enfants de
ce mariage, quel vol n'eût pas pris Lauzun, et qui peut dire jusqu'où il
serait arrivé\,?

J'ai raconté ailleurs ses humeurs, ses insignes malices et ses rares
singularités. Il jouit le reste de sa longue vie de ses privances avec
le roi, de ses distinctions à la cour, d'une grande considération, d'une
abondance extrême, de la vie et du maintien d'un très grand seigneur et
de l'agrément de tenir une des plus magnifiques maisons de la cour, et
de la meilleure table, soir et matin, la plus honorablement fréquentée,
et à Paris de même après la mort du roi. Tout cela ne le contentait
point. Il n'approchait familièrement du roi que par les dehors\,; il
sentait l'esprit et le coeur de ce monarque en garde contré lui, et dans
un éloignement que tout son art son application ne purent jamais
rapprocher. C'est ce qui lui fit épouser ma belle-soeur dans le projet
de se remettre en commerce sérieux avec le roi, à l'occasion que l'armée
de M. le maréchal de Lorge commandait en Allemagne, et ce qui le
brouilla avec lui sitôt après avec éclat, quand il vit ses desseins
échoués de ce côté-là. C'est ce qui lui fit faire le mariage du duc de
Lorge avec la fille de Chamillart pour se raccrocher par le crédit de ce
ministre, sans y avoir pu réussir. C'est ce qui lui fit faire le voyage
d'Aix-la-Chapelle, sous prétexte des eaux, pour y lier et y prendre des
connaissances qui le portassent à des particuliers avec le roi sur la
paix, ce qui lui fut encore inutile\,; c'est enfin ce qui le porta aux
extravagances qu'il fit de prétendue jalousie du fils presque enfant de
Chamillart pour faire peur au père, et l'engager à l'éloigner par
l'ambassade pour traiter de la paix. Tout lui manquant dans ses divers
projets, il s'affligeait sans cesse, et se croyait et se disait dans une
profonde disgrâce. Rien ne lui échappait pour faire sa cour avec un fond
de bassesse et un extérieur de dignité\,; et il faisait tous les ans une
sorte d'anniversaire de sa disgrâce par quelque chose d'extraordinaire,
dont l'humeur et la solitude était le fond, et souvent quelque
extravagance le fruit. Il en parlait lui-même, et disait qu'il n'était
pas raisonnable au retour annuel de cette époque, plus forte que lui. Il
croyait plaire au roi par ce raffinement de courtisan, sans s'apercevoir
qu'il s'en faisait moquer.

Il était extraordinaire en tout par nature, et se plaisait encore à
l'affecter, jusque dans le plus intérieur de son domestique et de ses
valets. Il contrefaisait le sourd et l'aveugle pour mieux voir et
entendre sans qu'on s'en défiât, et se divertissait à se moquer des
sots, même des plus élevés, en leur tenant des langages qui n'avaient
aucun sens. Ses manières étaient toutes mesurées, réservées,
doucereuses, même respectueuses\,; et de ce ton bas et emmiellé il
sortait des traits perçants et accablants par leur justesse, leur force
ou leur ridicule, et cela en deux ou trois mots, quelquefois d'un air de
naïveté ou de distraction, comme s'il n'y eût pas songé. Aussi était-il
redouté sans exception de tout le monde, et avec force connaissances, il
n'avait que peu ou point d'amis, quoiqu'il en méritât par son ardeur à
servir tant qu'il pouvait, et sa facilité à ouvrir sa bourse. Il aimait
à recueillir les étrangers de quelque distinction, et faisait
parfaitement les honneurs de la cour\,; mais ce ver rongeur d'ambition
empoisonnait sa vie. Il était très bon et très secourable parent.

Nous avions fait le mariage de M\textsuperscript{lle} de Malause,
petite-fille d'une soeur de M. le maréchal de Lorge, un an avant la mort
du roi, avec le comte de Poitiers, dernier de cette grande et illustre
maison, fort riche en grandes terres en Franche-Comté, tous deux sans
père ni mère. Il en fit la noce chez lui et les logea. Le comte de
Poitiers mourut presque en même temps que le roi, dont ce fut grand
dommage, car il promettait fort, et laissa sa femme grosse d'une fille,
grande héritière, qui a depuis épousé le duc de Randan, fils aîné du duc
de Lorge, et dont la conduite a fait honneur à la naissance. Dans l'été
qui suivit la mort de Louis XIV, il eut une revue de la maison du roi
que M. le duc d'Orléans fit dans la plaine qui longe le bois de
Boulogne. Passy y tient de l'autre côté, où M. de Lauzun avait une jolie
maison. M\textsuperscript{me} de Lauzun y était avec bonne compagnie, et
j'y étais allé coucher la veille de cette revue. M\textsuperscript{me}
de Poitiers mourait d'envie de la voir, comme une jeune personne qui n'a
rien vu encore, mais qui n'osait se montrer dans ce premier deuil de
veuve. Le comment fut agité dans la compagnie, et on trouva que
M\textsuperscript{me} de Lauzun l'y pouvait mener un peu enfoncée dans
son carrosse, et cela fut conclu ainsi. Parmi la gaieté de cette partie,
M. de Lauzun arriva de Paris, où il était allé le matin. On tourna un
peu pour la lui dire. Dès qu'il l'apprit, le voilà en furie jusqu'à ne
se posséder plus, à la rompre presque en écumant, et à dire à sa femme
les choses les plus désobligeantes avec les termes non seulement les
plus durs, mais les plus forts, les plus injurieux et les plus fous.
Elle s'en prit doucement à ses yeux, M\textsuperscript{me} de Poitiers à
pleurer aux sanglots, et toute la compagnie dans le plus grand embarras.
La soirée parut une année, et le plus triste réfectoire un repas de
gaieté en comparaison du souper. Il fut farouche au milieu du plus
profond silence, chacun à peine et rarement disait un mot à son voisin.
Il quitta la table au fruit, à son ordinaire, et s'alla coucher. On
voulut après se soulager et en dire quelque chose, mais
M\textsuperscript{me} de Lauzun arrêta tout poliment et sagement, et fit
promptement donner des cartes pour détourner tout retour de propos.

Le lendemain, dès le matin, j'allai chez M. de Lauzun pour lui dire très
fortement mon avis de la scène qu'il avait faite la veille. Je n'en eus
pas le temps\,; dès qu'il me vit entrer il étendit les bras, et s'écria
que je voyais un fou qui ne méritait pas ma visite, mais les
petites-maisons, fit le plus grand éloge de sa femme, qu'elle méritait
assurément\,; dit qu'il n'était pas digne de l'avoir, et qu'il devait
baiser tous les pas par où elle passait\,; s'accabla de pouilles\,;
puis, les larmes aux yeux, me dit qu'il était plus digue de pitié que de
colère\,; qu'il fallait m'avouer toute sa honte et sa misère\,: qu'il
avait plus de quatre-vingts ans\,; qu'il n'avait ni enfants ni
suivants\,; qu'il avait été capitaine des gardes\,; que, quand il le
serait encore, il serait incapable d'en faire les fonctions\,; qu'il se
le disait sans cesse, et qu'avec tout cela il ne pouvait se consoler de
ne l'être plus, depuis tant d'années qu'il avait perdu sa charge\,;
qu'il n'en avait jamais pu arracher le poignard de son coeur\,; que tout
ce qui lui en rappelait le souvenir le mettait hors de lui-même, et que
d'entendre dire que sa femme allait mener M\textsuperscript{me} de
Poitiers voir une revue des gardes du corps, où il n'était plus rien,
lui avait renversé la tète, et {[}l'avait{]} rendu extravagant au point
où je l'avais vu\,; qu'il n'osait plus se montrer devant personne après
ce trait de folie\,; qu'il s'allait enfermer dans sa chambre, et qu'il
se jetait à mes pieds pour me conjurer d'aller trouver sa, femme, et de
tacher d'obtenir qu'elle voulût avoir pitié d'un vieillard insensé, qui
mourait de douleur et de honte, et qu'elle daignât lui pardonner. Cet
aveu si sincère et si douloureux à faire, me pénétra. Je ne cherchai
plus qu'à le remettre et à le consoler. Le raccommodement ne fut pas
difficile\,; nous le tirâmes de sa chambre, non sans peine, et il lui en
parut visiblement une fort grande pendant plusieurs jours à se montrer,
à ce qu'on m'a dit, car je m'en allai le soir, mes occupations, dans ce
temps-là, me tenant de fort court.

J'ai réfléchi souvent, à cette occasion, sur l'extrême malheur de se
laisser entraîner à l'ivresse du monde, et au formidable état d'un
ambitieux que ni les richesses, ni le domestique le plus agréable, ni la
dignité acquise, ni l'âge, ni l'impuissance corporelle, n'en peuvent
déprendre, et qui, au lieu de jouir tranquillement de ce qu'il possède,
et d'en sentir le bonheur, s'épuise en regrets et en amertumes inutiles
et continuelles, et qui ne peut se représenter que, sans enfants et dans
un âge qui l'approche si fort de sa fin, posséder ce qu'il regrette,
quand même il pourrait l'exercer, serait des liens trompeurs qui
l'attacheraient à la vie, si prête à lui échapper, qui ne lui seraient
bons qu'à lui augmenter les regrets cuisants de la quitter. Mais on
meurt comme on a vécu, et il est rare que cela arrive autrement. De
quelle importance n'est-il donc pas de n'oublier rien pour tâcher de
vivre pour savoir mourir au monde et à la fortune avant que l'un et
l'autre et que la vie nous quittent, pour savoir vivre sans eux, et
tâcher et espérer de bien mourir ! Cette folie de capitaine des gardes
dominait si cruellement le duc de Lauzun, qu'il s'habillait souvent d'un
habit bleu à galons d'argent, qui, sans oser être semblable à l'uniforme
des capitaines des gardes du corps aux jours de revue, ou de changement
du guet, en approchait tant qu'il pouvait, mais bien plus de celui des
capitaines des chasses des capitaineries royales, et l'aurait rendu
ridicule si, à force de singularités et de ridicules, il n'y eût
accoutumé le monde, qui le craignait, et ne se fût rendu supérieur à
tous les ridicules.

Avec toute sa politique et sa bassesse, il tombait sur tout le monde\,;
toujours par un mot asséné le plus perçant, toujours en toute douceur.
Les ministres, les généraux d'armée, les gens heureux et leurs familles
étaient les plus maltraités. Il avait comme usurpé un droit de tout dire
et de tout faire sans que qui que ce fût osât s'en fâcher. Les seuls
Grammont étaient exceptés. Il se souvenait toujours de l'hospitalité et
de la protection qu'il avait trouvées chez eux au commencement de sa
vie. Il les aimait, il s'y intéressait\,; il était en respect devant
eux. Le vieux comte de Grammont en abusait et vengeait la cour par les
brocards qu'il lui lâchait à tout propos, sans que le duc de Lauzun lui
en rendit jamais aucun, ni s'en fâchât, mais il l'évitait doucement
volontiers. Il fit toujours beaucoup pour les enfants de ses soeurs. On
a vu ici en son temps combien l'évêque de Marseille s'était signalé à la
peste, et de ses biens et de sa personne. Quand elle fut tout à fait
passée, M. de Lauzun demanda une abbaye pour lui à M. le duc d'Orléans.
Il donna les bénéfices peu après et oublia M. de Marseille. M. de Lauzun
voulut l'ignorer, et demanda à M. le duc d'Orléans s'il avait eu la
bonté de se souvenir de lui. Le régent fut embarrassé. Le duc de Lauzun,
comme pour lever l'embarras, lui dit d'un ton doux et respectueux\,:
«\,Monsieur, il fera mieux une autre fois,\,» et avec ce sarcasme rendit
le régent muet, et s'en alla en souriant. Le mot courut fort, et M. le
duc d'Orléans, honteux, répara son oubli par l'évêché de Laon, et sur le
refus de M. de Marseille de changer d'épouse, il lui donna une grosse
abbaye, quoique M. de Lauzun fût mort.

Il empêcha une promotion de maréchaux de France par le ridicule qu'il y
donna aux candidats qui la pressaient. Il dit au régent, avec ce même
ton respectueux et doux, qu'au cas qu'il fît, comme on le disait, des
maréchaux de France inutiles, il le suppliait de se souvenir qu'il était
le plus ancien lieutenant général du royaume, et qu'il avait eu
l'honneur de commander des armées avec la patente de général. J'en ai
rapporté ailleurs de fort salées. Il ne se pouvait tenir là-dessus\,;
l'envie et la jalousie y avaient la plus grande part, et comme ses bons
mots étaient toujours fort justes et fort pointus, ils étaient fort
répétés.

Nous vivions ensemble en commerce le plus continuel, il m'avait même
rendu de vrais services, solides et d'amitié, de lui-même, et j'avais
pour lui toutes sortes d'attentions et d'égards, et lui pour moi.
Néanmoins je ne pus échapper à sa langue par un trait qui devait me
perdre, et je ne sais comment ni pourquoi il ne fit que glisser. Le roi
baissait, il le sentait\,; il commençait à songer pour après lui. Les
rieurs n'étaient pas pour M. le duc d'Orléans\,: on voyait pourtant sa
grandeur s'approcher. Tous les yeux étaient sur lui et l'éclairaient
avec malignité, par conséquent sur moi, qui depuis longtemps étais le
seul homme de la cour qui lui fût demeuré attaché publiquement, et qu'on
voyait le seul dans toute sa confiance. M. de Lauzun vint pour dîner
chez moi, et nous trouva à table. La compagnie qui s'y trouva lui déplut
apparemment, il s'en alla chez Torcy, avec qui alors je n'étais en nul
commerce, qui était aussi à table avec beaucoup de gens opposés à M. le
duc d'Orléans, Tallard entre autres et Tessé. «\,Monsieur, dit-il à
Torcy avec cet air doux et timide qui lui était si familier, prenez
pitié de moi, je viens de chercher à dîner avec M. de Saint-Simon\,; je
l'ai trouvé à table avec compagnie\,; je me suis gardé de m'y mettre\,;
je n'ai pas voulu être le reste de la cabale, je m'en suis venu ici en
chercher.\,» Les voilà tous à rire. Ce mot courut tout Versailles à
l'instant\,; M\textsuperscript{me} de Maintenon et M. du Maine le surent
aussitôt, et, toutefois, on ne m'en fit pas le moindre semblant\,; m'en
fâcher n'eût fait qu'y donner plus de cours\,; je pris la chose comme
l'égratignure au sang d'un mauvais chat, et je ne laissai pas apercevoir
à Lauzun que je le susse.

Trois ou quatre ans avant sa mort, il eut une maladie qui le mit à
l'extrémité. Nous y étions tous fort assidus, il ne voulut voir pas un
de nous que M\textsuperscript{me} de Saint-Simon une seule fois.
Languet, curé de Saint-Sulpice, y venait souvent, et perçait quelquefois
jusqu'à lui, qui tenait des discours admirables. Un jour qu'il y était,
le duc de La Force se glissa dans sa chambre\,; M. de Lauzun ne l'aimait
point du tout, et s'en moquait souvent. Il le reçut assez bien, et
continua d'entretenir tout haut le curé. Tout d'un coup il se tourne à
lui, lui fait des compliments et des remerciements, lui dit qu'il n'a
rien à lui donner de plus cher que sa bénédiction, tire son bras du lit,
la prononce et la lui donne\,; tout de suite se tourne au duc de La
Force, lui dit qu'il l'a toujours aimé et respecté comme l'aîné et le
chef de sa maison, et qu'en cette qualité il lui demande sa bénédiction.
Ces deux hommes demeurent confondus, et d'étonnement, sans proférer un
mot. Le malade redouble ses instances\,; M. de La Force, revenu à soi,
trouve la chose si plaisante qu'il lui donne sa bénédiction\,; et, dans
la crainte d'éclater, sort à l'instant et nous revient trouver dans la
pièce joignante, mourant de rire et pouvant à peine nous raconter ce qui
venait de lui arriver. Un moment après le curé sortit aussi, l'air fort
consterné, souriant tant qu'il pouvait pour faire bonne mine. Le malade,
qui le savait ardent et adroit à tirer des gens pour le bâtiment de son
église, avait dit souvent qu'il ne serait jamais de ses grues\,; il
soupçonna ses assiduités d'intérêt, et se moqua de lui en ne lui donnant
que sa bénédiction qu'il devait recevoir de lui, et du duc de La Force,
en même temps, en lui demandant persévéramment la sienne. Le curé, qui
le sentit, en fut très mortifié, et, en homme d'esprit, il ne le revit
pas moins, mais M. de Lauzun abrégeait les visites, et ne voulut point
entendre le français.

Un autre jour qu'on le tenait fort mal, Biron et sa femme, fille de
M\textsuperscript{me} de Nogent, se hasardèrent d'entrer sur la pointe
du pied, et se tinrent derrière ses rideaux, hors de sa vue\,; mais il
les aperçut par la glace de la cheminée lorsqu'ils se persuadaient n'en
pouvoir être ni vus ni entendus. Le malade aimait assez Biron, mais
point du tout sa femme qui était pourtant sa nièce et sa principale
héritière, il la croyait fort intéressée, et toutes ses manières lui
étaient insupportables. En cela il était comme tout le monde. Il fut
choqué de cette entrée subreptice dans sa chambre, et comprit
qu'impatiente de l'héritage, elle venait pour tâcher de s'assurer par
elle-même s'il mourrait bientôt. Il voulut l'en faire repentir, et s'en
divertir d'autant. Le voilà donc qu'il se prend tout d'un coup à faire
tout haut, comme se croyant tout seul, une oraison éjaculatoire, à
demander pardon à Dieu de sa vie passée, à s'exprimer comme un homme
bien persuadé de sa mort très prochaine, et qui dit que dans la douleur
où son impuissance le met de faire pénitence, il veut au moins se servir
de tous les biens que Dieu lui a donnés pour en racheter ses péchés, et
les léguer tous aux hôpitaux sans aucune réserve\,; que c'est l'unique
voie que Dieu lui laisse ouverte pour faire son salut après une si
longue vie passée sans y avoir jamais pensé comme il faut, et à
remercier Dieu de cette unique ressource qu'il lui laisse et qu'il
embrasse de tout son coeur. Il accompagna cette prière et cette
résolution d'un ton si touché, si persuadé, si déterminé, que Biron et
sa femme ne doutèrent pas un moment qu'il n'allât exécuter ce dessein,
et qu'ils ne fussent privés de toute la succession. Ils n'eurent pas
envie d'épier là davantage, et vinrent, confondus, conter à la duchesse
de Lauzun l'arrêt cruel qu'ils venaient d'entendre, et la conjurer d'y
apporter quelque modération. Là-dessus, le malade envoie chercher des
notaires, et voilà M\textsuperscript{me} de Biron éperdue. C'était bien
le dessein du testateur de la rendre telle. Il fit attendre les
notaires, puis les fit entrer, et dicta son testament qui fut un coup de
mort pour M\textsuperscript{me} de Biron. Néanmoins il différa de le
signer, et, se trouvant de mieux en mieux, ne le signa point. Il se
divertit beaucoup de cette comédie, et ne put s'empêcher d'en rire avec
quelques-uns quand il fut rétabli. Malgré son âge et une si grande
maladie, il revint promptement en son premier état sans qu'il y parût en
aucune sorte.

C'était une santé de fer avec les dehors trompeurs de la délicatesse. Il
dînait et soupait à fond tous les jours, faisait très grande chère et
très délicate, toujours avec bonne compagnie soir et matin, mangeait de
tout, gras et maigre, sans nulle sorte de choix que son goût, ni de
ménagement\,; prenait du chocolat le matin, et avait toujours sur
quelque table des fruits dans leur saison, des pièces de four dans
d'autres temps, de la bière, du cidre, de la limonade, d'autres liqueurs
pareilles à la glace, et allant et venant, en mangeait et en bu voit
toutes les après-dînées, et exhortait les autres à en faire autant\,; il
sortait de table le soir au fruit, et s'allait coucher tout de suite. Je
me souviens qu'une fois entre bien d'autres, il mangea chez moi, après
cette maladie, tant de poisson, de légumes et de toutes sortes de choses
sans pouvoir l'en empêcher, que nous envoyâmes le soir chez lui savoir
doucement s'il ne s'en était point fortement senti\,: on le trouva à
table qui mangeait de bon appétit. La galanterie lui dura fort
longtemps. Mademoiselle en fut jalouse, cela les brouilla à plusieurs
reprises. J'ai ouï dire à M\textsuperscript{me} de Fontenilles, femme
très aimable, de beaucoup d'esprit, très vraie et d'une singulière
vertu, depuis un très grand nombre d'années, qu'étant à Eu avec
Mademoiselle, M. de Lauzun y vint passer quelque temps, et ne put
s'empêcher d'y courir des filles\,; Mademoiselle le sut, s'emporta,
l'égratigna, le chassa de sa présence. La comtesse de Fiesque fit le
raccommodement\,: Mademoiselle parut au bout d'une galerie\,; il était à
l'autre bout, et il en fit toute la longueur sur ses genoux jusqu'aux
pieds de Mademoiselle. Ces scènes, plus ou moins fortes, recommencèrent
souvent dans les suites. Il se lassa d'être battu, et à son tour battit
bel et bien Mademoiselle, et cela arriva plusieurs fois, tant qu'à la
fin, lassés l'un de l'autre, ils se brouillèrent une bonne fois pour
toutes, et {[}ne{]} se revirent jamais depuis\,; il en avait pourtant
plusieurs portraits chez lui, et n'en parlait qu'avec beaucoup de
respect. On ne doutait pas qu'ils ne se fussent mariés en secret. À sa
mort, il prit une livrée presque noire, avec des galons d'argent, qu'il
changea en blancs, avec un peu de bleu quand l'or et l'argent furent
défendus aux livrées.

Son humeur naturelle triste et difficile, augmentée par la prison et
l'habitude de la solitude, l'avait rendu solitaire et rêveur, en sorte
qu'ayant chez lui la meilleure compagnie, il la laissait avec
M\textsuperscript{me} de Lauzun, et se retirait tout seul des
après-dînées entières, mais toujours plusieurs heures de suite, sans
livre, le plus souvent, car il ne lisait que des choses de fantaisie,
sans suite, et fort peu\,; en sorte qu'il ne savait rien que ce qu'il
avait vu, et jusqu'à la fin tout occupé de la cour et des nouvelles du
monde. J'ai regretté mille fois son incapacité radicale d'écrire ce
qu'il avait vu et fait. C'eût été un trésor des plus curieuses
anecdotes, mais il n'avait nulle suite ni application. J'ai souvent
essayé de tirer de lui quelques bribes. Autre misère. Il commençait à
raconter\,; dans le récit, il se trouvait d'abord des noms de gens qui
avaient eu part à ce qu'il voulait raconter. II quittait aussitôt
l'objet principal du récit pour s'attacher à quelqu'une de ces
personnes, et tôt après à une autre personne qui avait rapport à cette
première, puis à une troisième, et à la manière des romans\,; il
enfilait ainsi une douzaine d'histoires à la fois qui faisaient perdre
terre, et se chassaient l'une l'autre, sans jamais en finir pas une, et
avec cela le discours fort confus, de sorte qu'il n'était pas possible
de rien apprendre de lui, ni d'en rien retenir. Du reste, sa
conversation était toujours contrainte par l'humeur ou par la politique,
et n'était plaisante que par sauts et par les traits malins qui en
sortaient souvent. Peu de mois avant sa dernière maladie, c'est-à-dire à
plus de quatre-vingt-dix ans, il dressait encore des chevaux, et il fit
cent passades au bois de Boulogne, devant le roi qui allait à la Muette,
sur un poulain qu'il venait de dresser, et qui à peine l'était encore,
où il surprit les spectateurs par son adresse, sa fermeté et sa bonne
grâce. On ne finirait point à raconter de lui.

Sa dernière maladie se déclara sans prélude, presque en un moment, par
le plus horrible de tous les maux, un cancer dans la bouche. Il le
supporta jusqu'à la fin avec une fermeté et une patience incroyable,
sans plainte, sans humeur, sans le moindre contre-temps, lui qui en
était insupportable à lui-même. Quand il se vit un peu avancé dans son
mal, il se retira dans un petit appartement qu'il avait d'abord loué
dans cette vue dans l'intérieur du couvent des Petits-Augustins, dans
lequel on entrait de sa maison, pour y mourir en repos, inaccessible à
M\textsuperscript{me} de Biron et à toute autre femme, excepté à la
sienne, qui eut permission d'y entrer à toutes heures, suivie d'une de
ses femmes.

Dans cette dernière retraite, le duc de Lauzun n'y donna accès qu'à ses
neveux et à ses beaux-frères, et encore le moins et le plus courtement
qu'il put. Il ne songea qu'à mettre à profit son état horrible, et à
donner tout son temps aux pieux entretiens de son confesseur et de
quelques religieux de la maison, à de bonnes lectures, et à tout ce qui
pouvait le mieux préparer à la mort. Quand nous le voyions, rien de
malpropre, rien de lugubre, rien de souffrant\,; politesse,
tranquillité, conversation peu animée, fort indifférente à ce qui se
passait dans le monde, en parlant peu et difficilement\,; toutefois,
pour parler de quelque chose, peu ou point de morale, encore moins de
son état, et cette uniformité si courageuse et si paisible se soutint
égale quatre mois durant, jusqu'à la fin\,; mais, les dix ou douze
derniers jours, il ne voulut plus voir ni beaux-frères ni neveux\,; et
sa femme, il la renvoyait promptement. Il reçut tous les sacrements avec
beaucoup d'édification, et conserva sa tête entière jusqu'au dernier
moment.

Le matin du jour, dont il mourut la nuit suivante, il envoya chercher
Biron, lui dit qu'il avait fait pour lui tout ce que
M\textsuperscript{me} de Lauzun avait voulu\,; que, par son testament,
il lui donnait tous ses biens, excepté un legs assez médiocre à
Castelmoron, fils de son autre soeur, et des récompenses à ses
domestiques\,; que tout ce qu'il avait fait pour lui depuis son mariage,
et ce qu'il faisait en mourant, Biron le devait en entier à
M\textsuperscript{me} de Lauzun\,; qu'il n'en devait jamais oublier la
reconnaissance\,; qu'il lui défendait, par l'autorité d'oncle et de
testateur, de lui faire jamais ni peine, ni trouble, ni obstacle, et
d'avoir jamais aucun procès contre elle sur quoi que ce prit être. C'est
Biron lui-même qui me le dit le lendemain, dans les mêmes termes que je
les rapporte. {[}M. de Lauzun{]} lui dit adieu d'un ton ferme, et le
congédia. Il défendit, avec raison, toute cérémonie\,; il fut enterré
aux Petits-Augustins\,; il n'avait rien du roi que cette ancienne
compagnie des becs de corbin, qui fut supprimée deux jours après. Un
mois avant sa mort il avait envoyé chercher Dillon, chargé ici des
affaires du roi Jacques, et officier général très distingué, à qui il
remit son collier de l'ordre de la Jarretière, et un Georges d'onyx
entouré de parfaitement beaux et gros diamants, pour les renvoyer à ce
prince. Je m'aperçois enfin que j'ai été bien prolixe sur un homme, dont
la singularité extraordinaire de sa vie et le commerce continuel que la
proximité m'a donné avec lui m'a paru mériter de le faire connaître,
d'autant qu'il n'a pas assez figuré dans les affaires générales pour en
attendre rien des histoires qui paraîtront.

Un autre sentiment a allongé mon récit. Je touche à un but que je crains
d'atteindre, parce que mes désirs n'y peuvent s'accorder avec la
vérité\,; ils sont ardents, par conséquent cuisants, parce que l'autre
est terrible et ne laisse pas le moindre lieu à oser chercher à se la
pallier\,; cette horreur d'y venir enfin m'a arrêté, m'a accroché où
j'ai pu, m'a glacé. On entend bien qu'il s'agit de venir à la mort et au
genre de mort de M. le duc d'Orléans, et quel récit épouvantable,
surtout après un tel et si long attachement, puisqu'il a duré en moi
pendant toute sa vie, et qu'il durera toute la mienne pour me pénétrer
d'effroi et de douleur sur lui. On frémit jusque dans les moelles, par
l'horreur du soupçon que Dieu l'exauça dans sa colère.

\hypertarget{chapitre-iv.}{%
\chapter{CHAPITRE IV.}\label{chapitre-iv.}}

1723

~

{\textsc{Mort subite de M. le duc d'Orléans.}} {\textsc{- Diligence de
La Vrillière à se capter M. le Duc.}} {\textsc{- Le roi affligé.}}
{\textsc{- M. le Duc premier ministre.}} {\textsc{- Lourdise de M. le
duc de Chartres.}} {\textsc{- Je vais au lever du roi et j'y prends un
rendez-vous avec M. le Duc.}} {\textsc{- Je vais parler à la duchesse
Sforze, puis chez M\textsuperscript{me} la duchesse d'Orléans et chez M.
le duc de Chartres.}} {\textsc{- Leur réception.}} {\textsc{-
Conversation entre M. le Duc et moi dans son cabinet tête à tête.}}
{\textsc{- Je m'en retourne à Meudon.}} {\textsc{- M\textsuperscript{me}
de Saint-Simon à Versailles pour voir le roi, etc., sans y coucher\,; y
reçoit la visite de l'évêque de Fréjus et de La Vrillière\,; entrevoit
que le premier ne me désire pas à la cour, et que le dernier m'y
craint.}} {\textsc{- Je me confirme dans la résolution de longtemps
prise\,: nous allons à Paris nous y fixer.}} {\textsc{- Monseigneur et
M. le duc d'Orléans morts au même âge.}} {\textsc{- Effet de la mort de
M. le duc d'Orléans chez les étrangers, dans la cour, dans l'Église,
dans le parlement et toute la magistrature, dans les troupes, dans les
marchands et le peuple.}} {\textsc{- Obsèques de M. le duc d'Orléans.}}
{\textsc{- Visites du roi.}} {\textsc{- Maréchal de Villars entre dans
le conseil.}} {\textsc{- Indépendance {[}à l'égard{]} du grand écuyer
confirmée au premier écuyer.}} {\textsc{- Faute du grand écuyer par
dépit, dont le grand maître de France profite.}} {\textsc{- Mécanique
des comptes des diverses dépenses domestiques du roi à passer à la
chambre des comptes.}} {\textsc{- Mort de Beringhen, premier écuyer.}}
{\textsc{- Fortune de son frère, qui obtient sa charge.}} {\textsc{-
Nangis chevalier d'honneur de la future reine.}} {\textsc{- Le maréchal
de Tessé premier écuyer de la future reine, avec la survivance pour son
fils, et va ambassadeur en Espagne.}} {\textsc{- Mort de la maréchale
d'Humières.}} {\textsc{- Comte de Toulouse déclare son mariage.}}
{\textsc{- Novion fait premier président avec force grâces.}} {\textsc{-
Sa famille, son caractère, sa démission, sa mort.}} {\textsc{- Crozat et
Montargis vendent à regret leurs charges de l'ordre à Dodun et à
Maurepas, dont le râpé est donné à d'Armenonville, garde des sceaux, et
à Novion, premier président.}}

~

On a vu, il y a peu, qu'il {[}le duc d'Orléans{]} redoutait une mort
lente qui s'annonçait de loin, qui devient une grâce bien précieuse
quand celle d'en savoir bien profiter y est ajoutée, et que la mort la
plus subite fut celle qu'il préférait\,; hélas\,! il l'obtint, et plus
rapide encore que ne fut celle de feu Monsieur, dont la machine disputa
plus longtemps. J'allai, le 21 décembre, de Meudon à Versailles, au
sortir de table, chez M. le duc d'Orléans\,; je fus trois quarts d'heure
seul avec lui dans son cabinet, où je l'avais trouvé seul. Nous nous y
promenâmes toujours parlant d'affaires, dont il allait rendre compte au
roi ce jour-là même. Je ne trouvai nulle différence à son état
ordinaire, épaissi et appesanti depuis quelque temps, mais l'esprit net
et le raisonnement tel qu'il l'eut toujours. Je revins tout de suite à
Meudon\,; j'y causai en arrivant avec M\textsuperscript{me} de
Saint-Simon quelque temps. La saison faisait que nous y avions peu de
monde, je la laissai dans son cabinet et je m'en allai dans le mien.

Au bout d'une heure au plus, j'entends des cris et un vacarme subit\,;
je sors, et je trouve M\textsuperscript{me} de Saint-Simon tout effrayée
qui m'amenait un palefrenier du marquis de Ruffec, qui de Versailles me
mandait que M. le duc d'Orléans était en apoplexie. J'en fus vivement
touché, mais nullement surpris\,; je m'y attendais, comme on a vu,
depuis longtemps. Je pétille après ma voiture qui me fit attendre par
l'éloignement du château neuf aux écuries, je me jette dedans et m'en
vais tant que je puis. À la porte du parc, autre courrier du marquis de
Ruffec qui m'arrête, et qui m'apprend que c'en est fait. Je demeurai là
plus d'une demi-heure absorbé en douleur et en réflexions. À la fin je
pris mon parti d'aller à Versailles, où j'allai tout droit m'enfermer
dans mon appartement. Nangis, qui voulait être premier écuyer, aventure
dont je parlerai après, m'avait succédé chez M. le duc d'Orléans, et
expédié en bref, le fut par M\textsuperscript{me} Falari, aventurière
fort jolie, qui avait épousé un autre aventurier, frère de la duchesse
de Béthune. C'était une des maîtresses de ce malheureux prince. Son sac
était fait pour aller travailler chez le roi, et il causa près d'une
heure avec elle en attendant celle du roi. Comme elle était tout proche,
assis près d'elle chacun dans un fauteuil, il se laissa tomber de côté
sur elle, et oncques depuis n'eut pas le moindre rayon de connaissance,
pas la plus légère apparence.

La Falari, effrayée au point qu'on peut imaginer, cria au secours de
toute sa force, et redoubla ses cris. Voyant que personne ne répondait,
elle appuya comme elle put ce pauvre prince sur les deux bras contigus
des deux fauteuils, courut dans le grand cabinet, dans la chambre, dans
les antichambres sans trouver qui que ce soit, enfin dans la cour et
dans la galerie basse. C'était sur l'heure du travail avec le roi, que
les gens de M. le duc d'Orléans étaient sûrs que personne ne venait chez
lui, et qu'il n'avait que faire d'eux parce qu'il montait seul chez le
roi par le petit escalier de son caveau, c'est-à-dire de sa garde-robe,
qui donnait dans la dernière antichambre du roi, où celui qui portait
son sac l'attendait, et s'était à l'ordinaire rendu par le grand
escalier et par la salle des gardes. Enfin la Falari amena du monde,
mais point de secours qu'elle envoya chercher par qui elle trouva sous
sa main. Le hasard, ou pour mieux dire, la Providence avait arrangé ce
funeste événement à une heure où chacun était d'ordinaire allé à ses
affaires ou en visite, de sorte qu'il s'écoula une bonne demi-heure
avant qu'il vint ni médecin ni chirurgien, et peu moins pour avoir des
domestiques de M. le duc d'Orléans.

Sitôt que les gens du métier l'eurent envisagé, ils le jugèrent sans
espérance. On l'étendit à la hâte sur le parquet, on l'y saigna\,; il ne
donna pas le moindre signe de vie pour tout ce qu'on, put lui faire. En
un instant que les premiers furent avertis, chacun de toute espèce
accourut\,; le grand et le petit cabinet étaient pleins de monde. En
moins de deux heures tout fut fini, et peu à peu la solitude y fut aussi
grande qu'avait été la foule. Dès que le secours fut arrivé, la Falari
se sauva et gagna Paris au plus vite.

La Vrillière fut des premiers averti de l'apoplexie. Il courut aussitôt
l'apprendre au roi et à l'évêque de Fréjus, puis à M. le Duc, en
courtisan qui sait profiter de tous les instants critiques\,; et dans la
pensée que ce prince pourrait bien être premier ministre, comme il l'y
avait exhorté en l'avertissant, il se hâte de retourner chez lui et d'en
dresser à tout hasard la patente sur celle de M. le duc d'Orléans.
Averti de sa mort au moment même qu'elle arriva, il envoya le dire à M.
le Duc, et s'en alla chez le roi où le danger imminemment certain avait
amassé les gens de la cour les plus considérables.

Fréjus, dès la première nouvelle de l'apoplexie, avait fait l'affaire de
M. le Duc avec le roi qu'il y avait, sans doute, préparé d'avance sur
l'état où on voyait M. le duc d'Orléans, surtout depuis ce que je lui en
avais dit, de sorte que M. le Duc arrivant chez le roi, au moment qu'il
sut la mort, on fit entrer ce qu'il y avait de plus distingué en petit
nombre amassé à la porte du cabinet, où on remarqua le roi fort triste
et les yeux rouges et mouillés. À peine fut-on entré et la porte fermée
que Fréjus dit tout haut au roi que dans la grande perte qu'il faisait
de M. le duc d'Orléans, dont l'éloge ne fut que de deux mots, Sa Majesté
ne pouvait mieux faire que prier M. le Duc là présent de vouloir bien se
charger du poids de toutes les affaires, et d'accepter la place de
premier ministre comme l'avait M. le duc d'Orléans. Le roi, sans dire un
mot, regarda Fréjus, et consentit d'un signe de tête, et tout aussitôt
M. le Duc fit son remerciement. La Vrillière, transporté d'aise de sa
prompte politique, avait en poche le serment de premier ministre copié
sur celui de M. le duc d'Orléans, et proposa tout haut à Fréjus de le
faire prêter sur-le-champ. Fréjus le dit au roi comme chose convenable,
et à l'instant M. le Duc le prêta. Peu après M. le Duc sortit\,; tout ce
qui était dans le cabinet le suivit\,; la foule des pièces voisines
augmenta sa suite, et dans un moment il ne fut plus parlé que de M. le
Duc.

M. le duc de Chartres était à Paris, débauché alors fort gauche, chez
une fille de l'Opéra qu'il entretenait. Il y reçut le courrier qui lui
apprit l'apoplexie, et en chemin un autre qui lui apprit la mort. Il ne
trouva à la descente de son carrosse nulle foule, mais les seuls ducs de
Noailles et de Guiche, qui lui offrirent très apertement leurs services
et tout ce qui pouvait dépendre d'eux. Il les reçut comme des importuns
dont il avait hâte de se défaire, se pressa de monter chez
M\textsuperscript{me} sa mère où il dit qu'il avait rencontré deux
hommes qui lui avaient voulu tendre un bon panneau, mais qu'il n'avait
pas donné dedans, et qu'il avait bien su s'en défaire. Ce grand trait
d'esprit, de jugement et de politique promit d'abord tout ce que ce
prince a tenu depuis. On eut grand'peine à lui faire comprendre qu'il
avait fait une lourde sottise, il ne continua pas moins d'y retomber.

Pour moi, après avoir passé une cruelle nuit, j'allai au lever du roi,
non pour m'y montrer, mais pour y dire un mot à M. le Duc plus sûrement
et plus commodément, avec lequel j'étais en liaison continuelle depuis
le lit de justice des Tuileries, quoique fort mécontent du consentement
qu'il s'était laissé arracher pour le rétablissement des bâtards. Il se
mettait toujours au lever dans l'embrasure de la fenêtre du milieu,
vis-à-vis de laquelle le roi s'habillait\,; et, comme il était fort
grand, on l'apercevait aisément de derrière l'épaisse haie qui
environnait le lever. Elle était ce jour-là prodigieuse. Je fis signe à
M. le Duc de me venir parler, et à l'instant il perça la foule et vint à
moi\,: je le menai dans l'autre embrasure de la fenêtre la plus proche
du cabinet, et là je lui dis que je ne lui dissimulais point que j'étais
mortellement affligé\,; qu'en même temps j'espérais sans peine qu'il
était bien persuadé que si le choix d'un premier ministre avait pu
m'être déféré, je n'en eusse pas fait un autre que celui qui avait été
fait, sur quoi il me fit mille amitiés.

Je lui dis ensuite qu'il y avait dans le sac que M. le duc d'Orléans
devait porter à son travail avec le roi, lors du malheur de cette
cruelle apoplexie, chose sur quoi il était nécessaire que je
l'entretinsse présentement qu'il lui succédait\,; que je n'étais pas en
état de supporter le monde\,; que je le suppliais de m'envoyer avertir
d'aller chez lui sitôt qu'il aurait un moment de libre, et de me faire
entrer par la petite porte de son cabinet qui donnait dans la galerie,
pour m'éviter tout ce monde qui remplirait son appartement. Il me le
promit, et dans la journée, le plus gracieusement, et ajouta des excuses
sur l'embarras du premier jour de son nouvel état, s'il ne me donnait
pas une heure certaine, et celle que je voudrais. Je connaissais ce
cabinet et cette porte, parce que cet appartement avait été celui de
M\textsuperscript{me} la duchesse de Berry, à son mariage, dans la
galerie haute de l'aile neuve, et que le mien était tout proche, de
plain-pied, vis-à-vis de l'escalier.

J'allai de là chez la duchesse Sforze, qui était demeurée toujours fort
de mes amies, et fort en commerce avec moi, quoique je ne visse plus
M\textsuperscript{me} la duchesse d'Orléans depuis longtemps, comme il a
été marqué ici en son lieu. Je lui dis que, dans le malheur qui venait
d'arriver, je me croyais obligé, par respect et attachement pour feu M.
le duc d'Orléans, d'aller mêler ma douleur avec tout ce qui tenait
particulièrement à lui, officiers les plus principaux, même ses bâtards,
quoique je ne connusse aucun d'eux\,; qu'il me paraîtrait fort indécent
d'en excepter M\textsuperscript{me} la duchesse d'Orléans\,; qu'elle
savait la situation où j'étais avec cette princesse, que je n'avais
nulle volonté d'en changer\,; mais qu'en cette occasion si triste je
croyais devoir rendre à la veuve de M. le duc d'Orléans le respect
d'aller chez elle\,: qu'au demeurant, il m'était entièrement indifférent
de la voir ou non, content d'avoir fait à cet égard ce que je croyais
devoir faire\,; qu'ainsi, je la suppliais d'aller savoir d'elle si elle
voulait me recevoir ou non, et, au premier cas, d'une façon convenable,
également content du oui ou du non, parce que je le serais également de
moi-même en l'un et l'autre cas. Elle m'assura que M\textsuperscript{me}
la duchesse d'Orléans serait fort satisfaite de me voir et de me bien
recevoir, et qu'elle allait sur-le-champ s'acquitter de ma commission.
Comme M\textsuperscript{me} Sforze logeait fort près de
M\textsuperscript{me} la duchesse d'Orléans, j'attendis chez elle son
retour. Elle, me dit que M\textsuperscript{me} la duchesse d'Orléans
serait fort aise de me voir, et me recevrait de façon que j'en serais
content. J'y allai donc sur-le-champ.

Je la trouvai au lit avec peu de ses dames et de ses premiers officiers,
et M. le duc de Chartres, avec toute la décence qui pouvait suppléer à
la douleur. Sitôt que j'approchai d'elle, elle me parla du malheur
commun\,; pas un mot de ce qui était entre elle et moi\,; je l'avais
stipulé ainsi. M. le duc de Chartres s'en alla chez lui\,; la
conversation traînante dura tout le moins que je pus. Je m'en allai chez
M. {[}le duc{]} de Chartres, logé dans l'appartement qu'occupait
monsieur son père, avant qu'il fait régent. On me dit qu'il était
enfermé. J'y retournai trois autres fois dans la même matinée. À la
dernière, son premier valet de chambre en fut honteux, et l'alla avertir
malgré moi. Il vint sur le pas de la porte de son cabinet, où il était
avec je ne sais plus qui de fort commun\,: c'était la sorte de gens
qu'il lui fallait. Je vis un homme tout empêtré, tout hérissé, point
affligé, mais embarrassé à ne savoir où il en était. Je lui fis le
compliment le plus fort, le plus net, le plus clair, le plus énergique,
et à haute voix. Il me prit apparemment pour quelque tiercelet des ducs
de Guiche et de Noailles, et ne me fit pas l'honneur de me répondre un
mot. J'attendis quelques moments, et voyant qu'il ne sortait rien de ce
simulacre, je fis la révérence et me retirai sans qu'il fit un seul pas
pour me conduire, comme il le devait faire tout du long de son
appartement, et se rembucha dans son cabinet. Il est vrai qu'en me
retirant, je jetai les yeux sur la compagnie, à droite et à gauche, qui
me parut fort surprise. Je m'en allai chez moi, fort ennuyé de courir le
château.

Comme je sortais de table, un valet de chambre de M. le Duc me vint dire
qu'il m'attendait, et me conduisit par la petite porte droit dans son
cabinet. Il me reçut à la porte, la ferma, me tira un fauteuil et en
prit un autre. Je l'instruisis de l'affaire dont je lui avais parlé le
matin, et après l'avoir discutée, nous nous mimes sur celle du jour. Il
me dit qu'au sortir du lever du roi, il avait été chez M. le duc de
Chartres, auquel, après les compliments de condoléance, il avait offert
tout ce qui pourrait dépendre de lui pour mériter son amitié, et lui
témoigner son véritable attachement pour la mémoire de M. le duc
d'Orléans\,: qu'à cela, M. {[}le duc{]} de Chartres étant demeuré muet,
il avait redoublé de protestations et de désirs de lui complaire en
toutes choses\,; qu'à la fin il était venu un monosyllabe sec de
remerciement, et un air d'éconduite qui avait fait prendre à M. le Duc
le parti de s'en aller. Je lui rendis ce qui m'était arrivé ce même
matin avec le même prince, duquel nous nous fîmes nos complaintes l'un à
l'autre. M. le Duc me fit beaucoup d'amitiés et de politesses, et me
demanda, en m'en conviant, si je ne viendrais pas le voir un peu
souvent. Je lui répondis qu'accablé d'affaires et de monde comme il
allait être, je me ferais un scrupule de l'importuner, et ceux qui
auraient affaire à lui\,; que je me contenterais de m'y présenter quand
j'aurais quelque chose à lui dire, et que, comme je n'étais pas
accoutumé aux antichambres, je le suppliais d'ordonner à ses gens de
l'avertir quand je paraîtrais chez lui, et lui de me faire entrer dans
son cabinet au premier moment qu'il le pourrait, où je tâcherais de
n'être ni long ni importun. Force amitiés, compliments,
convis\footnote{Invitations.} , etc.\,; tout cela dura près de trois
quarts d'heure\,; et je m'enfuis à Meudon.

M\textsuperscript{me} de Saint-Simon alla le lendemain à Versailles
faire sa cour au roi sur cet événement, et voir M\textsuperscript{me} la
duchesse d'Orléans et Monsieur son fils. M. de Fréjus alla chez
M\textsuperscript{me} de Saint-Simon dès qu'il la sut à Versailles, où
elle ne coucha point. À travers toutes les belles choses qu'il lui dit
de moi et sur moi, elle crut comprendre qu'il me saurait plus volontiers
à Paris qu'à Versailles. La Vrillière qui la vint voir aussi, et qui
avait plus de peur de moi encore que le Fréjus, se cacha moins par moins
d'esprit et de tour, et scandalisa davantage M\textsuperscript{me} de
Saint-Simon par son ingratitude après tout ce que j'avais fait pour lui.
Ce petit compagnon comptait avoir tonnelé M. le Duc par sa diligence à
l'avertir et à le servir, et brusquer son duché tout de suite. Lorsqu'il
m'en avait parlé du temps de M. le duc d'Orléans, la généralité de mes
réponses ne l'avait pas mis à son aise à mon égard. Il voulait jeter de
la poudre aux yeux et tromper M. le Duc par de faux exemples, dont il
craignait l'éclaircissement de ma part. Il ne m'en fallait pas tant pour
me confirmer dans le parti que de longue main j'avais résolu de prendre
sur l'inspection de l'état menaçant de M. le duc d'Orléans. Je m'en
allai à Paris, bien résolu de ne paraître devant les nouveaux maîtres du
royaume que dans les rares nécessités ou de bienséances indispensables,
et pour des moments, avec la dignité d'un homme de ma sorte, et de celle
de tout ce que j'avais personnellement été. Heureusement pour moi je
n'avais, dans aucun temps, perdu de vue le changement total de ma
situation, et pour dire la vérité, la perte de Mgr le duc de Bourgogne,
et tout ce que je voyais dans le gouvernement m'avait émoussé sur toute
autre de même nature. Je m'étais vu enlever ce cher prince au même âge
que mon père avait perdu Louis XIII, c'est-à-dire, mon père à trente-six
ans, son roi de quarante et un\,; moi, à trente-sept, un prince qui
n'avait pas encore trente ans, prêt à monter sur le trône, et à ramener
dans le monde la justice, l'ordre, la vérité\,; et depuis, un maître du
royaume constitué à vivre un siècle, tel que nous étions lui et moi l'un
à l'autre, et qui n'avait pas six mois plus que moi. Tout m'avait
préparé à me survivre à moi-même, et j'avais tâché d'en profiter.

Monseigneur était mort à quarante-neuf ans et demi, et M. le duc
d'Orléans vécut deux mois moins. Je compare cette durée de vie si égale,
à cause de la situation où on a vu ces deux princes à l'égard l'un de
l'autre, jusqu'à la mort de Monseigneur. Tel est ce monde et son néant.

La mort de M. le duc d'Orléans fit un grand bruit au dedans et au
dehors\,; mais les pays étrangers lui rendirent incomparablement plus de
justice et le regrettèrent beaucoup plus que les Français. Quoique les
étrangers connussent sa faiblesse, et que les Anglais en eussent
étrangement abusé, ils n'en étaient pas moins persuadés, par leur
expérience, de l'étendue et de la justesse de son esprit, de la grandeur
de son génie et de ses vues, de sa singulière pénétration, de la sagesse
et de l'adresse de sa politique, de la fertilité de ses expédients et de
ses ressources, de la dextérité de sa conduite dans tous les changements
de circonstances et d'événements, de sa netteté à considérer les objets
et à combiner toutes choses, de sa supériorité sur ses ministres et sur
ceux que les diverses puissances lui envoyaient, du discernement exquis
à démêler, à tourner les affaires, de sa savante aisance à répondre
sur-le-champ à tout, quand il le voulait. Tarit de grandes et rares
parties pour le gouvernement le leur faisaient redouter et ménager, et
le gracieux qu'il mettait à tout, et qui savait charmer jusqu'aux refus,
le leur rendait encore aimable. Ils estimaient de plus sa grande et
naïve valeur. La courte lacune de l'enchantement par lequel ce
malheureux Dubois avait comme anéanti ce prince, n'avait fait que le
relever à leurs yeux par la comparaison de sa conduite, quand elle était
sienne, d'avec sa conduite quand elle n'en portait que le nom et qu'elle
n'était que celle de son ministre. Ils avaient vu, ce ministre mort, le
prince reprendre le timon des affaires avec les mêmes talents qu'ils
avaient admirés en lui auparavant\,; et cette faiblesse, qui était son
grand défaut, se laissait beaucoup moins sentir au dehors qu'au dedans.

Le roi, touché de son inaltérable respect, de ses attentions à lui
plaire, de sa manière de lui parler, et de celle de son travail avec
lui, le pleura et fut véritablement touché de sa perte, en sorte qu'il
n'en a jamais parlé depuis, et cela est revenu souvent, qu'avec estime,
affection et regret, tant la vérité perce d'elle-même malgré tout l'art
et toute l'assiduité des mensonges et de la plus atroce calomnie, dont
j'aurai occasion de parler dans les additions que je me propose de faire
à ces Mémoires, si Dieu m'en permet le loisir. M. le Duc, qui montait si
haut par cette perte, eut sur elle une contenance honnête et bienséante.
M\textsuperscript{me} la Duchesse se contint fort convenablement\,; les
bâtards, qui ne gagnaient pas au change, ne purent se réjouir. Fréjus se
tint à quatre. On le voyait suer sous cette gêne, sa joie, ses
espérances muettes lui échapper à tous propos, toute sa contenance
étinceler malgré lui.

La cour fut peu partagée, parce que le sens y est corrompu par les
passions. Il s'y trouva des gens à yeux sains, qui le voyaient comme
faisaient les étrangers, et qui continuellement témoins de l'agrément de
son esprit, de la facilité de son accès, de cette patience et de cette
douceur à écouter qui ne s'altérait jamais, de cette bonté dont il
savait se parer d'une façon si naturelle, quoique quelquefois ce n'en
fût que le masque, de ses traits plaisants à écarter et à éconduire sans
jamais blesser, sentirent tout le poids de sa perte. D'autres, en plus
grand nombre, en furent fâchés aussi, mais bien moins par regret que par
la connaissance du caractère du successeur et de celui encore de ses
entours. Mais le gros de la cour ne le regretta point du tout\,: les uns
de cabales opposées, les autres indignés de l'indécence de sa vie et du
jeu qu'il s'était fait de promettre sans tenir, force mécontents,
quoique presque tous bien mal à propos, une foule d'ingrats dont le
monde est plein, et qui dans les cours font de bien loin le plus grand
nombre, ceux qui se croyaient en passe d'espérer plus du successeur pour
leur fortune et leurs vues, enfin un monde d'amateurs stupides de
nouveautés.

Dans l'Église, les béats et même les dévots se réjouirent de la
délivrance du scandale de sa vie, et de la force que son exemple donnait
aux libertins, et les jansénistes et les constitutionnaires, d'ambition
ou de sottise, s'accordèrent à s'en trouver tous consolés. Les premiers,
séduits par des commencements pleins d'espérance, en avaient depuis
éprouvé pis que du feu roi\,; les autres, pleins de rage qu'il ne leur
eût pas tout permis, parce qu'ils voulaient tout exterminer, et anéantir
une bonne fois et solidement les maximes et les libertés de l'Église
gallicane, surtout les appels comme d'abus\footnote{L'appel comme d'abus
  était, dans l'ancienne monarchie, l'appel devant un tribunal laïque
  contre un jugement ecclésiastique, qu'on prétendait avoir été mal et
  abusivement rendu.}, établir la domination des évêques sans bornes, et
revenir à leur ancien état de rendre la puissance épiscopale redoutable
à tous, jusques aux rois, exultaient de se voir délivrés d'un génie
supérieur, qui se contentait de leur sacrifier les personnes, mais qui
les arrêtait trop ferme sur le grand but qu'ils se proposaient, vers
lequel tous leurs artifices n'avaient cessé de tendre, et ils espéraient
tout d'un successeur qui ne les apercevrait pas, qu'ils étourdiraient
aisément, et avec qui ils seraient plus librement hardis.

Le parlement, et comme lui tous les autres parlements, et toute la
magistrature, qui, par être toujours assemblée, est si aisément animée
du même esprit, n'avait pu pardonner à M. le duc d'Orléans les coups
d'autorité auxquels le parlement lui-même l'avait enfin forcé plus d'une
fois d'avoir recours, par les démarches les plus hardies, que ses longs
délais et sa trop lente patience avait laissé porter à le dépouiller de
toute autorité pour s'en revêtir lui-même. Quoique d'adresse, puis de
hardiesse, le parlement se fût soustrait à la plupart de l'effet de ces
coups d'autorité, il n'était plus en état de suivre sa pointe, et par ce
qui restait nécessairement des bornes que le régent y avait mises, ce
but si cher du parlement lui était échappé. Sa joie obscure et
ténébreuse ne se contraignit pas d'être délivré d'un gouvernement
duquel, après avoir arraché tant de choses, il ne se consolait point de
n'avoir pas tout emporté, de n'avoir pu changer son état de simple cour
de justice en celui de parlement d'Angleterre, mais en tenant la chambre
haute sous le joug.

Le militaire, étouffé sans choix par des commissions de tous grades et
par la prodigalité des croix de Saint-Louis, jetées à toutes mains, et
trop souvent achetées des bureaux et des femmes, ainsi que les
avancements en grades, était outré de l'économie extrême qui le
réduisait à la dernière misère, et de l'exacte sévérité d'une pédanterie
qui le tenait en un véritable esclavage. L'augmentation de la solde
n'avait pas fait la moindre impression sur le soldat ni sur le cavalier,
par l'extrême cherté des choses les plus communes et les plus
indispensables à la vie, de manière que cette partie de l'État, si
importante, si répandue, si nombreuse, plus que jamais tourmentée et
réduite sous la servitude des bureaux et de tant d'autres gens ou
méprisables ou peu estimables, ne put que se trouver soulagée par
l'espérance du changement qui pourrait alléger son joug et donner plus
de lien à l'ordre du service et plus d'égards au mérite et aux services.
Le corps de la marine, tombé comme en désuétude et dans l'oubli, ne
pouvait qu'être outré de cet anéantissement et se réjouir de tout
changement, quel qu'il pût être\,; et tout ce qui s'appelait gens de
commerce, arrêtés tout court partout pour complaire aux Anglais, et
gênés en tout par la compagnie des Indes, ne pouvaient être en de
meilleures dispositions.

Enfin, le gros de Paris et des provinces, désespéré des cruelles
opérations des finances et d'un perpétuel jeu de gobelets pour tirer
tout l'argent, qui mettait d'ailleurs toutes les fortunes en l'air et la
confusion dans toutes les familles, outré de plus de la prodigieuse
cherté où ces opérations avaient fait monter toutes choses, sans
exception de pas une, tant de luxe que de première nécessité pour la
vie, gémissait depuis longtemps après une délivrance et un soulagement
qu'il se figurait aussi vainement que certainement par l'excès du besoin
et l'excès du désir. Enfin, il n'est personne qui n'aime à pouvoir
compter sur quelque chose, qui ne soit désolé des tours d'adresse et de
passe-passe, et de tomber sans cesse, malgré toute prévoyance, dans des
torquets\footnote{Ce mot du style familier est synonyme de \emph{ruses,
  tromperies}.} et dans d'inévitables panneaux\,; de voir fondre son
patrimoine ou sa fortune entre ses mains, sans trouver de protection
dans son droit ni dans les lois, et de ne savoir plus comment vivre et
soutenir sa famille.

Une situation si forcée et si générale, nécessairement émanée de tant de
faces contradictoires successivement données aux finances, dans la
fausse idée de réparer la ruine et le chaos où elles s'étaient trouvées
à la mort de Louis XIV, ne pouvait faire regretter au public celui qu'il
en regardait comme l'auteur, comme ces enfants qui se prennent en
pleurant au morceau de bois qu'un imprudent leur a fait tomber en
passant sur le pied, qui jettent, de colère, ce bois de toute leur
force, comme la cause du mal qu'ils sentent, et qui ne font pas la
moindre attention à ce passant qui en est la seule et véritable cause.
C'est ce que j'avais bien prévu qui arriverait sur l'arrangement, ou
plutôt le dérangement de plus en plus des finances, et que je voulais
ôter de dessus le compte de M. le duc d'Orléans par les états généraux
que je lui avais proposés, qu'il avait agréés, et dont le duc de
Noailles rompit l'exécution à la mort du roi, pour son intérêt
personnel, comme on l'a vu en son lieu dans ces Mémoires, à la mort du
roi. La suite des années a peu à peu fait tomber les écailles de tant
d'yeux, et a fait regretter M. le duc d'Orléans à tous avec les plus
cuisants regrets, et {[}ils{]} lui ont à la fin rendu la justice qui lui
avait toujours été due.

Le lendemain de la mort de M. le duc d'Orléans, son corps fut porté de
Versailles à Saint-Cloud, et le lendemain qu'il y fut les cérémonies y
commencèrent. M. le comte de Charolais avec le duc de Gesvres et le
marquis de Beauvau, qui devaient porter la queue de son manteau,
allèrent, dans un carrosse du roi, entouré de ses gardes, à Saint-Cloud.
M. le comte de Charolais donna l'eau bénite, représentant le roi, et fut
reçu à la descente du carrosse et reconduit de même par M. le duc de
Chartres, qui s'était fait accompagner par les deux fils du duc du
Maine. Le cour fut porté de Saint-Cloud au Val-de-Grâce par l'archevêque
de Rouen, premier aumônier du prince défunt, à la gauche duquel était M.
le comte de Clermont, prince du sang, et le duc de Montmorency, fils du
duc de Luxembourg, sur le devant, avec tous les accompagnements
ordinaires. M. le prince de Conti accompagna le convoi avec le duc de
Retz, fils du duc de Villeroy, qui se fit de Saint-Cloud à Saint-Denis,
passant dans Paris avec la plus grande pompe. Le chevalier de Biron, à
qui son père avait donné sa charge de premier écuyer de M. le duc
d'Orléans, lorsqu'il fut fait duc et pair, y était à cheval, ainsi que
le comte d'Étampes, capitaine des gardes\,; tous les autres officiers
principaux de la maison dans des carrosses. Les obsèques furent
différées jusqu'au 12 février. M. le duc de Chartres, devenu duc
d'Orléans, M. le comte de Clermont et M. le prince de Conti firent le
grand deuil\,; l'archevêque de Rouen officia en présence des cours
supérieures, et Poncet, évêque d'Angers, fit l'oraison funèbre qui ne
répondit pas à la grandeur du sujet. Le roi visita à Versailles
M\textsuperscript{me} la duchesse d'Orléans, M\textsuperscript{me} la
Duchesse, et fit le même honneur à M. le duc de Chartres.

C'est le seul prince du sang qu'il ait visité. Il alla voir aussi
M\textsuperscript{me} la princesse de Conti, M\textsuperscript{lle} de
Chartres et M\textsuperscript{me} du Maine.

Deux jours après la mort de M. le duc d'Orléans, le maréchal de Villars
entra dans le conseil d'État, et eut le gouvernement des forts et
citadelle de Marseille qu'avait le feu premier écuyer.

Il me fait souvenir que j'ai dit plus haut que j'aurais à dire encore
quelque chose sur cette charge. Nonobstant l'arrêt du conseil de
régence, dont il a été parlé ici en son temps, qui l'avait
contradictoirement et nettement confirmé dans toutes les fonctions de sa
charge, et dans l'indépendance entière de celle de grand écuyer, ce
dernier n'avait cessé de le tracasser tant qu'il avait pu. Son fils, à
sa mort, ayant succédé à sa charge, voulut se délivrer de cette
continuelle importunité\,: le père était des amis de l'évêque de Fréjus
qui se piqua de le servir dans une affaire si juste. Beringhen présenta
un mémoire au roi, et un autre à M. le duc d'Orléans. Il fut communiqué
au grand écuyer qui y répondit et qui fut de nouveau tondu en plein par
un arrêt du conseil d'en haut\footnote{On ne doit pas confondre le
  conseil d'en haut avec le conseil d'État, ni avec le \emph{conseil des
  parties}, qui n'était qu'une section du conseil d'État (voy. t. II,
  p.~445). On pourrait le comparer plus exactement au conseil des
  ministres. Il se composait, en effet, du souverain, des princes du
  sang que le roi y appelait, du chancelier, du surintendant ou du
  contrôleur général des finances, des secrétaires d'État et de quelques
  seigneurs désignés par le roi. Les ducs de Beauvilliers et de
  Chevreuse furent membres du conseil d'en haut, pendant la dernière
  partie du règne de Louis XIV, quoiqu'ils ne fussent pas secrétaires
  d'État.}, en présence du roi et de M. le duc d'Orléans. Le prince
Charles de Lorraine, grand écuyer, en fut si piqué que Beringhen lui
ayant envoyé, comme de coutume, les comptes de la petite écurie à signer
sur son arrêté, il dit qu'il ne savait point signer ce qu'il ne voyait
point. On fit ce qu'on put pour lui faire entendre raison l'opiniâtreté
fut invincible\,; enfin il fallait bien que ces comptes fussent signés,
j'expliquerai cela tout à l'heure. Au bout de cinq ou six mois de délai,
M. le Duc lui déclara que s'il persistait dans son refus, lui les
signerait comme grand maître de la maison du roi, et en effet les signa.
Ainsi, le grand écuyer perdit, par humeur, une des plus belles
prérogatives de sa charge, ou se mit du moins en grand hasard de ne la
recouvrer jamais. Voici donc en quoi consistait la prétendue délicatesse
du grand écuyer, inconnue jusqu'alors à tout autre et à lui-même, et la
mécanique de ces signatures. Le grand maître de la maison du roi, celui
de l'artillerie, le grand écuyer et les premiers gentilshommes de la
chambre, chacun dans son année, sont ordonnateurs des dépenses qui se
font sous leurs charges, c'est-à-dire que sur leur signature qu'ils
mettent aux arrêtés des comptes de ces dépenses, ils passent sans autre
examen à la chambre des comptes, et les dépenses y sont allouées. Le
grand maître de la garde-robe, le premier écuyer et le premier maître
d'hôtel, pour la bouche du roi seulement, qui, du temps des Guise, fut
rendue indépendante du grand maître de la maison du roi, dont ils
possédaient la charge, ces trois officiers règlent et arrêtent les
comptes des dépenses qui se font sous leur charge, et les signent\,;
mais comme la chambre des comptes ne reconnaît point leur signature,
parce qu'ils ne sont pas ordonnateurs, il est d'usage que le grand
maître de la garde-robe envoie les comptes de la garde-robe au premier
gentilhomme de la chambre en année, qui est obligé de les signer sans
examen aucun, et sans les voir, à la seule inspection de la signature du
grand maître de la garde-robe, et il en est de même des comptes de la
bouche entre le premier maître d'hôtel du roi et le grand maître de sa
maison, et entre le grand et le premier écuyer pour les comptes de la
petite écurie.

Beringhen, premier écuyer, qui venait d'achever de faire confirmer
l'indépendance de sa charge, ne survécut pas de sept mois son père à qui
il y avait succédé. Il mourut le 1er décembre à quarante-trois ans,
homme obscur au dernier point, timide, solitaire, embarrassé du monde,
avec de l'esprit et de la lecture. Il ne laissa qu'une fille de la fille
du feu marquis de Lavardin, ambassadeur à Rome autrefois. Il n'avait
qu'un frère fort mal alors avec M. le duc d'Orléans, qui l'avait même
éloigné assez longtemps de Paris, à qui il avait été assez fou pour
disputer avantageusement une maîtresse, de sorte qu'il était entièrement
hors d'espérance de la charge de son frère\,; la mort si prompte de ce
prince la lui rendit. L'évêque de Fréjus lui fit donner la charge, et M.
le Duc, qui, par je ne sais quelle intrigue, y aurait voulu Nangis, lui
donna prématurément la charge de chevalier d'honneur de la future reine,
et au maréchal de Tessé, qui s'ennuyait beaucoup dans sa prétendue
retraite, la charge de premier écuyer de la future reine, qu'il avait
eue de la dernière Dauphine, lors de son mariage qu'il avait traité, et
en même temps la survivance pour son fils, en envoyant le père en
ambassade en Espagne.

La maréchale d'Humières, fille de M. de La Châtre qui a laissé des
mémoires\footnote{Les mémoires de La Châtre font partie de toutes les
  collections de mémoires relatifs à l'histoire de France. Ils
  concernent spécialement la minorité de Louis XIV et surtout la faction
  des Importants, qui menaça Mazarin en 1643. La Châtre faisait partie
  de cette cabale.}, mourut le même jour que M. le duc d'Orléans. Elle
avait été dame du palais de la reine, et, à près de quatre-vingt-huit
ans qu'elle avait, ayant pendant cette longue vie joui toujours d'une
santé parfaite de corps et d'esprit, on voyait encore qu'elle avait été
fort belle. Elle mourut uniquement de vieillesse, s'étant couchée la
veille en parfaite santé, allant et venant et sortant à son ordinaire.
Elle se retira, peu après la mort du maréchal d'Humières, dans le dehors
du couvent des Carmélites de la rue Saint-Jacques. C'est la première
duchesse qui, par une dévotion mal entendue dans sa retraite, quitta la
housse\footnote{La housse était une draperie dont certaines personnes,
  et, entre autres, les ducs et duchesses avaient droit d'orner leurs
  carrosses.}, et, comme les sottises sont plus volontiers imitées en
France qu'ailleurs, celle-là l'a été depuis par plusieurs autres, qui, a
son exemple, ont en même temps conservé leurs armes à leurs carrosses
avec les marques de leur dignité.

Le lendemain de la mort de M. le duc d'Orléans, le comte de Toulouse
déclara son mariage avec la soeur du duc de Noailles, veuve avec deux
fils du marquis de Gondrin, fils aîné du duc d'Antin. Elle avait été
dame du palais de la dernière Dauphine. Le monde, qui abonde en sots et
en jaloux, ne lui vit pas prendre le rang de son nouvel état sans envie
et sans murmure. Je n'ai pas lieu, comme on a vu ici plus d'une fois,
d'aimer le duc de Noailles, et que je ne m'en suis jamais contraint à
son égard\,; mais la vérité veut que je dise que, de la naissance que
sont les Noailles il n'y aurait pas à se récrier quand une Noailles
aurait épousé un prince du sang. Au moins ne niera-t-on pas l'extrême
différence d'une Noailles à une Séguier que nous avons vue duchesse de
Verneuil au mariage de Mgr le duc de Bourgogne, conviée à la noce par le
roi, y dîner à sa table au festin de la noce, et en possession de tout
ce dont a joui la comtesse de Toulouse. Le bas emploi de capitaine des
gardes du cardinal Mazarin, d'où le père du premier maréchal-duc de
Noailles passa si étrangement à la charge de premier capitaine des
gardes du corps, ce qui le fit duc et pair dans la suite, a trompé bien
des gens qui ignorent que ce même Noailles, capitaine des gardes du
cardinal Mazarin, était fils de la fille du vieux maréchal de
Roquelaure, et que la soeur de son père avait épousé le fils et frère
des deux maréchaux de Biron, duquel mariage vient le maréchal duc de
Biron d'aujourd'hui\,; qu'en remontant jusqu'au delà de 1250, on leur
trouve les meilleures alliances de leur province et des voisines, et que
la terre et le château de Noailles dont ils tirent leur nom, ils les
possèdent de temps immémorial.

Un fou succéda à un scélérat dans la place de premier président du
parlement de Paris, par la faveur de M. le Duc, qui aimait fort les
Gesvres, et qui crut se bien mettre avec le parlement en choisissant
Novion, le plus ancien des présidents à mortier, mais le plus
contradictoire à la remplir. Il n'était ni injuste ni malhonnête homme,
comme l'autre premier président de Novion, son grand-père, mais il ne
savait rien de son métier que la basse procédure, en laquelle, à la
vérité, il excellait comme le plus habile procureur. Mais par delà cette
ténébreuse science, il ne fallait rien attendre de lui. C'était un homme
obscur, solitaire, sauvage, plein d'humeurs et de caprices jusqu'à
l'extravagance\,; incompatible avec qui que ce fût, désespéré lorsqu'il
lui fallait voir quelqu'un, le fléau de sa famille et de quiconque avait
affaire à lui, enfin insupportable aux autres, et, de son aveu, très
souvent à lui-même. Il se montra tel dans une place où il avait affaire
avec la cour, avec sa compagnie, avec le public, contre lequel il se
barricadait, en sorte qu'on n'en pouvait approcher\,; et tandis qu'il
s'enfermait de la sorte, et que les plaideurs en gémissaient, souvent
encore de ses brusqueries et de ses \emph{sproposito} quand ils
pouvaient pénétrer jusqu'à lui, il s'en allait prendre l'air, disait-il,
dans la maison qu'il occupait avant d'être premier président, et causer
avec un charron, son voisin, sur le pas de sa boutique, qui était,
disait-il, l'homme du meilleur sens du monde.

Un pauvre plaideur, d'assez bas aloi, se désespérant un jour de {[}ne
le{]} pouvoir aborder pour lui demander une audience, tournait de tous
côtés dans sa maison du palais, ne sachant à qui adresser ni où donner
de la tête. Il entra dans la basse-cour\footnote{On entendait alors par
  basse-cour la partie d'un hôtel, qui était réservée aux remises,
  écuries, etc.} et vit un homme en veste, qui regardait panser les
chevaux, qui lui demanda brusquement ce qu'il venait faire là et ce
qu'il demandait. Le pauvre plaideur lui répondit bien humblement qu'il
avait un procès qui le désolait, qu'il avait grand intérêt de faire
juger, mais que, quelque peine qu'il prit, et quelque souvent qu'il se
présentât, il ne pouvait approcher de M. le premier président, qui était
d'une humeur si farouche et si fantasque, qu'il ne voulait voir
personne, et ne se laissait point aborder. Cet homme en veste lui
demanda s'il avait un placet pour sa cause, et de le lui donner, et
qu'il verrait s'il le pourrait faire arriver jusqu'au premier président.
Le pauvre plaideur lui tira son placet de sa poche, et le remercia bien
de sa charité, mais en lui marquant son doute qu'il pût venir à bout de
lui procurer audience d'un homme aussi étrange et aussi capricieux que
ce premier président, et se retira. Quatre jours après il fut averti par
son procureur que sa cause serait appelée à deux jours de là, dont il
fut bien agréablement surpris. Il alla donc à l'audience de la
grand'chambre avec son avocat, prêt à plaider. Mais quel fut son
étonnement quand il reconnut son homme en veste assis en place et en
robe de premier président\,! Il en pensa tomber à la renverse, et de
frayeur de ce qu'il lui avait {[}dit{]} de lui-même, pensant parler à
quelque quidam. La fin de l'aventure fut qu'il gagna son procès tout de
suite. Tel était Novion.

Il avait épousé une Berthelot, tante de M\textsuperscript{me} de Prie,
qui avait bien eu autant de part que MM. de Gesvres à le faire premier
président. Il sentait toute sa répugnance à se montrer dans les
fonctions de cette charge\,; mais, étant le doyen des présidents à
mortier, il ne put souffrir qu'un autre que lui y montât. Lorsque M. le
Duc déclara, à la Chandeleur 1724, la grande promotion de l'ordre à
faire à la Pentecôte suivante, Dodun, contrôleur général, et Maurepas,
secrétaire d'État, qui tous deux avaient grande envie de porter l'ordre,
renouvelèrent la difficulté qu'on avait faite à l'occasion de la
promotion du lendemain du sacre à Crozat et à Montargis, de leur y
laisser exercer leurs charges de grand trésorier et de greffier de
l'ordre\,; mais M. le duc d'Orléans, qui leur avait permis de les
acheter, passa par-dessus, et leur y fit faire leurs fonctions. M. le
Duc fut plus accessible aux désirs de deux hommes dont il s'accommodait.
Crozat et Montargis eurent ordre de vendre, le premier à Dodun, l'autre
à Maurepas, et ce ne fut pas sans de grands combats que les deux
vendeurs obtinrent la permission ordinaire de continuer à porter
l'ordre. En même temps M. le Duc donna le râpé de grand trésorier à
d'Armenonville, garde des sceaux, et celui de greffier au premier
président de Novion, qui, tout aise qu'il fût de porter l'ordre, se
trouva fort mécontent de payer le serment et d'avoir des croix et des
rubans bleus à acheter, et le marqua avec beaucoup d'indécence.

Enfin, ne pouvant plus tenir à exercer ses fonctions de premier
président, encore moins le public, qui avait affaire à lui sans cesse,
il s'en démit en septembre 1724, après l'avoir seulement gardée un an,
et s'en retourna ravi, et le public aussi d'en être délivré, à sa vie
chérie de ne plus voir personne, n'ayant plus aucune charge, enfermé
seul dans sa maison, et causant à son plaisir avec son voisin le
charron, sur le pas de la porte de sa boutique, et mourut en sa terre de
Grignon, en septembre 1731, à soixante-onze ans, regretté de personne.

Il avait perdu son fils unique dès 1720, qui avait laissé un fils. M. le
Duc fit la grâce entière, et donna à cet enfant de quinze ans, la charge
de président à mortier de son grand-père, en faisant celui-ci premier
président, et la donna à exercer à Lamoignon de Blancménil, lors avocat
général, jusqu'à ce que ce petit Novion fût en âge de la faire\,: abus
fort étrange de ces custodi-nos\footnote{Le \emph{custodi-nos} était
  celui qui gardait un bénéfice ecclésiastique pour le rendre à un autre
  au bout d'un certain temps.} de charges de président à mortier, qui
s'est introduit dans le parlement, pour les conserver dans les familles.

\hypertarget{conclusion}{%
\chapter{CONCLUSION}\label{conclusion}}

1723

Conclusion\,; vérité\,; désapprobation\,; impartialité.

Me voici enfin parvenu au terme jusqu'auquel je m'étais proposé de
conduire ces Mémoires. Il n'y en peut avoir de bons que de parfaitement
vrais, ni de vrais qu'écrits par qui a vu et manié lui-même les choses
qu'il écrit, ou qui les tient de gens dignes de la plus grande foi, qui
les ont vues et maniées\,; et de plus, il faut que celui qui écrit aime
la vérité jusqu'à lui sacrifier toutes choses. De ce dernier point,
j'ose m'en rendre témoignage à moi-même, et me persuader qu'aucun de
tout ce qui m'a connu n'en disconviendrait. C'est même cet amour de la
vérité qui a le plus nui à ma fortune\,; je l'ai senti souvent, mais
j'ai préféré la vérité à tout, et je n'ai pu me ployer à aucun
déguisement\,; je puis dire encore que je l'ai chérie jusque contre
moi-même. On s'apercevra aisément des duperies où je suis tombé, et
quelquefois grossières, séduit par l'amitié ou par le bien de l'État,
que j'ai sans cesse préféré à toute autre considération, sans réserve,
et toujours à tout intérêt personnel, comme encore {[}en{]} bien
d'autres occasions que j'ai négligé d'écrire, parce qu'elles ne
regardaient que moi, sans connexion d'éclaircissements ou de curiosité
sur les affaires ou le cours du monde. On peut voir que je persévérai à
faire donner les finances au duc de Noailles, parce que je l'en crus,
bien mal à propos, le plus capable, et le plus riche et le plus revêtu
d'entre les seigneurs à qui on les pût donner, dans les premiers jours
même de l'éclat de la profonde scélératesse qu'il venait de commettre à
mon égard. On le voit encore dans tout ce que je fis pour sauver le duc
du Maine contre mes deux plus chers et plus vifs intérêts, parce que je
croyais dangereux d'attaquer lui et le parlement à la fois, et que le
parlement était lors l'affaire la plus pressée, qui ne se pouvait
différer. Je me contente de ces deux faits, sans m'arrêter à bien
d'autres qui se trouvent répandus dans ces Mémoires, à mesure qu'ils
sont arrivés, lorsqu'ils ont trait à la curiosité du cours des affaires
ou des choses de la cour et du monde.

Reste à toucher l'impartialité, ce point si essentiel et tenu pour si
difficile, je ne crains point de le dire, impossible à qui écrit ce
qu'il a vu et manié. On est charmé des gens droits et vrais\,; on est
irrité contre les fripons dont les cours fourmillent\,; on l'est encore
plus contre ceux dont on a reçu du mal. Le stoïque est une belle et
noble chimère. Je ne me pique donc pas d'impartialité, je le ferais
vainement. On trouvera trop, dans ces Mémoires, que la louange et le
blâme coulent de source à l'égard de ceux dont je suis affecté, et que
l'un et l'autre est plus froid sur ceux qui me sont plus indifférents\,;
mais néanmoins vif toujours pour la vertu, et contre les malhonnêtes
gens, selon leur degré de vices ou de vertu. Toutefois, je me rendrai
encore ce témoignage, et je me flatte que le tissu de ces Mémoires ne me
le rendra pas moins, que j'ai été infiniment en garde contre mes
affections et mes aversions, et encore plus contre celles-ci, pour ne
parler des uns et des autres que la balance à la main, non seulement ne
rien outrer, mais ne rien grossir, m'oublier, me défier de moi comme
d'un ennemi, rendre une exacte justice, et faire surnager à tout la
vérité la plus pure. C'est en cette manière que je puis assurer que j'ai
été entièrement impartial, et je crois qu'il n'y a point d'autre manière
de l'être.

Pour ce qui est de l'exactitude et de la vérité de ce que je raconte, on
voit par les Mémoires mêmes que presque tout est puisé de ce qui a passé
par mes mains, et le reste, de ce que j'ai su par ceux qui avaient
traité les choses que je rapporte. Je les nomme\,; et leur nom ainsi que
ma liaison intime avec eux est hors de tout soupçon. Ce que j'ai appris
de moins sûr, je le marque\,; et ce que j'ai ignoré, je n'ai pas honte
de l'avouer. De cette façon les Mémoires sont de source, de la première
main. Leur vérité, leur authenticité ne peut être révoquée en doute\,;
et je crois pouvoir dire qu'il n'y en a point eu jusqu'ici qui aient
compris plus de différentes matières, plus approfondies, plus
détaillées, ni qui forment un groupe plus instructif ni plus curieux.

Comme je n'en verrai rien, peu m'importe. Mais si ces Mémoires voient
jamais le jour, je ne doute pas qu'ils n'excitent une prodigieuse
révolte. Chacun est attaché aux siens, à ses intérêts, à ses
prétentions, à ses chimères, et rien de tout cela ne peut souffrir la
moindre contradiction. On n'est ami de la vérité qu'autant qu'elle
favorise, et elle favorise peu de toutes ces choses-là. Ceux dont on dit
du bien n'en savent nul gré, la vérité l'exigeait. Ceux, en bien plus
grand nombre, dont on ne parle pas de même entrent d'autant plus en
furie que ce mal est prouvé par les faits\,; et comme au temps où j'ai
écrit, surtout vers la fin, tout tournait à la décadence, à la
confusion, au chaos, qui depuis n'a fait que croître, et que ces
Mémoires ne respirent qu'ordre, règle, vérité, principes certains, et
montrent à découvert tout ce qui y est contraire, qui règnent de plus en
plus avec le plus ignorant, mais le plus entier empire, la convulsion
doit donc être générale contre ce miroir de vérité. Aussi ne sont-ils
pas faits pour ces pestes des États qui les empoisonnent, et qui les
font périr par leur démence, par leur intérêt, par toutes les voies qui
en accélèrent la perte, mais pour ceux qui veulent être éclairés pour la
prévenir, mais qui malheureusement sont soigneusement écartés par les
accrédités et les puissants qui ne redoutent rien plus que la lumière,
et pour des gens qui ne sont susceptibles d'aucun intérêt que de ceux de
la justice, de la vérité, de la raison, de la règle, de la sage
politique, uniquement tendus au bien public.

Il me reste une observation à faire sur les conversations que j'ai eues
avec bien des gens, surtout avec Mgr le duc de Bourgogne, M. le duc
d'Orléans, M. de Beauvilliers, les ministres, le duc du Maine une fois,
trois ou quatre avec le feu roi, enfin avec M. le Duc et beaucoup de
gens considérables, et sur ce que j'ai opiné, et les avis que j'ai pris,
donnés ou disputés. Il y en a de tels, et en nombre, que je comprends
qu'un lecteur qui ne m'aura point connu sera tenté de mettre au rang de
ces discours factices que des historiens ont souvent prêtés du leur à
des généraux d'armées, à des ambassadeurs, à des sénateurs, à des
conjurés, pour orner leurs livres. Mais je puis protester, avec la même
vérité qui jusqu'à présent a conduit ma plume, qu'il n'y a aucun de tous
ces discours, que j'ai tenus et que je rapporte, qui ne soit exposé dans
ces Mémoires avec la plus scrupuleuse vérité, ainsi que ceux qui m'ont
été tenus\,; et que s'il y avait quelque chose que je pusse me
reprocher, {[}ce{]} serait d'avoir plutôt affaibli que fortifié les
miens dans le rapport que j'en ai fait ici, parce que la mémoire en peut
oublier des traits, et qu'animé par les objets et par les choses, on
parle plus vivement et avec plus de force qu'on ne rapporte après ce
qu'on a dit. J'ajouterai, avec la même confiance que j'ai témoignée
ci-dessus, que personne, de tout ce qui m'a connu et a vécu avec moi, ne
concevrait aucun soupçon sur la fidélité du récit que je fais de ces
conversations, pour fortes qu'elles puissent être trouvées, et qu'il n'y
en aurait aucun qui m'y reconnût trait pour trait.

Un défaut qui m'a toujours déplu, entre autres, dans les Mémoires, c'est
qu'en les finissant le lecteur perd de vue les personnages principaux
dont il y a été le plus parlé, dont la curiosité du reste de leur vie
demeure altérée. On voudrait voir tout de suite ce qu'ils sont devenus,
sans aller chercher ailleurs avec une peine que la paresse arrête aux
dépens de ce qu'on désirerait savoir. C'est ce que j'ai envie de
prévenir ici, si Dieu m'en donne le temps. Ce ne sera pas avec la même
exactitude que lorsque j'étais de tout. Quoique le cardinal Fleury ne
m'ait rien caché de ce que j'avais envie de savoir des affaires
étrangères, dont presque toujours il me parlait le premier, et aussi de
quelques affaires de la cour, tout cela était si peu suivi de ma part et
avec tant d'indifférence, et encore plus de moi avec les ministres ou
d'autres gens instruits, interrompu encore de si vastes lacunes, que
j'ai tout lieu de craindre que ce supplément ou suite de mes Mémoires ne
soit fort languissant, mal éclairé et fort différent de ce que j'ai
écrit jusqu'ici\,; mais au moins y verra-t-on ce que sont devenus les
personnages qui ont paru dans les Mémoires, qui est tout ce que je me
propose, jusqu'à la mort du cardinal Fleury \footnote{Ce paragraphe
  depuis \emph{un défaut} jusqu'à \emph{la mort du cardinal Fleury} a
  été omis dans les anciennes éditions. Saint-Simon a-t-il réellement
  écrit la suite de ces Mémoires jusqu'en 1743, époque de la mort de
  Fleury\,? On ne pourrait éclaircir ce doute que s'il était permis
  d'étudier les papiers du duc conservés au ministère des affaires
  étrangères. Nous l'avons vainement tenté\,; nous ne pouvons que
  recommander cette recherche à d'autres qui seront plus heureux que
  nous.}.

Dirai-je enfin un mot du style, de sa négligence, de répétitions trop
prochaines des mêmes mots, quelquefois de synonymes trop multipliés,
surtout de l'obscurité qui naît souvent de la longueur des phrases,
peut-être de quelques répétitions\,? J'ai senti ces défauts\,; je n'ai
pu les éviter, emporté toujours par la matière, et peu attentif à la
manière de la rendre, sinon pour la bien expliquer. Je ne fus jamais un
sujet académique, je n'ai pu me défaire d'écrire rapidement. De rendre
mon style plus correct et plus agréable en le corrigeant, ce serait
refondre tout l'ouvrage, et ce travail passerait mes forces, il courrait
risque d'être ingrat. Pour bien corriger ce qu'on a écrit il faut savoir
bien écrire\,; on verra aisément ici que je n'ai pas dû m'en piquer. Je
n'ai songé qu'à l'exactitude et à la vérité. J'ose dire que l'une et
l'autre se trouvent étroitement dans mes Mémoires, qu'ils en sont la loi
et l'âme, et que le style mérite en leur faveur une bénigne indulgence.
Il en a d'autant plus besoin, que je ne puis le promettre meilleur pour
la suite que je me propose\footnote{Cette dernière phrase a été
  supprimée par les précédents éditeurs.}.

FIN DES MÉMOIRES DE SAINT-SIMON.

\hypertarget{testament-olographe-du-duc-de-saint-simon.}{%
\chapter{TESTAMENT OLOGRAPHE DU DUC DE
SAINT-SIMON.}\label{testament-olographe-du-duc-de-saint-simon.}}

1755

Au nom du Pere\footnote{L'orthographe de ce testament est
  scrupuleusement reproduite avec toutes ses irrégularités et ses
  fautes.}, du Fils et du S. Esprit, un seul Dieu en trois Personnes.

Estant presentement dans la ville de Paris, dans la maison que je loüe
rüe Grenelle, faubourg S. Germain, Paroisse de S. Sulpice, le vingt
sixieme juin mil sept cent cinquante quatre, moy Loüis duc de S. Simon,
par la grace de Dieu sain de corps et d'esprit, après avoir serieusement
réfléchi sur l'instabilité de la vie humaine, mon age si avancé, la
servitude de la mort, l'incertitude de son heure\,: de peur d'estre
prévenu par elle, j'ay écrit de ma main et signé aussy de ma main le
présent testament olographe et la disposition de ma derniere volonté.

Premierement, comme Enfant de Dieu quoyque tres indigne, et de sa sainte
Eglise Catholique, Apostolique et Romaine dans laquelle je suis né, et
dans laquelle je veux vivre et mourir, moyennant la grace de Dieu qui
m'y a fait naistre et vivre, je me recommande en toute humilité, Foy et
Espérance mon ame a Dieu le Pere, le Fils et le S. Esprit qui est la
tres sainte et adorable Trinité, pour en obtenir tout indiqne que j'en
suis, misericorde et le salut éternel, par le prix infini de
l'Incarnation, des souffrance et du sang de Nostre Seigneur et
Redempteur Jesus Christ. Et encore je me recommande à la tres sainte
Vierge sa Mere, a S. Loüis mon patron, et a tous les Saints de la Cour
céleste, les priant d'interceder pour moy aupres de Dieu.

Secondement, je veux que mes debtes soyent payées le plus promptement
que faire se pourra.

Troisiemement, je veux que tous les legs faits par ma tres chere
éspouse, soyent acquités avec toutte l'exactitude et la promptitude
possible, singuliérement la fondation de trois sœurs de charité dans le
bourg de la Ferté Arnauld dit le Vidame, gage et maison d'icelles,
bouillons, nourriture, pauvres malades\,; et celle aussy d'un Vicaire
audit bien et Parroisse, si de mon vivant elles nestoient pas faittes.
Ce que j'ordonne d'autant plus expressément que j'en suis l'Executeur
testamentaire, que j'ay eu toujours ces fondations à cœur, que j'y ay
inutilement travaillé jusqu'a présent, et que je désire par dessus
toutes les choses de ce monde que ses volontés soyent pleinement
exécutées et accomplies, soit qu'elles soyent éxprimées ou non en ce
mien testament.

Quatriémement, lorsqu'il sera plu a Dieu me retirer de ce monde, je veux
que mon corps soit laissé au moins trente heures sans y toucher ny le
deplacer, sinon pour s'assurer qu'il n'y a plus de vie, qu'au bout de ce
temps il soit ouvert en deux endroits, scavoir au haut du nés, et a la
gorge au haut de la poitrine, pour reconnoistre a l'utilité publique,
les causes de cet enchiffrement\footnote{Pour \emph{enchifrènement}.}
qui m'a esté une vraye maladie, et de ces estouffements estranges dont
je me suis depuis toujours ressenti.

Cinquiémement, je veux que de quelque lieu que je meure, mon corps soit
aporté et inhumé dans le caveau de l'Eglise paroissiale dudit lieu de la
Ferté aupres de celuy de ma tres chere éspouse, et qui soit fait et mis
anneaux, crochets et liens de fer qui attachent nos deux cerceuils si
étroitement ensemble et si bien rivés, qu'il soit impossible de les
separer l'un de l'autre sans les briser tous deux. Je veux aussy et
ordonne tres expressement qu'il soit mis et rivé sur nos deux cercueils
une plaque de cuivre, sur chacune desquelles soyent respectivement
gravés nos noms et ages, le jour trop heureux pour moy de nostre mariage
et celuy de nostre mort\,: que sur la sienne, autant que l'espace le
pourra permettre, soyent gravées ses incomparables vertus\,: sa piété
inaltérable de toutte sa vie si vraye, si simple, si constante, si
uniforme, si solide, si admirable, si singulierement aimable qui la
rendüe les delices et l'admiration de tout ce qui l'a connüe, et sur
touttes les deux plaques, la tendresse extreme et reciproque, la
confience sans reserve, l'union intime parfaitte sans lacune, et si
pleinement reciproque dont il a plu a Dieu benir singulierement tout le
cours de nostre mariage, qui a fait de moy tant qu'il a duré, l'homme le
plus heureux, goustant sans cesse l'inestimable prix de cette Perle
unique, qui réunissant tout ce qu'il est possible d'aimable et
d'estimable avec le don du plus excellent conseil, sans jamais la plus
legere complaisance en elle mesme, ressembla si bien a la femme forte
decritte par le S. Esprit, de laquelle aussy la perte m'a rendu la vie a
charge, et le plus malheureux de tous les hommes par l'amertume et les
pointes que j'en ressents jour et nui en presque tous les moments de ma
vie. Je veux et j'ordonne tres expressement aussy, que le temoignage de
tant de si grandes et de si aimables vertus de nostre si parfaitte
union, et de l'extrême et continuelle douleur ou m'a plongé une
séparation si affreuse, soit écrit et gravé bien au long de la maniere
la plus durable sur un marbre, que pour cela je veux qu'il soit fort
long et large, appliqué pour estre vu de tout le monde dans l'Eglise
dudit la Ferté a l'endroit du mur le plus immediat au caveau de notre
sepulture avec nos armes et qualités, sans nulle magnificence ny rien
qui ne soit modeste. Je conjure tres instament l'Executeur de ce present
testament, d'avoir un soin et une attention particuliere à l'éxécution
exacte de tout le contenu de ce present article, pour laquelle je me
rapporte et legue pour la dépense ce que ledit Executeur jugera a
propos, dont je le constitue Ordonateur.

Sixiémement, je veux que le jour de l'inhumation de mon corps, il soit
fait, dit et célébré un service solemnel et des Messes basses autant
qu'il sera possible dans ladite Eglise de la Ferté pour le repos de mon
ame, avec les collectes pour le repos de celle de ma tres chere éspouse,
et qu'il soit donné le mesme jour audit lieu cinq cent francs aux
pauvres, et dit au plustost qu'il se pourra, en diverses Eglises, deux
mil Messes pour le repos de mon ame, et quinze cent francs aux pauvres.

Septiesmement, je donne et légue a la fabrique a l'Eglise paroissiale
dudit la Ferté la somme de mil livres une fois payée, laquelle sera mise
en fond acquis pour cela, qui produira cinquante livres de rente, ou mis
de mesme en rente fonciére, moyennant quoy laditte fabrique sera tenüe
de faire dire et célébrer tous les ans a perpétuité dans lad. Eglise
deux services, l'un le jour annuel de mon déceds, l'autre le vingt et un
janvier, jour du deceds de ma tres chere éspouse pour le repos de nos
ames avec les colléctes comme cy dessus, pour celuy ou celle dont ce ne
sera pas le jour du déceds. En outre douze Messes basses avec les
collectes cy dessus pour celuy ou celle dont ce ne sera pas le jour du
deceds pour le repos de nos ames, qui seront dittes en la mesme Eglise
le mesme jour de chaque service. Et de plus douze Messes basses a mesme
fin qui seront dittes en la mesme Eglise, à l'Autel la plus proche de
notre sépulture, alternativement par mois le jour de la datte de mon
déceds, et de celuy de ma tres chere espouse, avec comme dessus les
collectes pour celuy ou celle dont ce ne sera pas le jour du déceds\,:
lesquelles Messes basses et deux services seront annoncés au prosne de
laditte Paroisse le dimanche precedant imédiatement le jour desdits deux
services, et douze pour le repos de nos ames a la fin de la grand Messe
Parroissiale pour le repos de nos ames, en laquelle laditte annonce aura
esté faitte. Et la veille desdits deux services ou grandes Messes par
an, seront chantées les vespres, matines et laudes des morts pour le
repos de nos ames. Et si lesdits jours marqués pour celebrer lesdits
deux services et douze messes basses, et autres douze Messes basses une
par chacun mois se trouveroient empeschés par dimanches ou festes,
seront lesdits services et Messes basses avancées au jour le plus comode
et le plus prochain du jour naturel empesché.

Huitiesmement, je défends tres éxpressément touttes tentures, armoiries
et ceremonies quelconques, tant dans le lieu ou je mourray, qu'au
transport de mon corps, en toutte Eglise et en l'Eglise dudit la Ferté,
et partout ailleurs, ainsy que touttes littres\footnote{La \emph{litre}
  est une grande bande noire qu'on tend autour de l'église et sur
  laquelle sont appliqués les écussons des armes du défunt.} aux Eglises
de mes seigneuries.

Neuviémement, je prie Me la Mareschale de Montmorency de vouloir bien
recevoir comme une marque de ma vraye amitié la croix de bois bordée de
metail avec laquelle le saint abbé Réformateur de la Trappe a esté beni,
que depuis sa mort j'ay toujours portée, les choses qui luy ont servi
qui me restent de luy, quelques reliques que j'ay toujours portées, un
portrait de poche de ma tres chere espouse qui n'est jamais sorti de la
mienne depuis nostre mariage quoyque beaucoup moins bien qu'elle nestoit
alors, et ses tablettes que j'ay toujours portées depuis que j'ay eu
l'affreux malheur de la perdre.

Dixiémement, je laisse a ma fille, la Pssede Chimay, la bague d'un rubis
ou est gravé le portrait de Louis treize, que je porte a mon doigt
depuis plus de cinquante ans, un autre bague de composition ou est le
mesme portrait, les pieces de monnoyes de Varin et les medailles que
j'ay de ce grand et juste Prince qui a jamais nous doit estre si cher et
une bourse de cent jettons d'argent ou il est representé, et ce que j'ay
de mignatures peintes par ma mere et les portraits de sa chambre.

Onsiemement, je donne et substitue a ma petite fille et unique
heritiére, la Comtesse de Valentinois, tous les portraits que j'ay a la
Ferté et chés moy a Paris qui sont tous de famille, de reconnoissance,
ou d'intime amitié. Je la prie de les tendre et de ne les pas laisser
dans un gardemeuble.

Dousiemement, je donne à mon cousin M. de S. Simon, Evesque de Metz,
tous mes manuscrits tant de ma main qu'autres et les lettres que j'ay
gardées pour diverses raisons desquelles je proteste qu'aucune ne
regarde les affaires de mes biens et Maison.

Treisiemement, je donne et legue à Me de la Lande de present retirée aux
Hospitalieres de Pontoise, quinze cent livres par an sa vie durant.

Quatorsiémement, je lègue quatre cent francs par an leur vie durant
chacun a Lodier, qui a soin de mes livres et qui a déjà un legs de ma
chere espouse, a Piat, mon officier, qui me sert aussy de maistre
d'hostel, a Raimbault, mon valet de chambre, et a Talbot qui a soin de
mes chasses a la Ferté. Deux cent francs par an au dernier vivant a
Tocart et a sa femme chaque année depuis le jour de mon deceds, soit
qu'ils restent concierges du chasteau de la Ferté ou non, et deux cent
francs a Gabrielle Bertaut, sa vie durant, filleule de ma chere espouse,
et actuellement femme de chambre de Me de S. Germain-Beaupré.

Quinsiemement, je legue a Raimbault, mon valet de chambre, outre ce que
je luy ay légué cy dessus, ma garderobe, ma montre d'or, mes tabatieres,
mes croix d'or du S. Esprit et de S. Loüis, excepté le reste de
l'argenterie de ma garderobe, avertissant qu'il faut rendre mon collier
du S. Esprit et la croix qui y pend au grand Tresorier de l'ordre, et la
croix de S. Loüis que le Roy m'a donnée, au bureau de la guerre.

Seisiesmement, je legue une fois payé, trois mil livres au Sr.~Bertrand
que je ne puis trop louer depuis qu'il prend soin de mes affaires, mil
livres au Sr du Nesme, qui a esté mon tres bon et tres fidele maistre
d'hostel et qui l'est a present de M. de Maurepas, mil livres au Sr
Foucault, mon chirurgien, cinq cent francs a Monfort, mon cuisinier, six
cent francs a Broèller mon suisse, autres six cent francs a Contois, mon
laquais, deux cent francs a mon postillon, autant au frotteur, trois
cent francs a Laurent, deux cent francs a Marie qui fait bien les choses
de service dans la Maison, cent francs au garçon de cuisine et quatre
cent francs a mon cocher Fribourg, si on ne lit pas bien parce que j'ay
recrit la somme, c'est quatre cent francs que je luy donne. Declarant
bien expressement que je révoque tous les legs faits a ceux de mes
domestiques actuels qui ne seront plus a moy au jour de mon déceds. Je
suis si content de tous, principalement des principaux, et j'en ay
toujours esté si fidelement et si honnestement servi, que j'ay grand
regret de ne pouvoir le reconnoistre mieux.

Je donne à l'Abbaye de la Trappe le portrait original de leur saint abbé
et Reformateur, et je demande tres instament a tous les Abbés, Religieux
et Solitaires de cette Stemaison leurs prieres et sacrifices pour le
repos de mon ame, de celle de ma tres chere espouse et de tous les
miens.

Je prie Monsieur Daguesseau de Fresne, Conseiller d'Estat ordinaire,
duquel ainsy que de sa famille j'ay toujours receu beaucoup de marques
d'amitié, de voiloir bien m'en donner cette derniere, d'estre
l'Executeur de ce mien testament olographe, et de le faire executer et
accomplir de point en point selon sa forme et teneur, me démettant entre
ses mains de tous mes biens et de tout ce que j'ay en ce monde pour cet
effet. Je le supplie en mesme temps de vouloir bien accepter un de mes
plus beaux et plus agreables tableaux de Raphael qui represente la
SteVierge assise tenant Nostre Seigneur Jesus Christ son divin Fils sur
ses genoux, que je luy legue.

Lequel present testament, écrit de ma main, j'ay pour marque et
témoignage de ma derniere volonté signé de ma main audit lieu, an, mois
et jour que dessus.

\emph{Signé\,:}

(Suivent ces mentions.)

Contrôlé à Paris le 6 mars 1755, reçu soixante livres. \emph{Signé}
illisiblement.

Vu au greffe des insinuations du Châtelet de Paris, sans préjudice des
droits. Ce 6 mars 1755.

\emph{Signé\,:} Levacher, pour M. Thiers.

«\,Il est ainsi en l'original du testament ci-dessus littéralement
transcrit, de M. le Duc de Saint-Simon, décédé à Paris, le deux mars
dix-sept cent cinquante-cinq, déposé pour minute à Me Delaleu, notaire,
aux termes de l'acte d'ouverture dudit testament, dressé par Messire
Dargouger, Conseiller du Roi et Lieutenant civil de la a Prévôté de
Paris, le 2 mars 1755. Le tout étant en la possession de Me Louis
Édouard Dreux, notaire à Paris soussigné, comme successeur médiat dud.
Me Delaleu, ancien notaire à Paris.

«\,Paris, ce dix-neuf avril mil huit cent cinquante-six.
«\,\emph{Signé\,:} Dreux.\,»

\end{document}
